\chapter{The T86 Virtual machine}
At the FIT CTU, in the NI-GEN course students have to write their own compiler.
To ease the code generation part, the T86 architecture was created and a
virtual machine was written as a way to execute programs in that architecture.
The architecture was made as part of a masters thesis \cite{ivo2021tiny}.
This chapter explores the T86 architecture, the virtual machine that was made
for it and explains which parts of the virtual machine will have to be extended
or added to support debugging.
\subsection{T86 ISA}
The name is an abbreviation for Tiny x86, but in some aspects the architecture
is more similar to other architectures like ARM. It is an educational ISA, so
some properties are configurable, like the number of available registers.

The T86 uses a harvard architecture, this means that the data and instructions
are separated from each other. The author doesn't specify the reason for this,
put presumably it to ease the implementation of the virtual machine, since
the memory and instructions can be represented as two separate array-like
members of the virtual machine.

It has the registers we saw on x86\_64, ie. \texttt{PC}, \texttt{SP},
\texttt{BP} and the flags register. The intended roles for these registers are
the same os on x86\_64. It also has other general purpose registers. The number
of these is configurable.

The addressing modes, or what kind of operands instructions can have, include
immediate values, registers and memory accesses. These accesses are not only
immediate or register offsets, but can be combined in various way, like
\texttt{[R0 + R1 * 2]}, or \texttt{[R0 + 10 + R1 * 2]}. The addressing modes
are however not arbitrary, \texttt{[R0 + R1 + R2]} is not correct addressing
mode for T86. For full list refer to \cite{ivo2021tiny}.

The instructions the ISA provides are very similar to what we have already seen
when examining various real world examples. Interesting exceptions to that are
the IO instructions - \texttt{PUTCHAR} and \texttt{GETCHAR}, which allow for
very basic input and output. Also, an \texttt{DBG} and \texttt{BREAK} are
defined. These are used for debugging, but in a very different way that we have
seen in previous sections. We will touch upon them when discussing the virtual
machine implementation, since they are very much tied to it.
