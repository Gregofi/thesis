\chapter{Introduction}
Todays world is practically run by computers. They are everywhere, from our wrists to our cars.
All of these computers car run programs. These programs define what the computer does.
Programs are written by humans (or at least most of them) and are executed by a processor, 
an integral part of every computer. Processors understand several instructions,
which can be executed, and can also read and write to either memory or registers. However, processors can only
understand machine code. For example, very simple machine code instruction is \texttt{0100 0001} (in x86 architecture).
This instruction increases the value which is in register ECX by one. Real programs are made of thousands and thousands of
such instructions. Making sense of these programs for human is very, very difficult. To read a program a programmer would
have to have a mapping from machine code to instruction in his head at least for the most used instructions. Even if the
programmer knew such mappings, sequences of binary would be very hard to read. 

\section{Assembly language}
\begin{figure}\label{fig:assembly-example}
\begin{lstlisting}
PROC:
    PUSH    EBP
    MOV     EBP, ESP
    CMP     [EBP + 8], 0
    JLE     LN2            ; Jump if the EPC + 8 value is smaller than 0
    MOV     EAX, 1
    JMP     LN1
LN2:
    XOR     EAX, EAX
LN1:
    POP     EBP
    RET
\end{lstlisting}
\end{figure}
Assembly language is almost a direct mapping from instruction and operand names to machine code. For example the code \texttt{INC ECX} is the previously mentioned increment by one instruction. This is way more readable for programmer than \texttt{0100 0001}. Understanding instructions by themselves is simpler. But programs are still difficult to read. Consider example on figure \ref{fig:assembly-example}, this code checks if value which is in memory at offset \texttt{[EBP + 8]} is positive and stores $1$ to register \texttt{EAX} if it is positive, or zero if it is not positive. Understanding the instructions one by one is doable. But understanding what this program does as a whole is not apparent at first glance. 

Programs are executed from top to bottom, instruction by instruction. But certain instructions can change this control flow. For example the instruction \texttt{JMP dest}
jumps to a label (\texttt{LN1} in the example) and the execution continues from there. This allows the program to repeat or skip some part of the code.
Notice the JLE instruction. This instruction performs mentioned jump but only if certain other condition holds true.
These instructions can make it harder for programmer to follow the control flow \todo{GOTO statements considered harmful}.
Programs can also contain comments, as is apparent in the example. These bits only serve to make the code easier to understand.
They are left out when the translation to machine code is done.

Assembly programming is still very close to the underlying machine. To program in assembly, you still need to know
how computers work. If you want to create program for different architectures, you need to write it for each of them
separately because different architectures might use different instructions, have different registers or be completely 
different (for example stack based).

\section{High level programming languages}
Aim of these languages is to abstract away the details of a computer. One of the older languages is the C programming language (1972).
Example of a program in C:
\begin{minted}{C}
int positive(int n) {
    if (n > 0) {
        return 1;
    } else {
        return 0;
    }
}
\end{minted}
This program yields same result as the assembly one. Knowing about internals of the computer is no longer needed.
We do not store values into registers, instead they are kept in variables. We can use conditional statements,
which are more concise then using jumps \todo{GOTO considered harmful?}. 
The functions and variables themselves also have names from which
can be apparent what is the purpose of it (contrast to register names like EAX).

However, upon deeper inspection of the language, it does not abstract away everything. C has a feature called
\textit{pointers}. Those are special kind of variables which points into the computer memory\footnote{On modern OS,
a concept named \textit{virtual memory} is used. This maps addresses used by process to physical memory. Pointer values are virtual memory addresses. See \cite{modern-os}.}.
So C does not necessarily abstract everything away. That can sometimes be a good thing, because if you need to 
interact with the computer hardware you don't have to resort to assembly, but can use C. For example the Linux
 Kernel is written mostly in C \cite{linux-source}.

Not everyone needs that sort of intimacy with the hardware. So more and more abstract languages were created.
Here is a program that takes two sequences of ordered numbers and merges these two together so that the result
is still sorted sequence of numbers. It is written in three different languages to illustrate how can they differ.

\begin{figure}[H]\label{fig:merge-c}
    \begin{minted}{c}
    int* merge (int *src1, size_t len1, int *src2, size_t len2) {
        int *dest = malloc((len1 + len2) * sizeof(*dest));
        size_t i = 0;
        while (src1 != src1 + len1 && src2 != src2 + len2)
            dest[i++] = src1 <= src2 ? *(src1++) : *(src2++);
        while (src1 != src1 + len1)
            dest[i++] = *(src1++);
        while (src2 != src2 + len2)
            dest[i++] = *(src2++);
        return dest;
    }
    \end{minted}
    \caption{Merge of two sequences in C}
\end{figure}

Here, the array where we will store the result is dynamically allocated using
the \textit{malloc} functions. This function takes how many bytes it should
allocate as an argument and returns continuous memory of that size. The two
sequences to be merged are pointed at by variables. Since pointers are just
memory addresses, we need to know how long the sequences are and pass that
along with the pointers.

\begin{figure}[H]
    \begin{minted}{python}
    def merge(l1, l2):
        result = []
        idx1, idx2 = 0, 0
        while idx1 < len(l1) and idx2 < len(l2):
            result.append(l1[idx1] if l1[idx1] < l2[idx2] else l2[idx2])
        result.extend(l1[idx1:])
        result.extend(l2[idx2:])
        return result
    \end{minted}
    \caption{Merge of two sequences in Python}
\end{figure}

Python goes further. The code is similar but the length of the sequence (python calls it a \textit{list})
is baked into the sequence itself. Also, we do not have to specify how long will the list be upfront,
Python manages everything for us. In C, we had to have a while cycle that appended
the rest of the sequence to the output. In Python, lists (and other objects) have methods,
like the \textit{extend}. Notice how we had to specify what type the variables are in C.
This is not needed in Python, since the language is dynamically typed, whereas 
C is statically typed. This has its own pros and cons, which is briefly discussed in \todo{Compilling} chapter.

Python and C are so called imperative languages. These languages are 
made of statements which are executed one by one (similarly to assembly language).
The languages are similar in terms of how an algorithm will be written in either of them.
However, there are also other paradigms of programming, like object oriented or functional programming.

\begin{figure}[H]
    \begin{minted}{haskell}
    merge :: Ord a => [a] -> [a] -> [a]
    merge [] a = a
    merge a [] = a
    merge (h1:t1) (h2:t2) | h1 < h2 = h1:merge t1 (h2:t2)
                          | otherwise = h2:merge (h1:t1) t2
    \end{minted}
    \caption{Merge of two sequences in Haskell}
\end{figure}

Haskell is a functional programming language. As is apparent, functional programming is very different from imperative,
altrough the two words sometimes intertwine, for example through higher order functions. In the rest of this thesis, we will
mainly talk about imperative languages, but will sometimes mention how things differ for functional languages.

\section{Debugging}
Programs are mostly written by humans, and humans make mistakes \cite{human-error}. These mistakes may be syntactic ones, like
forgetting to put a semicolon at the end of a statement. This mistake makes it impossible for the program to be run. However,
programs can still be syntactically valid and still have errors in them. Consider the merge program in C (figure \ref{fig:merge-c}).
Lets write the allocation of array like this: \texttt{malloc(len1 + len2)}. We allocated bytes equal to number of elements
but we haven't incorporated their size (integers are often four bytes long)! Thus, the result would not fit into the allocated
array (unless both arrays were empty) and the program would most likely crash.

This is called undefined behaviour \cite{undefined-behavior}.
Everything that happens in the execution next is not guaranteed by the C standard. It may crash, or produce different
results than expected, or it may run perfectly fine. It also may run perfectly fine
when executed for the first time and crash for the second time.
Errors like this can be very difficult to catch, because they may show up only some of the time. Some languages do prevent this,
when you try to index an array out of its bounds the program will abort immediately with a descriptive error message,
but they may have their own different quirks. Some programming languages and paradigms are less error prone, but none 
can eliminate errors entirely \cite{bugs-by-language, bugs-by-language-2}.

Our merge function (figure \ref{fig:merge-c}) worked with invariant that the two arrays it received are sorted.
If they are not, then the resulting array will not be sorted. The output is wrong, but the merge function is not
at fault. Instead, we would have to inspect from which place the unsorted data came to be. We will call
errors in the code that manifest at run time \textbf{bugs}\footnote{The term \textit{bug} actually comes from a real bug that got stuck in relays.
They literally had to debug the machine by taking the bug out.}. The process of finding those error shall be
called \textbf{debugging}.

There are already many research papers which presents methods on preventing errors, or how to find them when they 
happen \cite{software-debugging-testing-verification, Debugging-difference-between-expert-and-novice, zeller2009programs, debugging-inputs}.
We will instead focus on a tool that helps with inspecting the program at runtime and finding bugs.

\section{Debugger}
Debugger is a program which can inspect another program. They often have complete control of the program.
The program being debugged will be called a \textit{debugee}. In this section, we will talk about debuggers
which works with high level programming languages. Later we will talk about assembly level debuggers and
how they tie into high level language debuggers.

Debugger are able to inspect the state of the program - value of its variables. Debuggers are also able to control
the flow of the program. They allow \textit{breakpoints} to be set at each line of the source code\footnote{Advanced debuggers allows breakpoints to be set inside expressions.
This is especially important for functional languages, as their functions often consists of one big expression.}.
When the program is about to execute the line of code with the breakpoint. The control is passed back to the
debugger and user can inspect the state of the program at that line. There are also conditional breakpoints,
which only triggers when some condition holds (for example the value of variable \texttt{i} must be equal to zero).

Finally, debuggers also allow \textit{stepping}. This also modifies the control flow of the program.
\begin{itemize}
    \item step in - Executes current statement and stops on the next one. If current statement is a function call then
                    it will be executed and program will be paused on the first statement in that function.
    \item step over - Same as step in, but if current statement is a function call then the program will
                      be paused on the next statement after the call.
    \item step out -  Executes as much as needed to return from current function. Stops on next statement that
                      should be executed after the function returns.
\end{itemize}

The goal of this thesis will be to implement a debugger for special architecture. But before we get it on, there
is one more thing we need to tackle.

\section{Compilers}
When we talked about evolution of programming from machine code to assembly to higher level languages, we haven't
talked about how they are executed. Machine code can be directly executed by processor, as we said, it is a sequence
of binary. Assembly is text, processors don't understand text. But assembly can be mapped to machine code almost 
1:1\footnote{There are some exceptions, like labels. But translating them is not very difficult.}.

However, high level programming languages do not map 1:1 to assembly. Some are close to it, like C, while others
are miles away, like Haskell. But as was said, processors understand only machine code. To this end, programs that
can translate source code into machine code, were created. They are called compilers and the translation process is
called compiling. For example, for the C language one might use the GCC or Clang compilers.
On figure \ref{fig:compiler-structure} can be seen basic structure of a compiler \cite{dragon-book}. 

\tikzstyle{compilerblock} = [rectangle, draw, minimum width=6cm, minimum height=1cm] 
\tikzstyle{tables} = [rectangle, draw, minimum width=4cm, minimum height=1cm] 
\begin{figure}\label{fig:compiler-structure}
    {\centering
    \begin{tikzpicture}
    \node (lexer)[compilerblock]{Lexical analyzer};
    \node (syntax)[compilerblock,below=of lexer]{Syntactic analyzer};
    \node (semantic)[compilerblock,below=of syntax]{Semantic analyzer};
    \node (imc)[compilerblock,below=of semantic]{Intermediate Code Generator};
    \node (gen)[compilerblock,below=of imc]{Code Generator};
    \node (symbol)[tables, left=of semantic]{Symbol table};
    \draw[->] (lexer) -- node[below] {} (syntax);
    \draw[->] (syntax) -- node[below] {} (semantic);
    \draw[->] (semantic) -- node[below] {} (imc);
    \draw[->] (imc) -- node[below] {} (gen);
    \end{tikzpicture} 
    \par}
    \caption{Simplified structure of a compiler. Some parts were left out, like optimizations.}
    \label{fig:compiler_tikz}
\end{figure}

\subsection{Lexical analyzer}
The lexical analyzer groups separate symbols into groups. For example the code
\begin{minted}{c}
foo = bar(1 + 2);
\end{minted}
might be translated into tokens like this
\begin{lstlisting}[stringstyle=\color{black}]
<id:"foo"> <assignment-operator> <id:"bar"> 
<left-bracket> <int-number:1> <plus-operator> 
<int-number:2> <right-bracket> <semicolon>
\end{lstlisting}
The Syntactic analyzer then works with these tokens.

\subsection{Syntantic and semantic analyzer}
Syntactic analysis accepts tokens and processes them into other intermediate representation. This is
most often an abstract syntax tree (abbr. AST, figure \ref{fig:ast}). It also checks that the source code complies to the grammar of the language.
Semantic analysis then checks that the program is semantically consistent. For example that used variable
has been declared before.

\begin{figure}\label{fig:ast}
    \centering
    \begin{tikzpicture}[,shorten >=1pt,node distance=1.8cm,on grid,initial/.style={}]
    \node (assignment) {$=$};
    \node (foo) [below left =of assignment] {id:foo};
    \node (bar) [below right =of assignment] {call:bar};
    \node (plus) [below right=of bar] {$+$};
    \node (one) [below left =of plus] {$1$};
    \node (two) [below right =of plus] {$2$};
    
    \draw[-, above, scale=0.7] 
    (assignment)   edge node[scale=0.7, left, yshift=0.1cm] {lhs}  (foo)
     (assignment)  edge node[scale=0.7, right, yshift=0.1cm] {rhs}  (bar)
     (bar)         edge node[scale=0.7, right, yshift=0.1cm] {expr} (plus)
     (plus)        edge node[scale=0.7, right, yshift=0.1cm] {rhs}  (two)
     (plus)        edge node[scale=0.7, left, yshift=0.1cm] {lhs}  (one);
    \end{tikzpicture}
    \caption{Simplified example of an abstract syntax tree.}
    \label{fig:astgraph}
\end{figure}
 
\subsection{Intermediate code generation}
This part converts AST into some other representation, most commonly called IR\footnote{IR means intermediate representation.
AST is also intermediate representation, but if we use IR we mean this one.}. IR is closer to machine code, to be easily translated,
but retain some properties that makes it easier to work with it. There are many types of IR. One of the most popular compilers, LLVM, uses
single static assignment (SSA) \cite{llvm}. Example of LLVM IR can be found on figure \ref{fig:llvm-ir-example}. Compilers perform most
optimizations on this intermediate representation. 

\begin{figure}\label{fig:llvm-ir-example}
    \begin{minted}{llvm}
        define dso_local i32 @_Z6squarei(i32 %0) {
          %2 = alloca i32, align 4
          store i32 %0, i32* %2, align 4
          %3 = load i32, i32* %2, align 4
          %4 = load i32, i32* %2, align 4
          %5 = mul nsw i32 %3, %4
          ret i32 %5
        }
    \end{minted}
    \caption{Simplified example of LLVM IR.}
\end{figure}

\subsection{Code generation}
Here, IR is translated directly to the target machine code or possibly assembly. Even though IR can seem very similar to assembly,
there are still some things to take care of. For example SSA IR doesn't have registers, it uses unlimited number of variables.
Other architectures might have some other traits that differ it from the IR and they all have to be accounted for when generating code.

\subsection{Modularity of compilers}
The main advantage of using an IR is that there is a common ground for every language. Imagine we write a compiler for the C language.
We need to write all five parts from figure \ref{fig:compiler-structure}. If we later decided that we also want to create a compiler
for Haskell, we just need to write everything up to the IR translation. Once we can translate Haskell into the IR, we can reuse the
previous part of the compiler to compile to machine code! This also works the other way around. If we compiled IR to the machine code
that works with the x86 architecture, and we want to compile to ARM, we just need to create the code generation part for the ARM architecture,
no need to write whole compiler. Also, most of the optimizations are done on the IR level, this also saves a lot of development time.
The parts of the compiler which are dependent on the source language are called \textbf{frontend} (Syntax, Semantic and IR translation), the parts that are dependent on
the target are called \textbf{backend} (Code generation).

This is widely used in practice. The LLVM \cite{llvm} project is a compiler backend. It uses its own IR (as was mentioned on figure \ref{fig:llvm-ir-example}).
It can compile this IR into many targets, including x86, ARM and Spark \todo{Ocitovat}. The \textit{Clang} project is a compiler frontend for C, C++ and Objective-C languages.
It translates these languages to the LLVM IR. Other frontends for LLVM also include \textit{ghc}, which is a Haskell compiler, or \textit{rustc}, which is a Rust compiler.
With LLVM, creating new programming language comes down to parsing it into an AST and transforming that AST into the LLVM IR.

\subsection{Interpreting programs}
Not all languages are compiled. Imagine a program which can evaluate arithmetic expressions, each phone nowadays has a program like this.
We don't have to stop there. Moving this up a notch, we can create a program that reads source code and executes it.
This is what interpreting means. Dynamically typed languages tend to be interpreted~\cite{python, lua, javascript}\todo{Instead of languages, cite some relevant source}, but it is not a rule~\cite{scala}. Since the interpreter is a program, it is another layer of abstraction. This can make
the resulting languages sometimes more abstract then the compiled ones, sometimes at the cost of performance~\cite{jit}.
Interpreters still \textit{compile} the code into some intermediate representation, but it's not compiled down to machine code, instead that IR is run by a program\footnote{Nowadays, interpreters use JIT compilation, which compiles some of the code some of the time into machine code~\cite{jit}}.

\subsection{Debugging source code}
As we mentioned previously, processors only understand machine code. To run a program written in high level programming language, we need to compile it. Debugger can then inspect the value of variables when the program is running or stop at specific lines. But the machine code doesn't have any variables, nor lines of source code. How is the debugger able to know on which instruction it should put the breakpoint? Compilers must work together with debuggers. They must store this information somewhere, so that debugger knows that line $x$ in source code corresponds to instruction $y$ in machine code. These \textit{debugging informations} are often not compiled with the program by default, for example to compile them with `clang` compiler one needs to use the `-g` flag. If not, then its still possible to debug it, but one will only see the instructions and raw memory, not variables and lines of source code.

\section{Teaching compilers}
Many schools about computer science have compiler course. At Faculty of Information Technologies, there is one too, called \textit{Code Generators} (NI-GEN). In this course, student is tasked to build a simple compiler for a C like language. The target of the compiler is T86 architecture. This is also an abbreviation, it stands for \textit{tiny x86}. This architecture does not have a processor that implements it, instead a virtual machine (or interpreter, if you will) was created for it. The architecture is supposed to ease the code generation and let the students focus on the more interesting parts of the compiler, like register allocation or optimization, instead of the nitty gritty detail of x86 (or different architecture, should the students choose it).

T86 however has one problem. It doesn't have a debugger. So if a compiler of some student generates code badly, it takes a non-trivial amount of effort to find the error. T86 offers some very light debugging capabilities, but they are very far from real debugging. Also, compiling debugging informations is not taught at all in the NI-GEN course. If a debugger was provided to the students, it might be incetiving to compile such information to ease their lives later, when they will need to find errors in their compilers which will inevitably happen.

\section{Goals of the thesis}
The primary goal is to create a debugger for the T86 virtual machine which allows to debug both on the machine code level and at the source code level. The debugger should be extensible enough to work with intermediate representation also. This will require non-trivial changes in the T86 virtual machine source code. The students compilers will also have to generate debugging information. Format of the debugging information should be so that it's easy for students to generate.

\section{Structure of the thesis}
\begin{enumerate}
    \item The \textit{Introduction} describes the motivation behind the thesis and basic terms with which should the reader be familiar.
    \item \textit{Debuggers - On the shoulders of giants} describes how are debuggers implemented and what support is needed on various levels (OS, processors, compilers) for them to work.
    \item \textit{T86 Virtual Machine} describes the T86 architecture and how are some of the parts implemented.
    \item \textit{Implementation} describes how the implementation is done. It is divided into three main parts, first is what changes and additions were necessary to the T86 VM. Then it is described how the debugger itself is implemented. Finally, a proof of concept compiler is created to demonstrate how the debugger works and that it is indeed simple to generate debugging information for it.
    \item \textit{Conclusions} summarizes the result of the thesis and speaks of possible future work. 
\end{enumerate}

