\chapter{Evaluation}
In this chapter, we evaluate the debugger, comparing its usage to the gdb and
lldb debugger. We will also measure the impact of the debugging onto the
performace.

\section{The development process}
The development of the thesis was done in a github repository. The power of
github was leveraged not only for keeping a history, but also for writing
issues, planning the development, or making sure that the repository is in
consistent and working state via github actions, which runs tests before pull
request is accepted. This repository can be used going forward.

The project itself contains many tests. The code is first tested via many unit
tests. The \texttt{GoogleTest} framework was used. The unit tests cover almost
all parts of the code.

\section{Usage}
There was an attepmt to follow the interface of GDB closely, so that the users
are somewhat familiar with the debugger before even running it. However, we had
to diverge on some features. For example, the GDB uses \texttt{stepi} for
single stepping. This has a common prefix with \texttt{step}. We however expect
our users to use the assembly level debugging more often than source level, so
we choose the \texttt{istep} command. The two commands have no shared prefix
and so typing \texttt{is} is enough.

Also, the program is run via \texttt{run}, this starts the VM but the program
is paused. In GDB, one has to use the \texttt{start} command, \texttt{run} runs
the program without stopping at the beginning. When we did a brief user
testing, it was confusing for the user. However, renaming the command
\texttt{run} to \texttt{start} would cause it to have the same prefix as
\texttt{step}. Considering this, we've made the decision to leave the command
as \texttt{run}.

There are other minor things, most of them come from the fact that we give more
priority to assembly level debugging, whereas GDB focus more on the source
level.
