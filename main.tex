% arara: xelatex
% arara: xelatex
% arara: xelatex

% options:
% thesis=B bachelor's thesis
% thesis=M master's thesis
% czech thesis in Czech language
% english thesis in English language
% hidelinks remove colour boxes around hyperlinks

\documentclass[thesis=M,english]{FITthesis}[2019/12/23]

\usepackage[utf8]{inputenc}
\usepackage{amsmath} %advanced maths
\usepackage{amssymb} %additional math symbols
\usepackage{lipsum,tikz}
\usepackage[cache=false, outputdir=out]{minted}
\usepackage{listings} % typesetting of sources
\usepackage[fontsize=11pt]{fontsize}
\usepackage{xcolor}
\usepackage{float}
\usepackage{fancyvrb}
\usepackage{todonotes}
\usepackage{pdfpages}

\usemintedstyle{borland}
\definecolor{codegreen}{rgb}{0,0.6,0}
\definecolor{codegray}{rgb}{0.5,0.5,0.5}
\definecolor{codepurple}{rgb}{0.58,0,0.82}
\definecolor{backcolour}{rgb}{0.95,0.95,0.92}

\lstdefinestyle{mystyle}{
    commentstyle=\color{codegreen},
    keywordstyle=\color{magenta},
    numberstyle=\tiny\color{codegray},
    stringstyle=\color{codepurple},
    basicstyle=\ttfamily\footnotesize,
    breakatwhitespace=false,
    breaklines=true,
    captionpos=b,
    keepspaces=true,
    showspaces=false,
    showstringspaces=false,
    showtabs=false,
    tabsize=4
}

\lstset{style=mystyle}

\usepackage{dirtree} %directory tree visualisation

% tikz thingies
\usetikzlibrary{calc,positioning,arrows,shapes.geometric}
\tikzstyle{process} = [rectangle, 
minimum width=3cm, 
minimum height=1cm, 
text centered, 
text width=3cm, 
draw=black, 
]

\usetikzlibrary{arrows.meta}
\tikzset{>={Latex[width=3mm,length=3mm]}}

% list of acronyms
% \usepackage[acronym,nonumberlist,toc,numberedsection=autolabel,nomain]{glossaries}
%% \iflanguage{czech}{\renewcommand*{\acronymname}{Seznam použitých zkratek}}{}
% \makeglossaries

\newcommand{\tg}{\mathop{\mathrm{tg}}} %cesky tangens
\newcommand{\cotg}{\mathop{\mathrm{cotg}}} %cesky cotangens

% % % % % % % % % % % % % % % % % % % % % % % % % % % % % % % % % % % 
% % % % % % % % % % % % % % % % % % % % % % % % % % % % % % % % % % % 
\department{Department of theoretical computer science}
\title{Tiny86 Debugger}
\authorGN{Filip} %author's given name/names
\authorFN{Gregor} %author's surname
\authorWithDegrees{Bc. Filip Gregor} %author's name with academic degrees
\author{Filip Gregor} %author's name without academic degrees
\supervisor{Ing. Petr Máj}

\acknowledgements{First and foremost, thanks to my parents, who have always
supported me and enabled me to pursue my dreams, whatever they may be. I would
also like to thank my supervisor, who spent an ungodly amount of time and
patience to break through my stubbornness and make this thesis what it is. Last
but not least, thanks to my girlfriend, Jana. Without her support, this work
would never have come to be.}

\abstractCS{
V mnoha univerzitních kurzech se vyučuje návrh překladače. Jeden takový předmět
se vyučuje na FIT ČVUT. Studenti mají za úkol vytvořit kompilátor z jazyka
TinyC podobnému jazyku C do architektury Tiny x86. Tato architektura nemá žádný
procesor, který by ji implementoval. Místo toho používá virtuální stroj.

Aby byl programovací jazyk použitelný, potřebuje vhodné nástroje, například
ladicí programy. Aby ladící program fungovaly, musí překladač emitovat ladicí
informace popisující vazby mezi generovaným kódem a zdrojovým kódem. Virtuální
stroj pro Tiny x86 nenabízí téměř žádnou podporu pro ladění. Studenti tedy
nemají žádný důvod generovat ladicí informace.

Tato práce popisuje, jak je podpora ladění poskytována na různých úrovních,
jako je procesor, operační systém a překladač. Popisuje také, jak vytvořit
velmi jednoduchý ladicí program za použití rozhraní systému Linux a Windows.
Dále je popsána podpora ladění virtuálního stroje Tiny x86, která byla v rámci
práce přídána. Ladící podpora byla inspirována rozhraním které nabízí systém
Linux. Nakonec je vytvořen debugger na úrovni assembleru a zdrojového kódu
inspirovaný návrhem ladících programů z reálného světa.
}
\abstractEN{
Many university courses teach compiler design. One such course is taught at FIT
CTU. The students are tasked to create a compiler from a C-like language called
TinyC into the Tiny x86 architecture. This architecture does not have any
processor that implements it. Instead, it uses a virtual machine.

To make a programming language usable, it needs proper tooling, like debuggers.
For debuggers to work, the compiler must emit debugging information describing
the connections between generated machine code and the source code. The virtual
machine has almost no debugging support. Hence the students do not have any
reason to generate debugging information.

That is about to change. This thesis describes how debugging support is
provided at various levels, like the central processing unit and the operating
system. It also describes how to create a very simple debugger on Linux. We
then add debugging support to the Tiny x86 virtual machine itself, which is
partially inspired by the studied Linux debugging interface. Finally, we create
an assembly and source-level debugger inspired by real-world debuggers' design.
}

\placeForDeclarationOfAuthenticity{Prague} %where you have signed the declaration
\keywordsCS{Ladění, ladící program, Překladač, Implementace ladícího programu, LLVM,
Linux, Windows, Tiny x86, Podpora pro ladění, Chyby v programech}
\keywordsEN{Debugging, Debugger, Debug, Debugger implementation, Compiler, LLVM,
Linux, Windows, Tiny x86, Debugging support, Errors in programs}
\declarationOfAuthenticityOption{4} %select as appropriate, according to the desired license
\website{https://github.com/Gregofi/tinyverse-monorepo}

\begin{document}

% \newacronym{LZW}{LZW}{Lempel Ziv Welch}
% \newacronym{RLE}{RLE}{Run-Length Encoding}

\chapter{Introduction}
Todays world is practically run by a computers. They are everywhere, from our wrists to our cars.
All of these computers car run some sort of programs. These programs define what the computer does.
Programs are executed by a processor, an integral part of every computer. Processors have several instructions,
which can be executed, and can also read and write to either memory or registers. However, processors can only
understand machine code. For example, very simple machine code instruction is \mintinline{0100 0001} (in x86 architecture).
This instruction increases the value which is in register ECX by one. Real programs are made of thousands and thousands of
such instructions. Making sense of these programs for human is very, very difficult. To read a program a programmer would
have to have a mapping from machine code to instruction in his head at least for the most used instructions. Even if the
programmer knew such mappings, sequences of binary would be very hard to read. 

\section{Assembly language}
Assembly language is almost a direct mapping from instruction and operand names to machine code. For example the code \mintinline{INC ECX}
is the previously mentioned increment by one instruction. This is way more readable for programmer than \mintinline{0100 0001}.
Understanding instructions by themselves is simpler. But programs are still difficult to read. Consider following example:
\begin{lstlisting}
PROC:
    PUSH    EBP
    MOV     EBP, ESP
    CMP     [EBP + 8], 0
    JLE     LN2            ; Jump if the EPC + 8 value is smaller than 0
    MOV     EAX, 1
    JMP     LN1
LN2:
    XOR     EAX, EAX
LN1:
    POP     EBP
    RET
\end{lstlisting}
This code checks if value which is in memory at offset \mintinline{[EBP + 8]} is positive and stores $1$ to register \mintinline{EAX} if it is positive, or zero
if it is not positive. Understanding the instructions one by one is doable. But understanding what this program does as a whole is not apparent at first glance. 

Programs are executed from top to bottom, instruction by instruction. But certain instructions can change this control flow. For example the instruction \mintinline{JMP dest}
jumps to a label (\mintinline{LN1} in the example) and the execution continues from there. This allows the program to repeat or skip some part of the code.
Notice the JLE instruction. This instruction performs mentioned jump but only if certain other condition holds true.
These instructions can make it harder for programmer to follow the control flow \todo{GOTO statements considered harmful}.
Programs can also contain comments, as is apparent in the example. These bits only serve to make the code easier to understand.
They are left out when the translation to machine code is done.

Assembly programming is still very close to the underlying machine. To program in assembly, you still need to know
how computers work. If you want to create program for different architectures, you need to write it for each of them
separately because different architectures might use different instructions, have different registers or be completely 
different (for example stack based).

\section{High level programming languages}
Aim of these languages is to abstract away the details of the computer. One of the older languages is the C programming language (1972).
Example of a program in C:
\begin{minted}{C}
int positive(int n) {
    if (n > 0) {
        return 1;
    } else {
        return 0;
    }
}
\end{minted}
This is the same program as the assembly one. Knowing about internals of the computer is no longer needed.
We do not store values into registers, instead they are kept in variables. We can use conditional statements,
which are more concise then using jumps. The functions and variables themselves also have names from which
can be apparent what is the purpose of it (contrast to register named EAX).

However, upon deeper inspection of the language, it does not abstract away everything. C has a feature called
\textit{pointers}. Those are special kind of variables which points into the computer memory\footnote{On modern OS,
a concept named virtual memory is used. This maps addresses used by process to physical memory. Pointer values are virtual memory addresses. See \cite{modern-os}.}.
So C does not necessarily abstract everything away. That can sometimes be a good thing, because if you need to 
interact with the computer hardware you don't have to resort to assembly, but can use C. For example the Linux
operating system Kernel is written in C.

Not everyone needs that sort of intimacy with the hardware. So more and more abstract languages were created.
Here is a program that takes two sequences of ordered numbers and merges these two together so that the result
is still sorted sequence of numbers. It is written in three different languages to illustrate how can they differ.
\begin{minted}{c}
int* merge (int *src1, size_t len1, int *src2, size_t len2) {
    int *dest = malloc((len1 + len2) * sizeof(*dest));
    size_t i = 0;
    while (src1 != src1 + len1 && src2 != src2 + len2)
        dest[i++] = src1 <= src2 ? *(src1++) : *(src2++);
    while (src1 != src1 + len1)
        dest[i++] = *(src1++);
    while (src2 != src2 + len2)
        dest[i++] = *(src2++);
    return dest;
}
\end{minted}
Here, the array where we will store the result is dynamically allocated using
the \textit{malloc} functions. This function takes how many bytes it should
allocate as an argument and returns continuous memory of that size. The two
sequences to be merged are pointed at by variables. Since pointers are just
memory addresses, we need to know how long the sequences are and pass that
along with the pointers.

\begin{minted}{python}
def merge(l1, l2):
    result = []
    idx1, idx2 = 0, 0
    while idx1 < len(l1) and idx2 < len(l2):
        result.append(l1[idx1] if l1[idx1] < l2[idx2] else l2[idx2])
    result.extend(l1[idx1:])
    result.extend(l2[idx2:])
    return result
\end{minted}

Python goes further. The code is similar but the length of the sequence (python calls it a \textit{list})
is baked into the sequence itself. Also, we do not have to say how long will the list be,
Python manages everything for us. In C, we had to have a while cycle that appended
the rest of the sequence to the output. In python, lists (and other objects) have methods,
like the \textit{extend}. Notice how we had to specify what type the variables are in C.
This is not needed in Python, since the language is dynamically typed, whereas 
C is statically typed. This has its own pros and cons, which is briefly discussed in \todo{Compilling} chapter.

Python and C are so called imperative languages. These languages are 
made of statements which are executed one by one (similarly to assembly language).
The languages are similar in terms how an algorithm will be written in either of them.
However, there are also other paradigms of programming, like object oriented or functional programming.

\begin{minted}{haskell}
merge :: Ord a => [a] -> [a] -> [a]
merge [] a = a
merge a [] = a
merge (h1:t1) (h2:t2) | h1 < h2 = h1:merge t1 (h2:t2)
                      | otherwise = h2:merge (h1:t1) t2
\end{minted}
Haskell is a functional programming language. As is apparent, functional programming is very different from imperative,
altrough the two words sometimes intertwine, for example through higher order functions. In the rest of this thesis, we will
mainly talk about imperative languages, but will sometimes mention how things differ for functional languages.

\section{Debugging}
Even though the languages keep getting better and better.

\section{Compilers}


\chapter{Debugging support}

\begin{quote}
  \textit{Debugging is twice as hard as writing the code in the first place. Therefore, if you write the code as cleverly as possible, you are, by definition, not smart enough to debug it.}\begin{flushright}
    \tiny{Brian W. Kernighan}
  \end{flushright}
\end{quote}

We already mentioned in the first chapter that to debug programs written in high level programming languages, we need the compiler to emit debugging information.
But debugging support must also be provided by operating system (if there is one) and by processor itself.
However, we can still debug programs in machine code. We briefly mentioned this in chapter 1 \todo{ref},
however, we won't see source code, but only assembly. This can still be useful, for example
for reverse engineering. And, as was already hinted, source level debugging is built upon
assembly debugging. In this chapter, we will describe on which levels must assembly level debugging
be supported. Additionaly, we'll discuss how are compilers and debuggers able to allows us 
to debug source code althrough the programs are still machine code programs. 

\section{Support on CPU level}
In first chapter it was said that CPU can only execute machine code which is made of instructions
and that it has certain registers. Which instructions and registers the CPU has can differ from CPU to CPU.
This is specified by \textit{Instruction Set Architecture} (ISA) \cite{aps-isa}. It is an abstract interface between
the hardware and lowest level software (machine code). It contains all information needed to write a program in machine
code. In general, ISA specifies following:
\begin{itemize}
    \item Set of machine code instructions - Specifies instructions the ISA has and what operands each instruction has.
    \item Register set - Which registers the ISA has\footnote{Strictly speaking ISA doesn't have to use registers. It's possible to use only stack or accumulator, but most used ISAs use registers, so we'll ignore those architectures. }.
    \item Addressing modes - Possible methods to refer to memory or register.
\end{itemize}
There are other specification, however they are not relevant to this thesis. For each instruction and operand there is
specified how they should be encoded into binary (remember, that's what the CPU can understand).
CPU then \textit{implements} some ISA. If two different CPUs implements the same ISA then they should be able to run the same machine code program.
For example, most PC use the \textit{x86} architecture \cite{aps-isa}, althrough the ARM architecture is also seeing use in personal computers, for example the Apple Sillicon is of ARM architecture.

The x86 architecture is so called \textit{Complex Instruction Set Architecture} (CISC).
It contains many instructions that do many things at once, have varying length and takes multiple clock cycles to complete \cite{intel-manual}.
On the other hand the ARM architecture is \textit{Reduced Instruction Set Architecture} (RISC).
The number of instructions is smaller, they are intended to be small building blocks from which complex operations may be created by using many of them. Each instruction in RISC also has same length. Both architectures have their pros and cons, althrough some literature suggest that in modern days the choice of architecture is irrelevant if one is only considering performance and power consuption  \cite{riscvscisc1, riscvscisc2}. Unless specified otherwise the rest of this chapter will be talking about x86. This is because T86 \todo{ref} is loosely based on x86, so it is most relevant for us.

In first chapter we briefly mentioned that machine code programs can instead be written in Assembly language.
Assembly is almost 1:1 mapping to machine code. When showing programs, we will show them in assembly.
In figure \ref{fig:assembly-example2}, we present another example of a program that was compiled from C
to assembly of the x86 architecture.
As seen, instructions have various operands. Most often registers (\texttt{RBP, RSP, EAX}), memory (\texttt{[rbp-4]} is reference to memory at address which
is in register \texttt{rbp} minus $4$), or labels (like \texttt{L2}). Labels are not part of machine code, instead
memory address has to be provided. This is a small part where assembly and machine code differ.
For detailed overview of the x86 instructions see \cite{intel-manual}.

\begin{figure}\label{fig:assembly-example2}
    \begin{lstlisting}
max:
    push    rbp
    mov     rbp, rsp
    mov     QWORD PTR [rbp-24], rdi
    mov     DWORD PTR [rbp-28], esi
    mov     rax, QWORD PTR [rbp-24]
    mov     eax, DWORD PTR [rax]
    mov     DWORD PTR [rbp-4], eax
    mov     DWORD PTR [rbp-8], 1
    jmp     .L2
.L3:
    mov     eax, DWORD PTR [rbp-8]
    cdqe
    lea     rdx, [0+rax*4]
    mov     rax, QWORD PTR [rbp-24]
    add     rax, rdx
    mov     eax, DWORD PTR [rax]
    cmp     DWORD PTR [rbp-4], eax
    cmovge  eax, DWORD PTR [rbp-4]
    mov     DWORD PTR [rbp-4], eax
    add     DWORD PTR [rbp-8], 1
.L2:
    mov     eax, DWORD PTR [rbp-8]
    cmp     eax, DWORD PTR [rbp-28]
    jl      .L3
    mov     eax, DWORD PTR [rbp-4]
    pop     rbp
    ret
    \end{lstlisting}
    \caption{Compiled C program with GCC 9.4 compiler as x86 assembly.}
\end{figure}

\subsection{Registers}

The x86 architecture has a set of general purpose registers.
Some of these are
\begin{itemize}
    \item RAX - Accumulator for operands and results data,
    \item RCX - Counter for string and loop operations,
    \item RSP - Stack pointer,
    \item RBP - Pointer to data on the stack.
\end{itemize}
The names and number of general purpose registers change based on bit mode. 64-bit (also named x86-64) mode has 16 of them, while 32-bit has 8.
Althrough the \texttt{RSP} and \texttt{RBP} are called general purpose they are often only used for pointing at the top of the stack,
resp. to the base of the stack. Stack is special part of program memory, it mostly has LIFO
semantics\footnote{For example the \texttt{mov eax, DWORD PTR [rbp - 4]} does not respect
LIFO semantics, because it reads directly from the stack and not from the top.}
It can be used to store intermediate result, arguments to functions, return address etc.
This register is weird in a sense that it has this very special purpose but is still considered part of
the general purpose registers~\cite{intel-manual}. One might use it for storing calculations, but it would make
rest of the instructions that work with stack behave unexpectedly.

Instruction pointer register (RIP on x86-64, EIP on x86) contains address of the current instruction to be executed. As we mentioned
in Introduction \todo{ref}, programs are executed sequentially from top to bottom, with certain instructions
having the ability to change the control flow. When an instruction get executed, the size of the instruction
will be added to the value in RIP register. This will advance the instruction pointer to the next instruction.
Or, if the instruction changes control flow, the value in instruction pointer will be changed to the destination
of the instruction. The register can also be changed directly.

Another interesting register is the \texttt{EFLAGS} register. The register contains group
of flags, which can alter various behavior of the CPU, or the CPU itself sets them
as result of some instruction. For example the instruction \texttt{cmp} compares its two operands
and if they are the same the \textit{zero} flag in the \texttt{EFLAGS} register will be set.

\subsection{Interrupts}
Interrupt is a special request to the CPU to stop execution of current program and to quickly react to
the reason that caused the request \cite{aps-interrupts}. Example of such event can be keyboard press or error in an program (division by zero).
There are two main categories \cite{intel-manual}
\begin{itemize}
    \item An \textbf{interrupt} is an asynchronous\footnote{Meaning that the interrupt may happen when another instruction is being processed (not on the CPU clock edge).} event that is typically triggered by an Input/Output (IO) device.
    \item An \textbf{exception}\footnote{Unfortunately, this term will become quite overloaded in this thesis.} is a synchronous event that is generated when the processor detects one or more
          predefined conditions when executing an instruction. These are further divided into three classes: faults, traps and aborts.
\end{itemize}

When an interrupt or exception happens, the processor halts execution of current program and switches to specific
interrupt handler. Interrupt handler is just another sequence of instructions that handles the interrupt.
Example of an exception is the \texttt{INT3} instruction.
When this instruction is executed an interrupt is generated. This instruction is specifically meant to be used as a breakpoint.
We can supply code that will be responsible for handling the breakpoint as the interrupt handler.
However, on modern PCs a Operating System (OS) is governing the PC. Alas, we cannot touch the interrupt handler directly.
Instead, an OS is going to have to provide another layer of support for debugging.

Recall the EFLAGS register mentioned in section \ref{X}. There is a special flag called trap flag. When it is set, cpu will issue an interrupt
after every executed instruction. This could be useful if we wanted to inspect execution instruction by instruction.

\todo{Debugging embedded}

\section{Operating system support}
Operating system is a layer between computer components (cpu, memory, input/output devices, \dots) and software. It is responsible for
handling all those resources so programmers do not have to think about it \cite{modern-os, os-concepts}. Managing resources is not only to make writing programs easier, but to make sure that they are
safe from each other. Modern operating system allows to run multiple programs at once (or at least offer the illusion that it can) and they make sure that
one program cannot overwrite data or otherwise interfere with other programs. Normal programs runs in so called \textit{user space}, which has limited capabilites.
Kernel on the other hand runs in \textit{kernel space}. It has full access to hardware of the computer, can use all instructions, can permit or mask interrupts and so on.

However, if programs were limited to user space all the time they would be very limited.
Sometimes, they need to escape the confiment of the OS, for example to read a file or communicate with other processes.
Operating systems provide an interface through which the user space program
can leverage small part of the kernel - system calls. They offer a way of requiring some service from the OS.
This API is often in form of C and C++ functions~\cite{os-concepts}. A part of these functions is a special instruction, like \texttt{SYSCALL} on x86~\cite{intel-manual}, that
switches the mode to kernel space. The kernel has to check if the call is correct, since it will
be executed in kernel space with full access.

The most prelevant operating systems today are Microsoft Windows, Linux and MacOS.
Linux and MacOS systems are somewhat similar, but Windows is very different.

\subsection{Linux}
Linux offers special system call which is very handy for debugging. It is called \texttt{ptrace} \cite{ptrace} - process
trace. It has following signature: \texttt{ptrace(PTRACE\_COMMAND, pid, ...)}. It takes a \texttt{PTRACE\_COMMAND},
which specifies the behaviour of the function (for example \texttt{PTRACE\_SINGLESTEP} for single step), pid of some
process and some other parameters, depending on the \texttt{PTRACE\_COMMAND} that was chosen.
It allows to observe and control the execution of another process, this process will be the debugee.
In the context of \texttt{ptrace}, we will instead use the word tracee, to be consistent with ptrace documentation.

\mintinline{c}{ptrace} has many commands, here are some of the most important:
\begin{itemize}
    \item \texttt{PTRACE\_PEEKTEXT, PTRACE\_PEEKDATA} - Read tracee's memory,
    \item \texttt{PTRACE\_POKETEXT, PTRACE\_POKEDATA} - Write into tracee's memory,
    \item \texttt{PTRACE\_GETREGS} - Read tracee's register values,
    \item \texttt{PTRACE\_SETREGSET} - Modify tracee's register values,
    \item \texttt{PTRACE\_GETSIGINFO} - Retrieve information about the signal that caused tracee to stop,
    \item \texttt{PTRACE\_CONT} - Restart the stopped tracee process,
    \item \texttt{PTRACE\_SINGLESTEP} - Restart the stopped tracee but stop it after executing one instruction.
\end{itemize}

Linux however needs some way of notifying the debugger that the tracee encountered a breakpoint, or that some other
event requiring debugger attention happened. To this end, \textit{signals} are used.
They are in principle similar to CPU interrupts. They are however on the OS level.
A signal is used in UNIX and Linux systems to notify a process that a particular event has occured \cite{os-concepts}.
Signals can be sent to processes. When such process receives a signal, it stops its execution and starts
the execution of a signal handler. There are various signal types. Most signals can have custom signal handler
defined by the process. If no handler is defined then a default one is provided by the OS.
However handlers for \texttt{SIGKILL} and \texttt{SIGSTOP} cannot be changed \cite{signals}.

Reason for rising a signal can be \todo{Tady toho asi bude vic}
\begin{itemize}
    \item CPU Interrupt (Division by zero, Breakpoint hit),
    \item System call (\textit{kill(pid, signal)}).
\end{itemize}

For example, the signal \texttt{SIGTERM} can be send to a process to ask it nicely to exit.
The process can handle this request, for example to save some state before exiting.
It can also however be completely ignored. For this, a signal \texttt{SIGKILL} can be used,
which cannot be handled, ignored or blocked.



\subsection{Windows}
Windows also has built-in support for debugging at the Win32API layer \cite{windows-msdn-debugging-api, windows-press-debugging-api}.
It builds on \textit{debug events} and \textit{debug functions}. Summary of some of the functions that Win32 API offers which all help with debugging:

\begin{itemize}
    \item \mintinline{c}{DebugActiveProcess} - Attaches the debugger to an active process.
    \item \mintinline{c}{DebugBreakProcess} - Causes a breakpoint exception to occur in the specified process.
                                          This passes control of the process to the debugger if there is one.
    \item \mintinline{c}{WaitForDebugEvent} - Waits for new debug events.
    \item \mintinline{c}{ContinueDebugEvent} - Continue the process execution after processing debug event.
    \item \mintinline{c}{OutputDebugString} - Sends a string to the debugger for display.
    \item \mintinline{c}{ReadProcessMemory} and \mintinline{c}{WriteProcessMemory} - Read and modify process virtual address space.
    \item \mintinline{c}{FlushInstructionCache} - Flushes instruction cache of the process.
\end{itemize}

The general structure of Windows debugger can be seen in figure \ref{fig:win32debugger}.
The debugger waits for debug events via function \mintinline{c}{WaitForDebugEvent}.
This function has a timeout parameter, so the debugger can also do other things while it's waiting.
These events are put in a queue, so the debugger will not miss any.

\begin{figure}
    \centering
    \scalebox{0.8}{
    \begin{tikzpicture}
        \draw (-7,0) -- (-7,-11) (0,0) -- (0,-11) (7,0) -- (7,-11);
        \node at (-7,.3) {Debugee};
        \node at (0,.3) {Win32 API};
        \node at (7,.3) {Debugger};
        \draw[<-] (0,-1) -- node[midway,above] {\mintinline{c}{CreateProcess}} (7,-1);
        \draw[<-] (-7,-2) -- node[midway,above] {Create} (0,-2);
        \draw[->] (0,-3) -- node[midway,above] {\mintinline{c}{CreateProcess} returns} (7,-3);
        \draw[<-] (0,-5) -- node[midway,above] {\mintinline{c}{ContinueDebugProcess}} (7,-5);
        \draw[<-] (0,-6) -- node[midway,above] {\mintinline{c}{WaitForDebugEvent}} (7,-6);
        \draw[dashed,->] (-7,-6.5) -- node[midway,above] {Exception} (0,-6.5);
        \draw[->] (0,-7) -- node[midway,above] {\mintinline{c}{WaitForDebugEvent} returns \texttt{true}} (7,-7);
        \draw[dashed, <-] (-7, -8) -- node[above left] {Debugger actions} (7, -8);
        \draw[<-] (0,-9) -- node[midway,above] {\mintinline{c}{ContinueDebugProcess}} (7,-9);
        \draw[<-] (0,-10) -- node[midway,above] {\mintinline{c}{WaitForDebugEvent}} (7,-10);
        \draw[dashed,->] (-7,-10.5) -- node[midway,above] {Exception} (0,-10.5);
    \end{tikzpicture}
    }
    \caption{A sequence diagram for debugger using Windows api. Inspired by \todo{NI-REV 6. lecture}}
    \label{fig:win32debugger}
\end{figure}

The debug events are thoroughly described in subsection \ref{section:Debug Events}. The main point of interest is the exceptions.
By these, we do not mean the standard C++ exceptions but rather Microsoft \textit{Structured Exception Handling}.

\subsubsection*{Debug Events}\label{section:Debug Events}
Debugging events are various incidents in the debuggee that causes the system to notify the debugger \cite{windows-msdn-debug-events}. These are stored in special \mintinline{c}{DEBUG_EVENT} structure, which is received in \texttt{WaitForDebugEvent} call from debugger. This structure contains various information about the event, the internals can be seen on figure \ref{fig:DebugEvent}. These events include loading and unloading a DLL, creating and exiting a process, sending debug strings via the \mintinline{c}{OutputDebugString} and so on. It also includes exceptions, those are probably the most important for us. 
\begin{figure}
\begin{minted}{c}
typedef struct _DEBUG_EVENT {
  DWORD dwDebugEventCode;
  DWORD dwProcessId;
  DWORD dwThreadId;
  union {
    EXCEPTION_DEBUG_INFO      Exception;
    CREATE_THREAD_DEBUG_INFO  CreateThread;
    CREATE_PROCESS_DEBUG_INFO CreateProcessInfo;
    EXIT_THREAD_DEBUG_INFO    ExitThread;
    EXIT_PROCESS_DEBUG_INFO   ExitProcess;
    LOAD_DLL_DEBUG_INFO       LoadDll;
    UNLOAD_DLL_DEBUG_INFO     UnloadDll;
    OUTPUT_DEBUG_STRING_INFO  DebugString;
    RIP_INFO                  RipInfo;
  } u;
} DEBUG_EVENT, *LPDEBUG_EVENT;
\end{minted}
\caption{Structure which contains info about debug event.}
\label{fig:DebugEvent}
\end{figure}

\subsubsection*{Structured Exception Handling}
This feature is specific to Windows only. For example, if division by zero was performed in a program on Linux,
a signal would be sent to the process. Windows don't have signals, instead, it uses Structured Exception Handling \cite{windows-msdn-seh}. 
From now on, we will be using the abbreviation 'SEH'.
An exception is an event that requires execution of code outside the normal flow of control. There are software exceptions,
like throwing an exception explicitly or by OS, and hardware exceptions, like the division by zero we mentioned.
Instruction with opcode \mintinline{c}{0xCC}, which is used for breakpoints, will also raise an exception. SEH unifies both of these things into one.

When an exception is triggered, control is transferred to the system. It saves the state of the thread and some other information.
This information can be used to continue execution from the point where the exception was thrown when it is resolved. It also
contains information about which type of exception was thrown, if execution can continue after handling the exception, address where the
exception occured and some others\footnote{See MSDN documentation \cite{windows-msdn-seh} for full detailed list}.
The system then searches for an exception handler which will handle the exception. The search is performed in this order:

\begin{enumerate}
    \item If the process is debugged the debugger is notified.
    \item If it is not or the debugger does not handle the exception, the frame-based exception handler is to be found\footnote{The handlers are not very important to us, see MSDN documentation if you're interested \cite{windows-msdn-seh}.}
    \item If no frame-based handler can be found, or no handler handles the exception, but the process is being debugged then the debugger gets notified once again.
    \item The system provides default handling, which is to terminate the program via \mintinline{c}{ExitProcess} most of the time.
\end{enumerate}

Here we see that every exception that occurs in the debuggee causes the debugger to be notified. Breakpoints are also caused by an exception, as was briefly mentioned before. There are two possible notifications to the debugger. The first is known as \textit{first-chance} notification \cite{windows-msdn-dbg-exc-handling}. The debugger can (and should) inspect the information about the exception and see if it was a breakpoint or single-step. These only occurs if the process is debugged (it wouldn't happen otherwise) and the debugger should handle them. If it is something else it can ignore the exceptions. When the program is continued via \mintinline{c}{ContinueDebugEvent}\footnote{This function has a special parameter, which is used to tell that the exception was or was not handled.}, the debugger is notified once again if no appropriate exception handler was found for the exception. This is known as \textit{last-chance} notification because if the debugger does not handle the exception the debuggee will be terminated. It gives the user a chance to debug why is his process terminating.

Here are some exceptions that tie into debugging:
\begin{itemize}
    \item \mintinline{c}{STATUS_BREAKPOINT} - Raised when a hardware-defined breakpoint was encountered. This includes the mentioned \mintinline{c}{INT3} instruction.
    \item \mintinline{c}{STATUS_SINGLE_STEP} - Raised when a single step was completed, ie. when instruction was executed and the trap flag is set.
\end{itemize}

\subsubsection*{Tying it all together}
Now we have all necessary building block to build a simple proof of concept Windows debugger. On figure \ref{fig:windows-debugger-mainloop}, you can see a basic idea of a main loop of the debugger. It waits for debug events and branches depending of the type of event. It needs not only handle exceptions, but other events also. For example if the debugee creates a thread that is something the debugger should be aware of. Modern debuggers trace all threads of the program.

\todo{Pridat dalsi figure kde je jak se hanndlujou tyhle blbosti} However, exceptions are the most interesting for us. There, breakpoint and single step handling should be done. On both of these, the debugger should handle the exception itself, so this is the \textit{first chance} notifications. There is also an \mintinline{DBG_CONTROL_C}, which happens on CTRL + C keyboard press. This should terminate the program. The debugger will pass the first chance and catch the last chance exception, so user has a final chance to look at the program state before it exits.

\begin{figure}
    \begin{minted}{c}
void EnterDebugLoop(const LPDEBUG_EVENT DebugEv)
{
   DWORD dwContinueStatus = DBG_CONTINUE; // exception continuation
   for(;;)
   {
      WaitForDebugEvent(DebugEv, INFINITE);
      switch (DebugEv->dwDebugEventCode)
      {
         case EXCEPTION_DEBUG_EVENT:
            // Handle exception debug events
         // Other debug events
      }
   ContinueDebugEvent(DebugEv->dwProcessId,
                      DebugEv->dwThreadId,
                      dwContinueStatus);
   }
}
\end{minted}
\caption{Windows debugger main loop}
\label{fig:windows-debugger-mainloop}
\end{figure}

\section{Compiler support}
\todo{Moved from introduction here for the time being}
When we talked about evolution of programming from machine code to assembly to higher level languages, we haven't
talked about how they are executed. Machine code can be directly executed by processor, as we said, it is a sequence
of binary. Assembly is text, processors don't understand text. But assembly can be mapped to machine code almost 
1:1\footnote{There are some exceptions, like labels. But translating them is not very difficult.}.

However, high level programming languages do not map 1:1 to assembly. Some are close to it, like C, while others
are miles away, like Haskell. But as was said, processors understand only machine code. To this end, programs that
can translate source code into machine code, were created. They are called compilers and the translation process is
called compiling. For example, for the C language one might use the GCC or Clang compilers.
On figure \ref{fig:compiler-structure} can be seen basic structure of a compiler \cite{dragon-book}. 

\tikzstyle{compilerblock} = [rectangle, draw, minimum width=6cm, minimum height=1cm] 
\tikzstyle{tables} = [rectangle, draw, minimum width=4cm, minimum height=1cm] 
\begin{figure}\label{fig:compiler-structure}
    {\centering
    \begin{tikzpicture}
    \node (lexer)[compilerblock]{Lexical analyzer};
    \node (syntax)[compilerblock,below=of lexer]{Syntactic analyzer};
    \node (semantic)[compilerblock,below=of syntax]{Semantic analyzer};
    \node (imc)[compilerblock,below=of semantic]{Intermediate Code Generator};
    \node (gen)[compilerblock,below=of imc]{Code Generator};
    \node (symbol)[tables, left=of semantic]{Symbol table};
    \draw[->] (lexer) -- node[below] {} (syntax);
    \draw[->] (syntax) -- node[below] {} (semantic);
    \draw[->] (semantic) -- node[below] {} (imc);
    \draw[->] (imc) -- node[below] {} (gen);
    \end{tikzpicture} 
    \par}
    \caption{Simplified structure of a compiler. Some parts were left out, like optimizations.}
    \label{fig:compiler_tikz}
\end{figure}

\subsection{Lexical analyzer}
The lexical analyzer groups separate symbols into groups. For example the code
\begin{minted}{c}
foo = bar(1 + 2);
\end{minted}
might be translated into tokens like this
\begin{lstlisting}[stringstyle=\color{black}]
<id:"foo"> <assignment-operator> <id:"bar"> 
<left-bracket> <int-number:1> <plus-operator> 
<int-number:2> <right-bracket> <semicolon>
\end{lstlisting}
The Syntactic analyzer then works with these tokens.

\subsection{Syntantic and semantic analyzer}
Syntactic analysis accepts tokens and processes them into other intermediate representation. This is
most often an abstract syntax tree (abbr. AST, figure \ref{fig:ast}). It also checks that the source code complies to the grammar of the language.
Semantic analysis then checks that the program is semantically consistent. For example that used variable
has been declared before.

\begin{figure}\label{fig:ast}
    \centering
    \begin{tikzpicture}[,shorten >=1pt,node distance=1.8cm,on grid,initial/.style={}]
    \node (assignment) {$=$};
    \node (foo) [below left =of assignment] {id:foo};
    \node (bar) [below right =of assignment] {call:bar};
    \node (plus) [below right=of bar] {$+$};
    \node (one) [below left =of plus] {$1$};
    \node (two) [below right =of plus] {$2$};
    
    \draw[-, above, scale=0.7] 
    (assignment)   edge node[scale=0.7, left, yshift=0.1cm] {lhs}  (foo)
     (assignment)  edge node[scale=0.7, right, yshift=0.1cm] {rhs}  (bar)
     (bar)         edge node[scale=0.7, right, yshift=0.1cm] {expr} (plus)
     (plus)        edge node[scale=0.7, right, yshift=0.1cm] {rhs}  (two)
     (plus)        edge node[scale=0.7, left, yshift=0.1cm] {lhs}  (one);
    \end{tikzpicture}
    \caption{Simplified example of an abstract syntax tree.}
    \label{fig:astgraph}
\end{figure}
 
\subsection{Intermediate code generation}
This part converts AST into some other representation, most commonly called IR\footnote{IR means intermediate representation.
AST is also intermediate representation, but if we use IR we mean this one.}. IR is closer to machine code, to be easily translated,
but retain some properties that makes it easier to work with it. There are many types of IR. One of the most popular compilers, LLVM, uses
single static assignment (SSA) \cite{llvm}. Example of LLVM IR can be found on figure \ref{fig:llvm-ir-example}. Compilers perform most
optimizations on this intermediate representation. 

\begin{figure}\label{fig:llvm-ir-example}
    \begin{minted}{llvm}
        define dso_local i32 @_Z6squarei(i32 %0) {
          %2 = alloca i32, align 4
          store i32 %0, i32* %2, align 4
          %3 = load i32, i32* %2, align 4
          %4 = load i32, i32* %2, align 4
          %5 = mul nsw i32 %3, %4
          ret i32 %5
        }
    \end{minted}
    \caption{Simplified example of LLVM IR.}
\end{figure}

\subsection{Code generation}
Here, IR is translated directly to the target machine code or possibly assembly. Even though IR can seem very similar to assembly,
there are still some things to take care of. For example SSA IR doesn't have registers, it uses unlimited number of variables.
Other architectures might have some other traits that differ it from the IR and they all have to be accounted for when generating code.

\subsection{Modularity of compilers}
The main advantage of using an IR is that there is a common ground for every language. Imagine we write a compiler for the C language.
We need to write all five parts from figure \ref{fig:compiler-structure}. If we later decided that we also want to create a compiler
for Haskell, we just need to write everything up to the IR translation. Once we can translate Haskell into the IR, we can reuse the
previous part of the compiler to compile to machine code! This also works the other way around. If we compiled IR to the machine code
that works with the x86 architecture, and we want to compile to ARM, we just need to create the code generation part for the ARM architecture,
no need to write whole compiler. Also, most of the optimizations are done on the IR level, this also saves a lot of development time.
The parts of the compiler which are dependent on the source language are called \textbf{frontend} (Syntax, Semantic and IR translation), the parts that are dependent on
the target are called \textbf{backend} (Code generation).

This is widely used in practice. The LLVM \cite{llvm} project is a compiler backend. It uses its own IR (as was mentioned on figure \ref{fig:llvm-ir-example}).
It can compile this IR into many targets, including x86, ARM and Spark \todo{Ocitovat}. The \textit{Clang} project is a compiler frontend for C, C++ and Objective-C languages.
It translates these languages to the LLVM IR. Other frontends for LLVM also include \textit{ghc}, which is a Haskell compiler, or \textit{rustc}, which is a Rust compiler.
With LLVM, creating new programming language comes down to parsing it into an AST and transforming that AST into the LLVM IR.

\subsection{Interpreting programs}
Not all languages are compiled. Imagine a program which can evaluate arithmetic expressions, each phone nowadays has a program like this.
We don't have to stop there. Moving this up a notch, we can create a program that reads source code and executes it.
This is what interpreting means. Dynamically typed languages tend to be interpreted~\cite{python, lua, javascript}\todo{Instead of languages, cite some relevant source}, but it is not a rule~\cite{scala}. Since the interpreter is a program, it is another layer of abstraction. This can make
the resulting languages sometimes more abstract then the compiled ones, sometimes at the cost of performance~\cite{jit}.
Interpreters still \textit{compile} the code into some intermediate representation, but it's not compiled down to machine code, instead that IR is run by a program\footnote{Nowadays, interpreters use JIT compilation, which compiles some of the code some of the time into machine code~\cite{jit}}.

\chapter{The T86 Virtual machine}
At the FIT CTU, in the NI-GEN course students have to write their own compiler.
To ease the code generation part, the T86 architecture was created and a
virtual machine was written as a way to execute programs in that architecture.
The architecture was made as part of a masters thesis \cite{ivo2021tiny}.
This chapter explores the T86 architecture, the virtual machine that was made
for it and explains which parts of the virtual machine will have to be extended
or added to support debugging.
\subsection{T86 ISA}
The name is an abbreviation for Tiny x86, but in some aspects the architecture
is more similar to other architectures like ARM. It is an educational ISA, so
some properties are configurable, like the number of available registers.

The T86 uses a harvard architecture, this means that the data and instructions
are separated from each other. The author doesn't specify the reason for this,
put presumably it to ease the implementation of the virtual machine, since
the memory and instructions can be represented as two separate array-like
members of the virtual machine.

It has the registers we saw on x86\_64, ie. \texttt{PC}, \texttt{SP},
\texttt{BP} and the flags register. The intended roles for these registers are
the same os on x86\_64. It also has other general purpose registers. The number
of these is configurable.

The addressing modes, or what kind of operands instructions can have, include
immediate values, registers and memory accesses. These accesses are not only
immediate or register offsets, but can be combined in various way, like
\texttt{[R0 + R1 * 2]}, or \texttt{[R0 + 10 + R1 * 2]}. The addressing modes
are however not arbitrary, \texttt{[R0 + R1 + R2]} is not correct addressing
mode for T86. For full list refer to \cite{ivo2021tiny}.

The instructions the ISA provides are very similar to what we have already seen
when examining various real world examples. Interesting exceptions to that are
the IO instructions - \texttt{PUTCHAR} and \texttt{GETCHAR}, which allow for
very basic input and output. Also, an \texttt{DBG} and \texttt{BREAK} are
defined. These are used for debugging, but in a very different way that we have
seen in previous sections. We will touch upon them when discussing the virtual
machine implementation, since they are very much tied to it.

\chapter{Implementation}
In this chapter, we describe how we went about the implementation of the
debugger and reason about the design choices we made. Also, we describe which
parts of the virtual machine were modified or added for the benefit of the
debugger or, in general, for the greater good.

\section{T86 ISA extensions}\label{section:parser}
In chapter~\ref{section:t86-vm}, we showed how to build a program for the T86
VM. It was necessary to use the builder the T86 VM provides. We remedied this
with a custom ELF-like format, which we provide a parser for. This format
semantically follows the ELF format and allows the creation of a T86
executable. Contrary to the ELF, it is a text format to allow ease of use. In
chapter \ref{section:t86-vm}, we showed how to build a program for the T86 VM
with the builder interface. To allow the usage of other programming languages,
we have created an ELF-like format for the T86 executables. The format is a
text one, making it easy to use. An example of a program in said format is
shown in figure \ref{fig:t86-program}. It is very similar to the assembly we
have shown in previous sections. Thanks to this, students can implement their
compiler in any programming language they want and emit the T86 program in this
format as a text file. An unfortunate side effect is that we can no longer use
the \texttt{DBG} instruction. However, the debugger we will later present will
be much more powerful than the \texttt{DBG} instruction.

As can be apparent from the example, we also use sections. The \texttt{.text}
section is the only mandatory one. It contains the instructions that will be
executed. Another one is the \texttt{.data} section. Here, either raw numbers
or strings can be written. The contents of this section are then loaded by the
VM and stored into the memory, beginning at memory cell 0 and upwards. There
are also debug sections, which we will present when discussing debugging
information.

\begin{figure}
    \begin{lstlisting}
.data
"Hello, World!\n"

.text
0   MOV [BP - 1], 0
1   JMP 8
2   MOV R0, [BP - 1]
3   MOV R1, [R0]
4   PUTCHAR R1
5   MOV R0, [BP - 1]
6   ADD R0, 1
7   MOV [BP - 1], R0
8   MOV R0, [BP - 1]
9   CMP R0, 13
10  JLE 2
11  HALT
    \end{lstlisting}
    \caption{Example of an T86 program which prints "Hello, World!".}
    \label{fig:t86-program}
\end{figure}
We will also add two new instructions. First is the \texttt{PUTNUM}
instruction, which prints the numerical value in the register and a newline.
This is intended as a very primitive debug instruction and to ease the
automated testing of the compiler. The only other way of output was to print a
char which was represented by the ASCII value. With this instruction, students
can bootstrap the basic implementation of their compiler more easily.

Another one is the \texttt{BKPT} instruction. This instruction is similar to
the \texttt{INT3} instruction from x86-64 or the \texttt{BKPT} instruction from
ARM. It is a software breakpoint. Control will be passed to the connected
debugger when the instruction is hit. This is thoroughly described in the next
section. 

\section{T86 debugging support}
We could bake the debugger into the virtual machine itself, which would likely
be the simplest way to implement it. However, the goal of the debugger is not
only to ease the code generation part but to be a learning point so that
students might grasp how a real debugger works\footnote{The VM followed the
same philosophy.}. Because of this, we aim to simulate the real-world debuggers
as closely as possible. The compilers may also have more targets in the future,
not just the T86 VM. If we made the debugger part of the T86 VM, we could not
use it for a possible new virtual machine. In conclusion, the virtual machine
and the debugger will be two entirely different programs and as such, two
completely different processes.

In the debugger implementation for Linux, the subject of section
\ref{section:linux-dbg}, we described how an operating system's kernel allows
the debugger's implementation via a specific API. There is no operating system
between the virtual machine and the program. Still, we will strive to make the
API similar to the ptrace API. The debugger and the VM will have to communicate
together somehow. For interprocess communication, there are several
possibilities.

Both the VM and the debugger use an abstract class representing an interface
that provides two methods, \texttt{Send} and \texttt{Receive}. The
implementation of this interface then handles the concrete way of
communication. The debugger and VM do not care about it; they merely use these
two methods. There are currently two implementations of this interface. One is
using network communication through sockets. This way, the debugger may attach
to an existing process, even on an entirely different computer. It, however,
has a disadvantage. The messages sent are often short and we need to send a lot
of them. This proved too slow, even with few messages being sent. The second
implementation is via threads. The debugger runs the VM in another thread, and
they communicate via shared queues. This is far faster and allows the debugger
to run the process by himself, making it easier to use and behave like
real-world debuggers.

The format of the communication is a text one, merely because of the ease
of use as opposed to binary format, it is also clearer to see what is
happening. The commands that the virtual machine API offers are
\begin{itemize}
    \item \texttt{PEEKREG x} - Return values of all normal registers.
    \item \texttt{POKEREG x y} - Sets the value in register \texttt{x} to
        $y$.
    \item \texttt{PEEKFLOATREG} - Return values of all float registers.
    \item \texttt{POKEFLOATREG x y} - Sets the value in float register
        \texttt{x} to $y$.
    \item \texttt{PEEKDEBUGREG} - Return value in all debug registers.
    \item \texttt{POKEDEBUGREG x y} - Sets the value in debug register
        \texttt{x} to $y$.
    \item \texttt{PEEKDATA x cnt} - Return value in memory at addresses $x$ to $x + cnt - 1$.
    \item \texttt{POKEDATA x y} - Writes a value $y$ into a memory at
        address \texttt{x}.
    \item \texttt{PEEKTEXT x cnt} - Returns instruction from $x$ to $x + cnt - 1$.
    \item \texttt{POKETEXT x INS} - Rewrite the instruction at address
        \texttt{x} with the newly supplied instruction.
    \item \texttt{CONTINUE} - Continue the execution.
    \item \texttt{TERMINATE} - End the execution.
    \item \texttt{REASON} - Get the reason why the program stopped (breakpoint,
        singlestep, halt).
    \item \texttt{SINGLESTEP} - Does native level single step.
    \item \texttt{TEXTSIZE} - Returns the size of the program.
\end{itemize}
An example of how those commands can be used for communication between the
virtual machine and the debugger is shown in figure \ref{fig:dbg-vm-seq}. The
interface is similar to basic ptrace commands. We separate the memory and
instruction writing because T86 uses Harvard architecture, whereas Linux
doesn't separate text and data address spaces\cite{ptrace}, so the two requests
were equivalent there. The API is made to be simple on purpose. Anything more
complex should be handled in the debugger itself.

The \texttt{Cpu} class is the backbone of the whole virtual machine. It is
responsible for executing the program. The \texttt{Cpu} has some light
debugging capabilities; they were mentioned in section
\ref{section:t86-debug-cap}. Unrolling must be done to display proper values in
registers and memory. This was described in section
\ref{section:superscalar-cpu}. This is already implemented in the VM by its
author, and we can use the same mechanisms for the software breakpoints and
single-stepping.

The \texttt{Cpu} class offers the \texttt{halted} function. This function
returns true if the \texttt{Cpu} executed the \texttt{HALT} instruction, which
marks the end of a program. It also has a \texttt{tick} method, which does one
tick of the CPU. This does not mean that one tick executes one instruction;
there are several pipeline stages that the instruction needs to go through. To
run the VM, one would have to write a loop that checks if the VM has halted. If
it did not, then one tick would be performed. As of now, the Cpu cannot signal
abnormal conditions (excluding halt), like breakpoint hits.

We added another manager-like class called \textit{OS}. This class will take
care of running the program via the CPU class. We also added an
\textit{interrupt} capability to the CPU. For example, when the \texttt{BKPT}
instruction is executed, CPU signals interrupt number $3$. To check if and
which interrupt happened, the CPU now provides a function for it, similar to
the \texttt{halted} function. The OS calls the \texttt{tick} method
periodically, and after every tick, it checks if a halt or interrupt occurred.
If it did, then it passes it to some handler.

When an interrupt that is caused by some debugging features happens (like the
\texttt{BKPT} instruction), the OS calls a method in the \texttt{Debug} class.
This class is also a new addition and is responsible for communication with the
debugger. It uses the text protocol we mentioned previously.

There are several features we added to the CPU to enhance debugging. One of
them is the \texttt{BKPT} instruction, which we have already talked about.
Executing this instruction causes an interrupt \texttt{3} to occur. It is also
possible to set a special flag that causes the CPU to send the interrupt
\texttt{1} after every executed instruction. 

We also added debug registers. They are a special type of registers designed
for triggering breaks on memory access. There are a total of five debug
registers, with the first four containing the memory cell addresses. The fifth
register, called the control register, contains four bits that indicate the
status of each of the first four registers. If a register is active and the
program writes to a memory cell with the same address as is stored in the
register, an interrupt \texttt{2} is generated. Furthermore, the control
register's bits from 8 to 11 reveal which register caused the interrupt. For
instance, if bit 10 is set to 1, the third register is responsible for the
interrupt and the address stored in that register is the one that was written
into.

\begin{figure}
    \centering
    \scalebox{0.8} {
    \begin{tikzpicture}
        \draw (0,0) -- (0,-17.2) (7,0) -- (7,-17.2);
        \node at (7,.3) {Debugger};
        \node at (0,.3) {Virtual machine};
        \draw[<-] (0,-1) -- node[midway,above] {Initializes connection} (7,-1);
        \draw[->] (0,-2) -- node[midway,above] {Accepts connection} (7,-2);
        \draw[<-] (0,-3) -- node[midway,above] {\texttt{PEEKTEXT 5 1}} (7,-3);
        \draw[->] (0,-4) -- node[midway,above] {Ok} (7,-4);
        \draw[<-] (0,-5) -- node[midway,above] {\texttt{POKETEXT 5 BRKPT}} (7,-5);
        \draw[->] (0,-6) -- node[midway,above] {Ok} (7,-6);
        \draw[<-] (0,-7) -- node[midway,above] {\texttt{CONTINUE}} (7,-7);
        \draw[->] (0,-8) -- node[midway,above] {Ok} (7,-8);
        \draw[->] (0,-10) -- node[midway,above] {Program stopped} (7,-10);
        \draw[<-] (0,-11) -- node[midway,above] {\texttt{REASON}} (7,-11);
        \draw[->] (0,-12) -- node[midway,above] {Reason: \texttt{BKPT} instruction} (7,-12);
        \draw[<-] (0,-13) -- node[midway,above] {\texttt{CONTINUE}} (7,-13);
        \draw[->] (0,-15) -- node[midway,above] {Program stopped} (7,-15);
        \draw[<-] (0,-16) -- node[midway,above] {\texttt{REASON}} (7,-16);
        \draw[->] (0,-17) -- node[midway,above] {Reason: \texttt{HALT} instruction} (7,-17);
    \end{tikzpicture}
    }
    \caption{A sequence diagram for the communication between the virtual machine and the debugger.}
    \label{fig:dbg-vm-seq}
\end{figure}

\section{Native Debugger}
The implementation is done in the C++ language. It uses newer standards up to
the C++20 standard. The debugger is implemented as a library. We will call this
the backend of the debugger. A command line interface was also developed,
through which the users might interact with the debugger. This will be called
the frontend of the debugger.

The implemented debugger consists of two main parts. The first one aims to
support native (instruction) level debugging. This part work without
\textbf{any} debugging information whatsoever. The second part focuses on
source-level debugging, described in the section \ref{section:source-debugger}.

The native debugger is split into two additional main layers to make it more
modular. The first layer is called a \texttt{Process}. It is an interface
representing the debuggee process. The implementation of this interface is
responsible for dealing with the concrete architecture, the API of that
architecture, and the communication with the debuggee. One implementation is
provided for the T86 VM. For instance, it has a method called \texttt{ReadText}
and \texttt{WriteText}. The internals of these methods use the
\texttt{PEEKTEXT} and \texttt{POKETEXT} API we described. Outside of this
class, the communication API is never used. If, in the future, another virtual
machine is made, for whichever architecture, it is only needed to implement
this interface. The rest of the debugger can be used as-is.

Another layer is the \texttt{Native} class, which implements the complicated
logic behind a debugger, like setting a breakpoint, handling single-step, and
so forth. It is the primary bread and butter of the native part of the
debugger. Most algorithms are similar to the Linux debugger implementation
presented in section \ref{section:linux-dbg}. For illustration, in figure
\ref{t86dbg:breakpoint} we show a snippet of code used to create a breakpoint.
It first reads the text at the address where we want to set the breakpoint. The
breakpoint opcode then rewrites this text, and the backup of the text is
stored.

When we arrive at the breakpoint and want to continue further, we need to unset
the breakpoint, i.e., replace the breakpoint opcode in the T86 program with the
backup we saved, do a native-level single step, and write the breakpoint back.

Since breakpoints change the underlying code of the debuggee, we need to be
careful when presenting information to the user. If we printed the text that we
get from the debuggee, it might contain the \texttt{BKPT} instructions we set
earlier. We need to mix it with the backup code stored in breakpoints to show
the assembly of the program correctly.

The Native class uses a \texttt{DebugEvent} structure which indicates what
caused the VM to stop. It is implemented as a \texttt{variant} of multiple
structures, for instance, the \texttt{BreakpointHit} or the
\texttt{WatchpointTrigger} structure. It is a variant because the watchpoint
also needs to convey information about an address that caused the break, as do
breakpoints. It could also signal if the break was caused by reading or writing
to the memory cell, although for now, the T86 VM only interrupts on writing.

\begin{figure}
    \begin{minted}{c++}
SoftwareBreakpoint CreateSoftwareBreakpoint(uint64_t address) {
    auto opcode = GetSoftwareBreakpointOpcode();
    // Read the text at the breakpoint address
    auto backup = process->ReadText(address, 1).at(0);
    // Rewrite it with the breakpoint opcode
    std::vector<std::string> data = {std::string(opcode)};
    process->WriteText(address, data);
    // Check that it was truly written
    auto new_opcode = process->ReadText(address, 1).at(0);
    if (new_opcode != opcode) {
        Error(...);
    }
    // Create a breakpoint object which keeps the text backup
    return SoftwareBreakpoint{backup, true};
}
    \end{minted}
    \caption{Debugger code in \texttt{Native} class to enable a breakpoint.}
    \label{t86dbg:breakpoint}
\end{figure}

The native debugger has the following features:
\begin{itemize}
    \item Breakpoints - Can set, unset, enable and disable software breakpoints.
    \item Watchpoints - Can set and unset watchpoints on memory writes.
    \item Single stepping - Can do native level step into, which executes
        current instruction, out, which runs the program until it leaves
        current function and over, which treats function calls as a single
        instruction.
    \item Text manipulation - Can read and write into the debuggee text area,
        effectively allows to overwrite the running code.
    \item Data manipulation - Can read and write into the program memory area.
    \item Register manipulation - Can manipulate with normal, float and debug registers.
\end{itemize}

\section{Source debugger}\label{section:source-debugger}
With the solid foundation represented by the native part of the debugger, we
can extend it by providing some form of source-level debugging. For this part,
we need to remember that the debugger will only be used by students. As such,
we ought to have gentler debugging information than DWARF, but we certainly can
take inspiration from it.

As we previously mentioned, the executable with T86 code is separated into
sections. The \texttt{.text} and \texttt{.data} sections are for the VM. We
will introduce new sections where debugging information will be stored. All
those sections will have \verb|.debug_| prefix. The simplest new section is
\verb|.debug_source|, which should contain the original source code which was
compiled into this executable. This later allows us, with the combination of
other information, to display the source code lines.

The main philosophy of the source-level debugger is to allow an arbitrary
amount of debug information. For instance, the user can generate information
about one function only, and for that function, source debugging capabilities
will work, but not for any other. This means that users can generate debugging
information incrementally.

The debugger is, of course, not only limited to TinyC language. Any imperative
language that can be encoded with the following debugging information is
suitable to be debugged at the source level. We show an example of this in a
provided test case for the debugger. It can be found as an attachment to the
thesis on path \todo{doplnit cestu}.

\subsection{Line information}
Line information is one of the simplest to generate. Since it will probably be
the most generated information by the students, we choose the following simple
format for it: \texttt{<line>:<address>}. The section \verb|.debug_line| is
made up of these lines. The format is self-explanatory. It says that line
\texttt{x} corresponds to address \texttt{y}. This is very easy to generate.

With this information, we are able to do source-level breakpoints. If the
source code is also provided, we can show the user on which line is the
debugged program currently paused. It is not necessary to specify every line in
the program. The debugger will refuse to put a source-level breakpoint on some
line if it does not have the necessary information.

\subsection{Debugging information format}
In the line information, we provided a straightforward format. However, we will
need a more sophisticated structure to describe some advanced constructs of the
source code. Here we will draw inspiration from the Dwarf Debugging Information
(DIE). Take a look at figure \ref{fig:t86dbg-die}, which shows an example of
such debugging information. It has a tree-like structure which, in some ways,
mimics the original program. The nodes of this tree are also called debugging
information entries (DIEs). Those entries can have other entries as their
children, and each entry has a tag that is part of its name (for example, the
\verb|compilation_unit| tag). They can also have attributes that describe their
properties. As can be seen, this is very similar to the DWARF debugging
information format. Unlike DWARF, it will be a text format. This allows us to
generate the format easily and to spot mistakes quickly. We don't want the
students to debug their generated debugging information.

For instance, the tag \verb|DIE_function| represents a function. As attributes,
it has a name, beginning address, and end address. With this additional
information, we can set a breakpoint on a function name. We can also display in
which function we are located when a break happens.

It also has one direct child, a \verb|DIE_scope|. The scope entry is mainly
used for keeping track of which variables are currently active because the T86
(or any other assembly language in general) has no notion of scopes. In the
scope entry in the example, only one variable called \texttt{d} exists. Thanks
to those entries, we can list currently active variables. We, however, often
need to examine the value of a variable. To achieve this, information about the
location and the variable type is needed.

\begin{figure}
    \begin{lstlisting}
DIE_compilation_unit: {
DIE_function: {
    ATTR_name: main,
    ATTR_begin_addr: 0,
    ATTR_end_addr: 10,
    DIE_scope: {
        ATTR_begin_addr: 0
        ATTR_begin_addr: 10
        DIE_variable: {
            ATTR_name: d,
        },
    }
}
}
    \end{lstlisting}
    \caption{Debugging function information for the T86 debugger.}
    \label{fig:t86dbg-die}
\end{figure}

The type information is encoded as a standalone DIE. Currently, three type
entries are present, one for primitive types (int, double, or char), one for
pointers, and one for structured types (\texttt{struct} or \texttt{class} in
C++). Other types can be easily added in the future. The types are saved as
separate entries, and as such, we need some way to link them together with the
variables. We will use the \verb|ATTR_id| attribute to achieve this. This
attribute should be unique for every entry. This role is similar to the
\texttt{id} attribute of HTML elements~\cite{html4}. The variables themselves
have the \verb|ATTR_type| attribute, which will have an id of the type as its
value. An example of a pointer type that points to an int type is in figure
\ref{fig:t86dbg-types}. If we had a variable that is a pointer to int, it would
need to have the \verb|ATTR_type: 1| attribute because the id of a pointer type
to integer is one.

The primitive types need to have their size. For T86, this is the number of
memory cells it occupies, which will almost always be one since one memory cell
is 64 bits. It also has a name for its primitive type. Currently, three
are supported:
\begin{itemize}
    \item \texttt{int} - A signed integer.
    \item \texttt{float} - A floating point number.
    \item \texttt{char} - A number representing an ASCII character.
\end{itemize}

Structured types are more complicated. They need to have a list of members
which are stored in the structure. For each member, an offset from the
beginning of the structure must also be specified. It also must provide a size
because the compiler might align it, and it may be larger than the sum of the
size of its members.

\begin{figure}
    \begin{lstlisting}
DIE_primitive_type: {
    ATTR_name: int,
    ATTR_id: 0,
    ATTR_size: 1,
},
DIE_pointer_type: {
    ATTR_type: 1,
    ATTR_id: 1,
    ATTR_size: 1
},
    \end{lstlisting}
    \caption{Debugging type information for the T86 debugger.}
    \label{fig:t86dbg-types}
\end{figure}

With this information, we can show the type of the variable. Nevertheless, the
most valuable thing is its value. Variables are either stored in memory,
registered, or optimized out completely. We will follow DWARF's footsteps and
provide a language for a virtual machine that one can interpret to gain
information about the variable location.

Several examples can be seen in figure \ref{fig:t86dbg-vm}. Same as DWARF, it
is a stack-based machine. After computation is done, the top value on the stack
is considered to be the location of the variable. The instruction \texttt{PUSH}
pushes either offset or register onto the stack. In the first example, the
final location is the register \texttt{R0}. In the second example, the location
is a memory offset equal to the value in register \texttt{BP} minus two. The
third example has the same meaning, but a shorthand was provided since the
variables will most often be stored at some offset from the register
\texttt{BP}. There is also a dereference instruction, which can be used to
inspect to which location a pointer points.

\begin{figure}
    \begin{lstlisting}
- [PUSH R0]
- [PUSH BP; PUSH -2; ADD]
- [BASE_REG_OFFSET -2]
    \end{lstlisting}
    \caption{Debugging type information for the T86 debugger.}
    \label{fig:t86dbg-vm}
\end{figure}

This information is stored in a variable attribute called
\texttt{ATTR\_location}. If all this information is provided, we know where the
variable is stored and may look up its value. Together with the type
information, we might also properly interpret the value and report it to the
user.

\subsection{Expressions}
We could make a very straightforward implementation of getting variable value
by its name. It is only a matter of finding the variable DIE with the correct
name and interpreting its location and type. However, we often need to inspect
some more complicated expressions. For example, we may want to display some
struct member or a value at which pointer points.

The debugger has a built-in interpreter for such expressions. It builds an AST
from the expression and interprets it using
AST-walk~\cite{crafting-interpreters}. It leverages the \texttt{Native} class
to fetch variable values. It supports almost all C operators, excluding the
assignment operator. An example of such an expression is \texttt{foo[2]->bar + 3}.

\subsection{Frontend}
Finally, we provide a command line interface that leverages the debugger
library to make the debugger usable. It provides many commands, and its manual
can be found as an appendix to this thesis\todo{DODAT}.

The main priority of the CLI is to make the debugger easy to use. It consists
of several commands, one of them is \texttt{breakpoint set 5}, which will set a
source-level breakpoint on the fifth line of the program. It is, however, not
necessary to write the whole command. Any prefix will do, like \texttt{b s 5}.
The CLI leverages the \textit{linenoise}~\cite{linenoise} library to make the
REPL satisfying to use.

The CLI also displays various information on program stop, like why the program
stopped, on which address or line, and prints the surrounding lines of assembly
or source. The CLI can also list breakpoints and display their locations in the
disassembly or the source code. Figure \ref{fig:cli-hit} provides an example of
a breakpoint hit report. The debugger had all debugging information available
here. It can show the line in the source code where the breakpoint happened,
name the offending function, and variables in scope.

\begin{figure}
    \begin{lstlisting}
Process stopped, reason: Software breakpoint hit at line 11
function main at 7-18; active variables: a, b
      9:    int b = 6;
     10:    swap(&a, &b);
@->  11:    print(a);
     12:    print(b);
     13:}
    \end{lstlisting}
    \caption{Example of the CLI reporting a breakpoint hit.}
    \label{fig:cli-hit}
\end{figure}

\chapter{Evaluation}
In this chapter, we evaluate the debugger, comparing its usage to the gdb and
lldb debugger. We will also measure the impact of the debugging onto the
performace.

\section{The development process}
The development of the thesis was done in a github repository. The power of
github was leveraged not only for keeping a history, but also for writing
issues, planning the development, or making sure that the repository is in
consistent and working state via github actions, which runs tests before pull
request is accepted. This repository can be used going forward.

The project itself contains many tests. The code is first tested via many unit
tests. The \texttt{GoogleTest} framework was used. The unit tests cover almost
all parts of the code.

\section{Usage}
There was an attepmt to follow the interface of GDB closely, so that the users
are somewhat familiar with the debugger before even running it. However, we had
to diverge on some features. For example, the GDB uses \texttt{stepi} for
single stepping. This has a common prefix with \texttt{step}. We however expect
our users to use the assembly level debugging more often than source level, so
we choose the \texttt{istep} command. The two commands have no shared prefix
and so typing \texttt{is} is enough.

Also, the program is run via \texttt{run}, this starts the VM but the program
is paused. In GDB, one has to use the \texttt{start} command, \texttt{run} runs
the program without stopping at the beginning. When we did a brief user
testing, it was confusing for the user. However, renaming the command
\texttt{run} to \texttt{start} would cause it to have the same prefix as
\texttt{step}. Considering this, we've made the decision to leave the command
as \texttt{run}.

There are other minor things, most of them come from the fact that we give more
priority to assembly level debugging, whereas GDB focus more on the source
level.

\chapter{Conclusion}
\section{Summary}
We explored debugging capabilities of modern CPU architectures. We also
described how debugging is supported at various layers, from operating systems
to compilers. Then, we discuss the T86 architecture and its debugging support.
We remedy this by adding a debugging API inspired by modern architectures. We
also created a native and source-level debugger, which is production ready and
already used in the NI-GEN course at FIT CTU. This debugger is extensible
enough that if a new architecture, virtual machines, or source language comes
into play, it should be fairly easy to add debugging support.

The debugger encourages students in the NI-GEN course to investigate how the
debugger works, the connection between the generated machine code and the
source code, and to emit information about those connections so that the
debugger can work at the source level. It can also make their lives easier
since the debugger works at the native level without additional work, allowing
them to debug the code their compiler generated.

The tool is, along with the enhanced T86 virtual machine, openly available at \\  
\verb|https://github.com/Gregofi/t86-wdebug|.

\section{Future Work}
There are many possible improvements to be made. We have created a new
executable text format for T86 ISA and, consequently, a parser for that format.
Students, however, have to generate this text, which includes not only the
instructions themselves but also the debugging information. A builder interface
could go a long way, especially for the most commonly used languages in the
NI-GEN course.

The native part of the debugger is fairly complete. It can always be extended
for new architectures and virtual machines should they emerge. The expression
command, which evaluates a TinyC expression, cannot handle function calls.
There is also no way to format the output (for instance, to print an integer as
a hexadecimal number). The expression interpreter can always be extended to
handle other languages.

New types can always be added to the debugging information. For example, we are
missing enums, qualifiers like \texttt{const} and \texttt{mutable}, or more
advanced types that are in the C++ standard library. Those types are not in the
TinyC language that is being taught in the NI-GEN course, but the debugger is
not strictly tied to the language. Some parts of the program could be
optimized, like the step out which can prove slow, as demonstrated in
section~\ref{section:benchmark}.

The debugger currently does not handle frame information in any way. This could
be a helpful addition so that it displays call frames and allows one to step
out of them. This would require a new section, which would have to contain
enough information to simulate an unwind. The exact mechanism was described in
section~\ref{section:call-frames}.

The T86 VM generates statistics about the program execution. It records things
like pipeline stalls. Since the debugger has the capability to connect the
source to the assembly, it could display the code hot spots in the source code
directly.

Last but not least, a graphical user interface could be created for the
debugger. Currently, it only offers a command line interface. A graphical
interface could be more pleasant to work in. Alternatively, it could be hooked
to an existing editor, like the Visual Studio Code, which allows the usage of
other debuggers like LLDB or GDB.


\bibliographystyle{iso690.bst}
\bibliography{ref}

\appendix

% \printglossaries

\chapter{Contents of CD}\label{app:CDcontent}

Visualise the contents of enclosed media. Use of \verb|dirtree| is recommended. Note that directories src and text with appropriate contents are mandatory.

\begin{figure}
	\dirtree{%
		.1 readme.txt\DTcomment{the file with CD contents description}.
		.1 data\DTcomment{the data files directory}.
		.2 graphs\DTcomment{the directory of graphs of experiments}.
		.3 *.eps\DTcomment{the B/W graphs}.
		.3 *.png\DTcomment{the color graphs}.
		.3 *.dat\DTcomment{the graphs data files}.
		.1 exe\DTcomment{the directory with executable WBDCM program}.
		.2 wbdcm\DTcomment{the WBDCM program executable (UNIX)}.
		.2 wbdcm.exe\DTcomment{the WBDCM program executable (Windows)}.
		.1 src\DTcomment{the directory of source codes}.
		.2 wbdcm\DTcomment{the directory of WBDCM program}.
		.3 Makefile\DTcomment{the makefile of WBDCM program (UNIX)}.
		.2 thesis\DTcomment{the directory of \LaTeX{} source codes of the thesis}.
		.3 figures\DTcomment{the thesis figures directory}.
		.3 *.tex\DTcomment{the \LaTeX{} source code files of the thesis}.
		.1 text\DTcomment{the thesis text directory}.
		.2 thesis.pdf\DTcomment{the Diploma thesis in PDF format}.
		.2 thesis.ps\DTcomment{the Diploma thesis in PS format}.
	}
\end{figure}


\end{document}
