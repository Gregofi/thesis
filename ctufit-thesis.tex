%% This is the ctufit-thesis example file. It is used to produce theses
%% for submission to Czech Technical University, Faculty of Information Technology.
%%
%% Get the newest version from
%% https://gitlab.fit.cvut.cz/theses-templates/FITthesis-LaTeX
%%
%%
%% Copyright 2021, Eliska Sestakova and Ondrej Guth
%%
%% This work may be distributed and/or modified under the
%% conditions of the LaTeX Project Public Licenese, either version 1.3
%% of this license or (at your option) any later version.
%% The latest version of this license is in
%%  https://www.latex-project.org/lppl.txt
%% and version 1.3 or later is part of all distributions of LaTeX
%% version 2005/12/01 or later.
%%
%% This work has the LPPL maintenance status `maintained'.
%%
%% The current maintainer of this work is Ondrej Guth.
%% Contact ondrej.guth@fit.cvut.cz for bug reports.
%% Alternatively, submit bug reports into the tracker at
%% https://gitlab.fit.cvut.cz/theses-templates/FITthesis-LaTeX/issues
%%
%%

%%%%%%%%%%%%%%%%%%%%%%%%%%%%%%%%%%%%%%%%%
% CLASS OPTIONS
% language: czech/english/slovak
% thesis type: bachelor/master/dissertation
% colour: bw for black&white OR no option for default colour scheme
%%%%%%%%%%%%%%%%%%%%%%%%%%%%%%%%%%%%%%%%%
\documentclass[english,master,unicode,bw]{ctufit-thesis}

%%%%%%%%%%%%%%%%%%%%%%%%%%%%%%%%%%
% FILL IN THIS INFORMATION
%%%%%%%%%%%%%%%%%%%%%%%%%%%%%%%%%%
\ctufittitle{Tiny86 Debugger}
\ctufitauthorfull{Bc. Filip Gregor}
\ctufitauthorsurnames{Gregor} % replace with your surname(s) / family name(s)
\ctufitauthorgivennames{Filip} % replace with your first name(s) / given name(s)
\ctufitsupervisor{Ing. Petr Máj} % replace with name of your supervisor/advisor (include academic degrees)
\ctufitdepartment{Department of theoretical computer science} % replace with the department of your defence
\ctufityear{2023} % replace with the year of your defence
\ctufitdeclarationplace{Prague} % replace with the place where you sign the declaration
\ctufitdeclarationdate{\today} % replace with the date of signature of the declaration
\ctufitabstractCZE{
Programátoři často potřebují kontrolovat stav svých programů za běhu. Právě pro
tento účel byl vytvořen speciální nástroj zvaný debugger. Přestože je tento
nástroj velmi rozšířen, málokdo ví, jak přesně funguje. Částečně je to proto,
že musí být podporován na více úrovních, jako je procesor, operační systém a
překladač. Většina kurzů, které o nich vyučují, se debugováním nezabývá.

Tato práce zkoumá, jakou podporu musí poskytovat procesor a operační systém,
aby bylo možné provádět debugování na nativní úrovni. Implementace malého
debuggeru je demonstrována na architektuře x86-64 a operačním systému Linux.
Pozornost je poté přesunuta na podporu překladače pro debugování na úrovni
zdrojového kódu.

Na základě těchto poznatků je v práci představen debugger pro architekturu T86
a jazyk TinyC, které jsou využívány v kurzu překladačů NI-GEN na FIT ČVUT.
Tento debugger je plně funkční a usnadňuje studentům práci s architekturou T86.
Kromě toho práce představuje návrh a implementaci nového formátu debugovacích
informací, který zachovává zajímavé koncepty z reálných debuggerů a zároveň je
jeho použití extrémně jednoduché na strojové i lidské úrovni, což je ideální
pro jeho zamýšlené použití ve výuce.
}

\ctufitabstractENG{
Programmers often need to inspect the state of their programs at runtime. A
special tool called a debugger exists precisely for this purpose. Despite its
widespread use, very few know how exactly this tool works. This is partly
because it must be supported at multiple layers, like the CPU, the operating
system, and the compiler. Most of the courses that teach about these do not
delve into debugging.

This thesis explores what support must be provided by the CPU and operating
system to enable native-level debugging. A small debugger implementation is
demonstrated on the x86-64 architecture and the Linux operating system. Focus
is then shifted onto compiler support for source-level debugging.

Using this knowledge, the thesis presents a debugger for the T86 architecture
and the TinyC language, both of which are used by the NI-GEN compilers course
at FIT, CTU. This debugger is fully functional, making it easier for students
to work with the T86. Additionally, the thesis presents a design and
implementation of a novel format of debugging information that keeps the
interesting concepts from real-world debuggers while being extremely simple to
use on both machine and human level, which is ideally suited for its intended
classroom use.
}

\ctufitkeywordsCZE{Debugování, Debugger, Překladač, Implementace debuggeru,
LLVM, Linux, Windows, Tiny x86, Podpora pro debugování, Chyby v programech}
\ctufitkeywordsENG{Debugging, Debugger, Debug, Debugger implementation, Compiler, LLVM,
Linux, Windows, Tiny x86, Debugging support, Errors in programs}
%%%%%%%%%%%%%%%%%%%%%%%%%%%%%%%%%%
% END FILL IN
%%%%%%%%%%%%%%%%%%%%%%%%%%%%%%%%%%

%%%%%%%%%%%%%%%%%%%%%%%%%%%%%%%%%%
% CUSTOMIZATION of this template
% Skip this part or alter it if you know what you are doing.
%%%%%%%%%%%%%%%%%%%%%%%%%%%%%%%%%%

\RequirePackage{iftex}[2020/03/06]
\iftutex % XeLaTeX and LuaLaTeX
    \RequirePackage{ellipsis}[2020/05/22] %ellipsis workaround for XeLaTeX
\else
    \RequirePackage[utf8]{inputenc}[2018/08/11] %this file encoding
    \RequirePackage{lmodern}[2009/10/30] % vector flavor of Computer Modern font
\fi

% hyperlinks
\RequirePackage[pdfpagelayout=TwoPageRight,colorlinks=false,allcolors=decoration,pdfborder={0 0 0.1}]{hyperref}[2020-05-15]

% uncomment the following to hide all hyperlinks
% \RequirePackage[pdfpagelayout=TwoPageRight,hidelinks]{hyperref}[2020-05-15]

\RequirePackage{pdfpages}[2020/01/28]

\setcounter{secnumdepth}{4} % numbering sections; 4: subsubsection



%%%%%%%%%%%%%%%%%%%%%%%%%%%%%%%%%%
% CUSTOMIZATION of this template END
%%%%%%%%%%%%%%%%%%%%%%%%%%%%%%%%%%


%%%%%%%%%%%%%%%%%%%%%%
% DEMO CONTENTS SETTINGS
% You may choose to modify this part.
%%%%%%%%%%%%%%%%%%%%%%
\usepackage{dirtree}
\usepackage{lipsum,tikz}
\usepackage{csquotes}
\usepackage[style=iso-numeric]{biblatex}
\addbibresource{text/bib-database.bib}
\usepackage{listings} % typesetting of sources
\usepackage[cache=false, outputdir=out]{minted} % typesetting of sources
\usepackage{todonotes}
\usepackage[fontsize=11pt]{fontsize}
% \usepackage{lmodern}
\usemintedstyle{borland}
% Tikz thingies
\usepackage{multirow}
\usepackage{caption}
\usepackage{subcaption}
\usepackage{array}
% \usepackage{showframe}

\captionsetup[table]{skip=10pt}

\lstdefinestyle{mystyle}{
    commentstyle=\color{codegreen},
    keywordstyle=\color{magenta},
    numberstyle=\tiny\color{codegray},
    stringstyle=\color{codepurple},
    basicstyle=\ttfamily\footnotesize,
    breakatwhitespace=false,
    breaklines=true,
    captionpos=b,
    keepspaces=true,
    showspaces=false,
    showstringspaces=false,
    showtabs=false,
    tabsize=4
}
\lstset{style=mystyle}

\usetikzlibrary{calc,positioning,arrows,shapes.geometric}
\tikzstyle{process} = [rectangle, 
minimum width=3cm, 
minimum height=1cm, 
text centered, 
text width=3cm, 
draw=black, 
]

\usetikzlibrary{arrows.meta}
\tikzset{>={Latex[width=3mm,length=3mm]}}

\begin{document} 
\frontmatter\frontmatterinit % do not remove these two commands

\includepdf[pages={1-}]{assignment-include.pdf} % replace that file with your thesis assignment provided by study office

\thispagestyle{empty}\cleardoublepage\maketitle % do not remove these three commands

\imprintpage % do not remove this command

\tableofcontents % do not remove this command
%%%%%%%%%%%%%%%%%%%%%%
% list of other contents: figures, tables, code listings, algorithms, etc.
% add/remove commands accordingly
%%%%%%%%%%%%%%%%%%%%%%
\listoffigures % list of figures
\begingroup
\let\clearpage\relax
\listoftables % list of tables
% \lstlistoflistings % list of source code listings generated by the listings package
% \listoflistings % list of source code listings generated by the minted package
\endgroup
%%%%%%%%%%%%%%%%%%%%%%
% list of other contents END
%%%%%%%%%%%%%%%%%%%%%%

%%%%%%%%%%%%%%%%%%%
% ACKNOWLEDGMENT
% FILL IN / MODIFY
% This is a place to thank people for helping you. It is common to thank your supervisor.
%%%%%%%%%%%%%%%%%%%
\begin{acknowledgmentpage}
    First and foremost, thanks to my parents, who have always supported me and
    enabled me to pursue my dreams, whatever they may be.

    I would also like to thank my supervisor, who spent an unbelievable amount
    of time and patience to break through my stubbornness and make this thesis
    what it is.

    Last but not least, thanks to my girlfriend, Jana. Without her support,
    this work would never have come to be.
\end{acknowledgmentpage} 
%%%%%%%%%%%%%%%%%%%
% ACKNOWLEDGMENT END
%%%%%%%%%%%%%%%%%%%


%%%%%%%%%%%%%%%%%%%
% DECLARATION
% FILL IN / MODIFY
%%%%%%%%%%%%%%%%%%%
% INSTRUCTIONS
% ENG: choose one of approved texts of the declaration. DO NOT CREATE YOUR OWN. Find the approved texts at https://courses.fit.cvut.cz/SFE/download/index.html#_documents (document Declaration for FT in English)
% CZE/SLO: Vyberte jedno z fakultou schvalenych prohlaseni. NEVKLADEJTE VLASTNI TEXT. Schvalena prohlaseni najdete zde: https://courses.fit.cvut.cz/SZZ/dokumenty/index.html#_dokumenty (prohlášení do ZP)
\begin{declarationpage}
I hereby declare that the presented thesis is my own work and that I have cited all
sources of information in accordance with the Guideline for adhering to ethical
principles when elaborating an academic final thesis.

I acknowledge that my thesis is subject to the rights and obligations stipulated by the
Act No. 121/2000 Coll., the Copyright Act, as amended. In accordance with Article 46(6)
of the Act, I hereby grant a nonexclusive authorization (license) to utilize this thesis,
including any and all computer programs incorporated therein or attached thereto and
all corresponding documentation (hereinafter collectively referred to as the “Work”), to
any and all persons that wish to utilize the Work. Such persons are entitled to use the
Work in any way (including for-profit purposes) that does not detract from its value.
This authorization is not limited in terms of time, location and quantity. However, all
persons that makes use of the above license shall be obliged to grant a license at least
in the same scope as defined above with respect to each and every work that is created
(wholly or in part) based on the Work, by modifying the Work, by combining the Work
with another work, by including the Work in a collection of works or by adapting the
Work (including translation), and at the same time make available the source code of
such work at least in a way and scope that are comparable to the way and scope in
which the source code of the Work is made available.
\end{declarationpage}
%%%%%%%%%%%%%%%%%%%
% DECLARATION END
%%%%%%%%%%%%%%%%%%%

\printabstractpage % do not remove this command


\mainmatter\mainmatterinit % do not remove these two commands

%%%%%%%%%%%%%%%%%%%
% THE THESIS
% MODIFY ANYTHING BELOW THIS LINE
%%%%%%%%%%%%%%%%%%%

\chapter{Introduction}
At the core of every computer program lies the Central Processing Unit (CPU),
which is responsible for executing programs. The CPU excels at performing very
primitive operations very fast. These operations are called instructions, they
can perform simple arithmetics, move values from and to memory, and change the
control flow of the program. The instructions are encoded as a sequences of
binary numbers which are easy for the CPU to understand, but for humans they
are rather incomprehensible.

To help programmers better understand written programs, a text mapping was
created, called \textit{assembly language}. Each instruction is now assigned an
text representation, as are the operands of the instruction. The control flow
instructions now doesn't have to jump to address offset, but can instead use
labels. Example of a simple program in both an assembly language and a machine
code can be seen in figure~\ref{fig:simple-assembly}. If a programmer knows the
instruction set architecture~\cite{isa} of the processor he can easily
recognize the instructions the program is made of.

\begin{figure}
    \begin{minipage}{.45\textwidth}
    \begin{lstlisting}
positive:
        push    rbp
        mov     rbp, rsp
        mov     -4[rbp], edi
        cmp     -4[rbp], 0
        jle     neg
        mov     eax, 1
        jmp     pos
neg:
        mov     eax, 0
pos:
        pop     rbp
        ret
    \end{lstlisting}
    \end{minipage}
    \hfill\vline\hfill
    \begin{minipage}{.45\textwidth}
    \begin{lstlisting}
   01010101 01001000
   10001001 11100101
   10001001 01111101
   11111100 10000011
   01111101 11111100
   00000000 01111110
   00000111 10111000
   00000001 00000000
   00000000 00000000
   11101011 00000101
   10111000 00000000
   00000000 00000000
   00000000 01011101
   11000011
    \end{lstlisting}
    \end{minipage}
    \caption{An example of a program that checks if a number is positive, shown
    in both an assembly language and a machine code.}
    \label{fig:simple-assembly}
\end{figure}

But, as the computers became increasingly more powerful, so did the programs
became bigger and more complex. When programming in the assembly language,
the programmer must have an extensive knowledge of the processor's internal
workings.

To spare the programmers from this, high level programming languages were
created. These are designed to abstract from the specific machine the program
will run on, allowing programmers to focus more on their task. As shown in
figure~\ref{fig:c-positive}, even a simple program written in the C programming
language, one of the oldest programming language around, provides a clear
understanding of the functionality. In contrast, examining the equivalent
program in assembly language, as seen in figure~\ref{fig:simple-assembly},
requires knowledge of the specific architecture of the machine. High level
languages also introduces control flow statements, which makes the code easier
to follow than when jumps are used~\cite{gotobad}.

\begin{figure}
\begin{minted}[
]{c}
bool positive(int n) {
    if (n > 0) {
        return true;
    } else {
        return false;
    }
}
\end{minted}
\caption{A function written in the C programming language that check if a
    number is positive.}
\label{fig:c-positive}
\end{figure}

However, we previously said that processors can only understand machine code
and high level languages are far from it. Therefore, a translation of the
high-level language into machine code is necessary, this is a task for a
\textit{compiler}. A compiler is a program that reads some source code of
high-level language and produces machine code. Compilers are a very complicated
piece of software, and we will talk about them in detail in chapter
\ref{section:source-level-debugging}. For now, it is important to uderstand
that the computer cannot run the source code of a high-level language directly,
and that it is translated into machine code. Additionaly, compilers often take
advantage of special features of the architecture to make the programs
faster~\cite{dragon-book}.

\begin{figure}[H]
    \begin{minted}[
            linenos,
]{c}
int binary_search(int* arr, int len, int n) {
    int lo = 0;
    int hi = len;
    while (lo < hi) {
        int i = (lo + hi)/2;
        if (arr[i] < n) {
            lo = i;
        } else if (arr[i] > n) {
            hi = i;
        } else {
            return 1;
        }
    }
    return 0;
}
    \end{minted}
    \caption{Implementation of the binary search algorithm written in the C
    programming language.}
    \label{fig:binary-search}
\end{figure}

In the figure \ref{fig:binary-search} we present a more complicated example of
a program written in high level programming language. This is an implementation
of the binary search algorithm. As an input, it receives sorted sequence of
numbers and some number $n$ and checks if that number $n$ is in the sequence.
This algorithm is widely used when searching in sorted sequence because of its
$\mathcal{O}(\text{log}_2(n))$ complexity~\cite{pruvodce}.

\section{Debugging}
Programs are mostly written by humans, who tend to make
mistakes~\cite{human-error}. We are no exception, as we have also made a
mistake in the binary search program. Let us try to run the program with a
\texttt{[1,2,3]} sequence and search for the number $4$. This number is not in
the sequence, so the expected output would be $0$. Instead, if we ran the
program it would run forever, because of a mistake we made in the source code.
Such mistakes are called a \textit{bug}\footnote{The term \textit{bug}
actually comes from a real bug that got stuck in relays back when computers
were made from relays. They literally had to debug the machine by taking the
bug out.}. The process of finding these mistakes and correcting them is called
\textit{debugging}~\cite{art-of-testing}.

There are several ways approaches to debugging. We could try to look at the
source code and find the mistake this way. Here we could assume that the
condition \texttt{lo < hi} never comes to be since it is the most obvious place
where we could get stuck forever. Now, it would be helpful if we could see the
states of \texttt{lo} and \texttt{hi} in each iteration of the cycle. We could
resort to print statements, but that is not very flexible. If we changed our
minds and wanted to also see the value of variable \texttt{i}, we would have to
recompile the program and rerun it. The output can also quickly get
overwhelming, especially in an infinite loop. However, all is not lost, as a
special program was created to inspect running programs, called a
\textit{debugger}.

Debugger is able to inspect the state of another program, like the values of
its variables. It is also able to control the flow of the program. They allow
\textit{breakpoints} to be set at each line of the source
code\footnote{Advanced debuggers allow breakpoints to be set inside
expressions. This is especially important for functional languages, as their
functions often consist of one big expression.}. When the program is about to
execute the line of code with the breakpoint, the control is passed back to the
debugger and the user can inspect the state of the program at that line. There
are also conditional breakpoints, which only trigger when some condition holds.
Example of such condition can be that the breakpoint gets activated only when
\texttt{i == 3}.

Finally, debuggers also allow \textit{stepping}. This also modifies the control
flow of the program.
\begin{itemize}
    \item \textit{step in} - Executes current statement and stops on the next
        one. If the current statement is a function call then it will be
        executed and the program will be paused on the first statement in that
        function.
    \item \textit{step over} - Same as a step in, but if the current statement
        is a function call then the program will be paused on the next
        statement after the call.
    \item \textit{step out} -  Executes as much as needed to return from the
        current function. Stops on the next statement that should be executed
        after the function returns.
\end{itemize}

Now, back to our program. Let us put a breakpoint on line $8$ after $i$ is set.
We will monitor how the \texttt{lo} and \texttt{hi} change. If the program is
run with the debugger attached, an output similar to what is displayed in
figure~\ref{fig:lldb-debug1} will be seen. Here, it is possible to see the line
on which the execution was stopped. It is also possible to print the state of
variables. In each loop, we could print the value of a variable and then
continue execution until another breakpoint is hit. There is only one set, so
the execution will again be stopped on line $8$. The value of \texttt{hi} will
not change, which is expected. Value of \texttt{lo} will gain following values:
$0, 1, 2, 2, 2, \dots$ The value apparently gets stuck at $2$ and never change.
The value of variable $i$ is computed as $i = (\text{lo} + \text{hi})/2 = (2 +
3)/2 = 2$, because division in C rounds the value down. The fix is to change
the line $9$ to \texttt{lo = i + 1}. With the debugger, it was simple to find
out where the error came from and we didn't have to recompile the program.

\begin{figure}
\begin{minted}{c}
   5   	    int hi = len;
   6   	    while (lo < hi) {
   7   	        int i = (lo + hi)/2;
-> 8   	        if (arr[i] < n) {
   9   	            lo = i;
   10  	        } else if (arr[i] > n) {
   11  	            hi = i;
Target 0: (a.out) stopped.
> p lo
(int) $0 = 0
> p hi
(int) $1 = 3
\end{minted}
    \caption{Debugging session in the LLDB~\cite{lldb} debugger, showing
    breakpoint hit report and printing variable values.}
    \label{fig:lldb-debug1}
\end{figure}

We previously said that processors themselves only understand machine code. So
how it is possible that we can debug the program and the debugger knows about
lines, variables, etc. when the program itself is just machine code. The
compiler has to lend a hand here. It embeds information about the source code.
For example, it maps lines of source code to machine code instructions. Thanks
to this mapping, the debugger knows that line $x$ corresponds to instruction
$y$ in the machine code and can put a breakpoint there. If the compiler doesn't
emit any information into the executable the debugger would only work with
assembly, as seen in figure \ref{fig:lldb-debug2}. This is a lot more
discomforting than debugging source code directly.

\begin{figure}
\begin{lstlisting}
->  0x100003e3c <+112>: b      0x100003e78               ; <+172>
    0x100003e40 <+116>: ldr    x8, [sp, #0x20]
    0x100003e44 <+120>: ldrsw  x9, [sp, #0xc]
    0x100003e48 <+124>: ldr    w8, [x8, x9, lsl #2]
\end{lstlisting}
\caption{Example of debugging a program in the LLDB debugger without debugging
    information generated by the compiler.}
\label{fig:lldb-debug2}
\end{figure}

\section{Teaching Compilers}
Many schools about computer science have a compiler course, and the Faculty of
Information Technology, CTU is no exception. The course is called \textit{Code
Generators} (NI-GEN). In this course, students are tasked to build a simple
compiler from a C-like language called TinyC. The target of the compiler is the
Tiny x86 (T86) architecture. This architecture does not have a processor that
implements it. Instead, a virtual machine, a program that reads the assembly
and executes it, was created for it. The architecture is supposed to ease the
code generation and let the students focus on the more interesting parts of the
compiler, like register allocation or optimization, instead of the nitty gritty
detail of real CPU architectures.

There is, however, a problem with using T86, as it has almost non-existing
debugging support. So if a compiler of some student generates the code badly,
which is frankly inevitable, it takes a non-trivial amount of effort to find
the error. T86 has some very light debugging capability, but it is far from
real debugging. Also, compiling debugging information is not taught in the
NI-GEN course because there is no reason to as of now. If a debugger was
provided to the students, it might be incentivizing to compile such information
to ease their lives later, when they will need to find errors in their
compilers. This way, they will also learn how and what information the compiler
needs to embed for the debugger to work.

\section{Goals of the Thesis}
The primary goal is to add debugging support to the T86 and to create a
debugger that supports debugging both on the machine code level and on the
source code level. The debugger should be extensible enough to also work with
an intermediate representation. The debugger should be similar to real-world
debuggers, in terms of how it works. This will require non-trivial changes in
the T86 virtual machine source code. The students' compilers will also have to
generate debugging information. The format of the debugging information should
be so that it is not discouraging for students to generate but also comparable
to debugging information generated by real compilers.

\section{Structure of the Thesis}
\begin{enumerate}
    \item The \textit{Introduction} is the motivation behind the thesis and
        introduces basic terms with which should the reader be familiar.
    \item \textit{Debugging support} describes how are debuggers implemented
        and what debugging support is provided on various levels (OS,
        processors, compilers) to make their implementation possible.
    \item \textit{Tiny x86} describes the T86 architecture and discusses some
        of the parts of the virtual machine, mainly its existing debugging
        capabilities.
    \item \textit{Implementation} focuses on extending the T86 instruction set
        and adding a debugging interface to the virtual machine. It also
        describes the implementation of the debugger and the format of the
        debugging information chosen.
    \item \textit{Evaluation} evaluates the performance of the debugger and its
        ease of use.
    \item \textit{Conclusions} summarizes the result of the thesis and speaks
        of possible future work.
\end{enumerate}

\chapter{Debugging support}

\begin{quote}
  \textit{Debugging is twice as hard as writing the code in the first place. Therefore, if you write the code as cleverly as possible, you are, by definition, not smart enough to debug it.}\begin{flushright}
    \tiny{Brian W. Kernighan}
  \end{flushright}
\end{quote}

We already mentioned in the first chapter that to debug programs written in high level programming languages, we need the compiler to emit debugging information.
But debugging support must also be provided by operating system (if there is one) and by processor itself.
However, we can still debug programs in machine code. We briefly mentioned this in chapter 1 \todo{ref},
however, we won't see source code, but only assembly. This can still be useful, for example
for reverse engineering. And, as was already hinted, source level debugging is built upon
assembly debugging. In this chapter, we will describe on which levels must assembly level debugging
be supported. Additionaly, we'll discuss how are compilers and debuggers able to allows us 
to debug source code althrough the programs are still machine code programs. 

\section{Support on CPU level}
In first chapter it was said that CPU can only execute machine code which is made of instructions
and that it has certain registers. Which instructions and registers the CPU has can differ from CPU to CPU.
This is specified by \textit{Instruction Set Architecture} (ISA) \cite{aps-isa}. It is an abstract interface between
the hardware and lowest level software (machine code). It contains all information needed to write a program in machine
code. In general, ISA specifies following:
\begin{itemize}
    \item Set of machine code instructions - Specifies instructions the ISA has and what operands each instruction has.
    \item Register set - Which registers the ISA has\footnote{Strictly speaking ISA doesn't have to use registers. It's possible to use only stack or accumulator, but most used ISAs use registers, so we'll ignore those architectures. }.
    \item Addressing modes - Possible methods to refer to memory or register.
\end{itemize}
There are other specification, however they are not relevant to this thesis. For each instruction and operand there is
specified how they should be encoded into binary (remember, that's what the CPU can understand).
CPU then \textit{implements} some ISA. If two different CPUs implements the same ISA then they should be able to run the same machine code program.
For example, most PC use the \textit{x86} architecture \cite{aps-isa}, althrough the ARM architecture is also seeing use in personal computers, for example the Apple Sillicon is of ARM architecture.

The x86 architecture is so called \textit{Complex Instruction Set Architecture} (CISC).
It contains many instructions that do many things at once, have varying length and takes multiple clock cycles to complete \cite{intel-manual}.
On the other hand the ARM architecture is \textit{Reduced Instruction Set Architecture} (RISC).
The number of instructions is smaller, they are intended to be small building blocks from which complex operations may be created by using many of them. Each instruction in RISC also has same length. Both architectures have their pros and cons, althrough some literature suggest that in modern days the choice of architecture is irrelevant if one is only considering performance and power consuption  \cite{riscvscisc1, riscvscisc2}. Unless specified otherwise the rest of this chapter will be talking about x86. This is because T86 \todo{ref} is loosely based on x86, so it is most relevant for us.

In first chapter we briefly mentioned that machine code programs can instead be written in Assembly language.
Assembly is almost 1:1 mapping to machine code. When showing programs, we will show them in assembly.
In figure \ref{fig:assembly-example2}, we present another example of a program that was compiled from C
to assembly of the x86 architecture.
As seen, instructions have various operands. Most often registers (\texttt{RBP, RSP, EAX}), memory (\texttt{[rbp-4]} is reference to memory at address which
is in register \texttt{rbp} minus $4$), or labels (like \texttt{L2}). Labels are not part of machine code, instead
memory address has to be provided. This is a small part where assembly and machine code differ.
For detailed overview of the x86 instructions see \cite{intel-manual}.

\begin{figure}\label{fig:assembly-example2}
    \begin{lstlisting}
max:
    push    rbp
    mov     rbp, rsp
    mov     QWORD PTR [rbp-24], rdi
    mov     DWORD PTR [rbp-28], esi
    mov     rax, QWORD PTR [rbp-24]
    mov     eax, DWORD PTR [rax]
    mov     DWORD PTR [rbp-4], eax
    mov     DWORD PTR [rbp-8], 1
    jmp     .L2
.L3:
    mov     eax, DWORD PTR [rbp-8]
    cdqe
    lea     rdx, [0+rax*4]
    mov     rax, QWORD PTR [rbp-24]
    add     rax, rdx
    mov     eax, DWORD PTR [rax]
    cmp     DWORD PTR [rbp-4], eax
    cmovge  eax, DWORD PTR [rbp-4]
    mov     DWORD PTR [rbp-4], eax
    add     DWORD PTR [rbp-8], 1
.L2:
    mov     eax, DWORD PTR [rbp-8]
    cmp     eax, DWORD PTR [rbp-28]
    jl      .L3
    mov     eax, DWORD PTR [rbp-4]
    pop     rbp
    ret
    \end{lstlisting}
    \caption{Compiled C program with GCC 9.4 compiler as x86 assembly.}
\end{figure}

\subsection{Registers}

The x86 architecture has a set of general purpose registers.
Some of these are
\begin{itemize}
    \item RAX - Accumulator for operands and results data,
    \item RCX - Counter for string and loop operations,
    \item RSP - Stack pointer,
    \item RBP - Pointer to data on the stack.
\end{itemize}
The names and number of general purpose registers change based on bit mode. 64-bit (also named x86-64) mode has 16 of them, while 32-bit has 8.
Althrough the \texttt{RSP} and \texttt{RBP} are called general purpose they are often only used for pointing at the top of the stack,
resp. to the base of the stack. Stack is special part of program memory, it mostly has LIFO
semantics\footnote{For example the \texttt{mov eax, DWORD PTR [rbp - 4]} does not respect
LIFO semantics, because it reads directly from the stack and not from the top.}
It can be used to store intermediate result, arguments to functions, return address etc.
This register is weird in a sense that it has this very special purpose but is still considered part of
the general purpose registers~\cite{intel-manual}. One might use it for storing calculations, but it would make
rest of the instructions that work with stack behave unexpectedly.

Instruction pointer register (RIP on x86-64, EIP on x86) contains address of the current instruction to be executed. As we mentioned
in Introduction \todo{ref}, programs are executed sequentially from top to bottom, with certain instructions
having the ability to change the control flow. When an instruction get executed, the size of the instruction
will be added to the value in RIP register. This will advance the instruction pointer to the next instruction.
Or, if the instruction changes control flow, the value in instruction pointer will be changed to the destination
of the instruction. The register can also be changed directly.

Another interesting register is the \texttt{EFLAGS} register. The register contains group
of flags, which can alter various behavior of the CPU, or the CPU itself sets them
as result of some instruction. For example the instruction \texttt{cmp} compares its two operands
and if they are the same the \textit{zero} flag in the \texttt{EFLAGS} register will be set.

\subsection{Interrupts}
Interrupt is a special request to the CPU to stop execution of current program and to quickly react to
the reason that caused the request \cite{aps-interrupts}. Example of such event can be keyboard press or error in an program (division by zero).
There are two main categories \cite{intel-manual}
\begin{itemize}
    \item An \textbf{interrupt} is an asynchronous\footnote{Meaning that the interrupt may happen when another instruction is being processed (not on the CPU clock edge).} event that is typically triggered by an Input/Output (IO) device.
    \item An \textbf{exception}\footnote{Unfortunately, this term will become quite overloaded in this thesis.} is a synchronous event that is generated when the processor detects one or more
          predefined conditions when executing an instruction. These are further divided into three classes: faults, traps and aborts.
\end{itemize}

When an interrupt or exception happens, the processor halts execution of current program and switches to specific
interrupt handler. Interrupt handler is just another sequence of instructions that handles the interrupt.
Example of an exception is the \texttt{INT3} instruction.
When this instruction is executed an interrupt is generated. This instruction is specifically meant to be used as a breakpoint.
We can supply code that will be responsible for handling the breakpoint as the interrupt handler.
However, on modern PCs a Operating System (OS) is governing the PC. Alas, we cannot touch the interrupt handler directly.
Instead, an OS is going to have to provide another layer of support for debugging.

Recall the EFLAGS register mentioned in section \ref{X}. There is a special flag called trap flag. When it is set, cpu will issue an interrupt
after every executed instruction. This could be useful if we wanted to inspect execution instruction by instruction.

\todo{Debugging embedded}

\section{Operating system support}
Operating system is a layer between computer components (cpu, memory, input/output devices, \dots) and software. It is responsible for
handling all those resources so programmers do not have to think about it \cite{modern-os, os-concepts}. Managing resources is not only to make writing programs easier, but to make sure that they are
safe from each other. Modern operating system allows to run multiple programs at once (or at least offer the illusion that it can) and they make sure that
one program cannot overwrite data or otherwise interfere with other programs. Normal programs runs in so called \textit{user space}, which has limited capabilites.
Kernel on the other hand runs in \textit{kernel space}. It has full access to hardware of the computer, can use all instructions, can permit or mask interrupts and so on.

However, if programs were limited to user space all the time they would be very limited.
Sometimes, they need to escape the confiment of the OS, for example to read a file or communicate with other processes.
Operating systems provide an interface through which the user space program
can leverage small part of the kernel - system calls. They offer a way of requiring some service from the OS.
This API is often in form of C and C++ functions~\cite{os-concepts}. A part of these functions is a special instruction, like \texttt{SYSCALL} on x86~\cite{intel-manual}, that
switches the mode to kernel space. The kernel has to check if the call is correct, since it will
be executed in kernel space with full access.

The most prelevant operating systems today are Microsoft Windows, Linux and MacOS.
Linux and MacOS systems are somewhat similar, but Windows is very different.

\subsection{Linux}
Linux offers special system call which is very handy for debugging. It is called \texttt{ptrace} \cite{ptrace} - process
trace. It has following signature: \texttt{ptrace(PTRACE\_COMMAND, pid, ...)}. It takes a \texttt{PTRACE\_COMMAND},
which specifies the behaviour of the function (for example \texttt{PTRACE\_SINGLESTEP} for single step), pid of some
process and some other parameters, depending on the \texttt{PTRACE\_COMMAND} that was chosen.
It allows to observe and control the execution of another process, this process will be the debugee.
In the context of \texttt{ptrace}, we will instead use the word tracee, to be consistent with ptrace documentation.

\mintinline{c}{ptrace} has many commands, here are some of the most important:
\begin{itemize}
    \item \texttt{PTRACE\_PEEKTEXT, PTRACE\_PEEKDATA} - Read tracee's memory,
    \item \texttt{PTRACE\_POKETEXT, PTRACE\_POKEDATA} - Write into tracee's memory,
    \item \texttt{PTRACE\_GETREGS} - Read tracee's register values,
    \item \texttt{PTRACE\_SETREGSET} - Modify tracee's register values,
    \item \texttt{PTRACE\_GETSIGINFO} - Retrieve information about the signal that caused tracee to stop,
    \item \texttt{PTRACE\_CONT} - Restart the stopped tracee process,
    \item \texttt{PTRACE\_SINGLESTEP} - Restart the stopped tracee but stop it after executing one instruction.
\end{itemize}

Linux however needs some way of notifying the debugger that the tracee encountered a breakpoint, or that some other
event requiring debugger attention happened. To this end, \textit{signals} are used.
They are in principle similar to CPU interrupts. They are however on the OS level.
A signal is used in UNIX and Linux systems to notify a process that a particular event has occured \cite{os-concepts}.
Signals can be sent to processes. When such process receives a signal, it stops its execution and starts
the execution of a signal handler. There are various signal types. Most signals can have custom signal handler
defined by the process. If no handler is defined then a default one is provided by the OS.
However handlers for \texttt{SIGKILL} and \texttt{SIGSTOP} cannot be changed \cite{signals}.

Reason for rising a signal can be \todo{Tady toho asi bude vic}
\begin{itemize}
    \item CPU Interrupt (Division by zero, Breakpoint hit),
    \item System call (\textit{kill(pid, signal)}).
\end{itemize}

For example, the signal \texttt{SIGTERM} can be send to a process to ask it nicely to exit.
The process can handle this request, for example to save some state before exiting.
It can also however be completely ignored. For this, a signal \texttt{SIGKILL} can be used,
which cannot be handled, ignored or blocked.



\subsection{Windows}
Windows also has built-in support for debugging at the Win32API layer \cite{windows-msdn-debugging-api, windows-press-debugging-api}.
It builds on \textit{debug events} and \textit{debug functions}. Summary of some of the functions that Win32 API offers which all help with debugging:

\begin{itemize}
    \item \mintinline{c}{DebugActiveProcess} - Attaches the debugger to an active process.
    \item \mintinline{c}{DebugBreakProcess} - Causes a breakpoint exception to occur in the specified process.
                                          This passes control of the process to the debugger if there is one.
    \item \mintinline{c}{WaitForDebugEvent} - Waits for new debug events.
    \item \mintinline{c}{ContinueDebugEvent} - Continue the process execution after processing debug event.
    \item \mintinline{c}{OutputDebugString} - Sends a string to the debugger for display.
    \item \mintinline{c}{ReadProcessMemory} and \mintinline{c}{WriteProcessMemory} - Read and modify process virtual address space.
    \item \mintinline{c}{FlushInstructionCache} - Flushes instruction cache of the process.
\end{itemize}

The general structure of Windows debugger can be seen in figure \ref{fig:win32debugger}.
The debugger waits for debug events via function \mintinline{c}{WaitForDebugEvent}.
This function has a timeout parameter, so the debugger can also do other things while it's waiting.
These events are put in a queue, so the debugger will not miss any.

\begin{figure}
    \centering
    \scalebox{0.8}{
    \begin{tikzpicture}
        \draw (-7,0) -- (-7,-11) (0,0) -- (0,-11) (7,0) -- (7,-11);
        \node at (-7,.3) {Debugee};
        \node at (0,.3) {Win32 API};
        \node at (7,.3) {Debugger};
        \draw[<-] (0,-1) -- node[midway,above] {\mintinline{c}{CreateProcess}} (7,-1);
        \draw[<-] (-7,-2) -- node[midway,above] {Create} (0,-2);
        \draw[->] (0,-3) -- node[midway,above] {\mintinline{c}{CreateProcess} returns} (7,-3);
        \draw[<-] (0,-5) -- node[midway,above] {\mintinline{c}{ContinueDebugProcess}} (7,-5);
        \draw[<-] (0,-6) -- node[midway,above] {\mintinline{c}{WaitForDebugEvent}} (7,-6);
        \draw[dashed,->] (-7,-6.5) -- node[midway,above] {Exception} (0,-6.5);
        \draw[->] (0,-7) -- node[midway,above] {\mintinline{c}{WaitForDebugEvent} returns \texttt{true}} (7,-7);
        \draw[dashed, <-] (-7, -8) -- node[above left] {Debugger actions} (7, -8);
        \draw[<-] (0,-9) -- node[midway,above] {\mintinline{c}{ContinueDebugProcess}} (7,-9);
        \draw[<-] (0,-10) -- node[midway,above] {\mintinline{c}{WaitForDebugEvent}} (7,-10);
        \draw[dashed,->] (-7,-10.5) -- node[midway,above] {Exception} (0,-10.5);
    \end{tikzpicture}
    }
    \caption{A sequence diagram for debugger using Windows api. Inspired by \todo{NI-REV 6. lecture}}
    \label{fig:win32debugger}
\end{figure}

The debug events are thoroughly described in subsection \ref{section:Debug Events}. The main point of interest is the exceptions.
By these, we do not mean the standard C++ exceptions but rather Microsoft \textit{Structured Exception Handling}.

\subsubsection*{Debug Events}\label{section:Debug Events}
Debugging events are various incidents in the debuggee that causes the system to notify the debugger \cite{windows-msdn-debug-events}. These are stored in special \mintinline{c}{DEBUG_EVENT} structure, which is received in \texttt{WaitForDebugEvent} call from debugger. This structure contains various information about the event, the internals can be seen on figure \ref{fig:DebugEvent}. These events include loading and unloading a DLL, creating and exiting a process, sending debug strings via the \mintinline{c}{OutputDebugString} and so on. It also includes exceptions, those are probably the most important for us. 
\begin{figure}
\begin{minted}{c}
typedef struct _DEBUG_EVENT {
  DWORD dwDebugEventCode;
  DWORD dwProcessId;
  DWORD dwThreadId;
  union {
    EXCEPTION_DEBUG_INFO      Exception;
    CREATE_THREAD_DEBUG_INFO  CreateThread;
    CREATE_PROCESS_DEBUG_INFO CreateProcessInfo;
    EXIT_THREAD_DEBUG_INFO    ExitThread;
    EXIT_PROCESS_DEBUG_INFO   ExitProcess;
    LOAD_DLL_DEBUG_INFO       LoadDll;
    UNLOAD_DLL_DEBUG_INFO     UnloadDll;
    OUTPUT_DEBUG_STRING_INFO  DebugString;
    RIP_INFO                  RipInfo;
  } u;
} DEBUG_EVENT, *LPDEBUG_EVENT;
\end{minted}
\caption{Structure which contains info about debug event.}
\label{fig:DebugEvent}
\end{figure}

\subsubsection*{Structured Exception Handling}
This feature is specific to Windows only. For example, if division by zero was performed in a program on Linux,
a signal would be sent to the process. Windows don't have signals, instead, it uses Structured Exception Handling \cite{windows-msdn-seh}. 
From now on, we will be using the abbreviation 'SEH'.
An exception is an event that requires execution of code outside the normal flow of control. There are software exceptions,
like throwing an exception explicitly or by OS, and hardware exceptions, like the division by zero we mentioned.
Instruction with opcode \mintinline{c}{0xCC}, which is used for breakpoints, will also raise an exception. SEH unifies both of these things into one.

When an exception is triggered, control is transferred to the system. It saves the state of the thread and some other information.
This information can be used to continue execution from the point where the exception was thrown when it is resolved. It also
contains information about which type of exception was thrown, if execution can continue after handling the exception, address where the
exception occured and some others\footnote{See MSDN documentation \cite{windows-msdn-seh} for full detailed list}.
The system then searches for an exception handler which will handle the exception. The search is performed in this order:

\begin{enumerate}
    \item If the process is debugged the debugger is notified.
    \item If it is not or the debugger does not handle the exception, the frame-based exception handler is to be found\footnote{The handlers are not very important to us, see MSDN documentation if you're interested \cite{windows-msdn-seh}.}
    \item If no frame-based handler can be found, or no handler handles the exception, but the process is being debugged then the debugger gets notified once again.
    \item The system provides default handling, which is to terminate the program via \mintinline{c}{ExitProcess} most of the time.
\end{enumerate}

Here we see that every exception that occurs in the debuggee causes the debugger to be notified. Breakpoints are also caused by an exception, as was briefly mentioned before. There are two possible notifications to the debugger. The first is known as \textit{first-chance} notification \cite{windows-msdn-dbg-exc-handling}. The debugger can (and should) inspect the information about the exception and see if it was a breakpoint or single-step. These only occurs if the process is debugged (it wouldn't happen otherwise) and the debugger should handle them. If it is something else it can ignore the exceptions. When the program is continued via \mintinline{c}{ContinueDebugEvent}\footnote{This function has a special parameter, which is used to tell that the exception was or was not handled.}, the debugger is notified once again if no appropriate exception handler was found for the exception. This is known as \textit{last-chance} notification because if the debugger does not handle the exception the debuggee will be terminated. It gives the user a chance to debug why is his process terminating.

Here are some exceptions that tie into debugging:
\begin{itemize}
    \item \mintinline{c}{STATUS_BREAKPOINT} - Raised when a hardware-defined breakpoint was encountered. This includes the mentioned \mintinline{c}{INT3} instruction.
    \item \mintinline{c}{STATUS_SINGLE_STEP} - Raised when a single step was completed, ie. when instruction was executed and the trap flag is set.
\end{itemize}

\subsubsection*{Tying it all together}
Now we have all necessary building block to build a simple proof of concept Windows debugger. On figure \ref{fig:windows-debugger-mainloop}, you can see a basic idea of a main loop of the debugger. It waits for debug events and branches depending of the type of event. It needs not only handle exceptions, but other events also. For example if the debugee creates a thread that is something the debugger should be aware of. Modern debuggers trace all threads of the program.

\todo{Pridat dalsi figure kde je jak se hanndlujou tyhle blbosti} However, exceptions are the most interesting for us. There, breakpoint and single step handling should be done. On both of these, the debugger should handle the exception itself, so this is the \textit{first chance} notifications. There is also an \mintinline{DBG_CONTROL_C}, which happens on CTRL + C keyboard press. This should terminate the program. The debugger will pass the first chance and catch the last chance exception, so user has a final chance to look at the program state before it exits.

\begin{figure}
    \begin{minted}{c}
void EnterDebugLoop(const LPDEBUG_EVENT DebugEv)
{
   DWORD dwContinueStatus = DBG_CONTINUE; // exception continuation
   for(;;)
   {
      WaitForDebugEvent(DebugEv, INFINITE);
      switch (DebugEv->dwDebugEventCode)
      {
         case EXCEPTION_DEBUG_EVENT:
            // Handle exception debug events
         // Other debug events
      }
   ContinueDebugEvent(DebugEv->dwProcessId,
                      DebugEv->dwThreadId,
                      dwContinueStatus);
   }
}
\end{minted}
\caption{Windows debugger main loop}
\label{fig:windows-debugger-mainloop}
\end{figure}

\section{Compiler support}
\todo{Moved from introduction here for the time being}
When we talked about evolution of programming from machine code to assembly to higher level languages, we haven't
talked about how they are executed. Machine code can be directly executed by processor, as we said, it is a sequence
of binary. Assembly is text, processors don't understand text. But assembly can be mapped to machine code almost 
1:1\footnote{There are some exceptions, like labels. But translating them is not very difficult.}.

However, high level programming languages do not map 1:1 to assembly. Some are close to it, like C, while others
are miles away, like Haskell. But as was said, processors understand only machine code. To this end, programs that
can translate source code into machine code, were created. They are called compilers and the translation process is
called compiling. For example, for the C language one might use the GCC or Clang compilers.
On figure \ref{fig:compiler-structure} can be seen basic structure of a compiler \cite{dragon-book}. 

\tikzstyle{compilerblock} = [rectangle, draw, minimum width=6cm, minimum height=1cm] 
\tikzstyle{tables} = [rectangle, draw, minimum width=4cm, minimum height=1cm] 
\begin{figure}\label{fig:compiler-structure}
    {\centering
    \begin{tikzpicture}
    \node (lexer)[compilerblock]{Lexical analyzer};
    \node (syntax)[compilerblock,below=of lexer]{Syntactic analyzer};
    \node (semantic)[compilerblock,below=of syntax]{Semantic analyzer};
    \node (imc)[compilerblock,below=of semantic]{Intermediate Code Generator};
    \node (gen)[compilerblock,below=of imc]{Code Generator};
    \node (symbol)[tables, left=of semantic]{Symbol table};
    \draw[->] (lexer) -- node[below] {} (syntax);
    \draw[->] (syntax) -- node[below] {} (semantic);
    \draw[->] (semantic) -- node[below] {} (imc);
    \draw[->] (imc) -- node[below] {} (gen);
    \end{tikzpicture} 
    \par}
    \caption{Simplified structure of a compiler. Some parts were left out, like optimizations.}
    \label{fig:compiler_tikz}
\end{figure}

\subsection{Lexical analyzer}
The lexical analyzer groups separate symbols into groups. For example the code
\begin{minted}{c}
foo = bar(1 + 2);
\end{minted}
might be translated into tokens like this
\begin{lstlisting}[stringstyle=\color{black}]
<id:"foo"> <assignment-operator> <id:"bar"> 
<left-bracket> <int-number:1> <plus-operator> 
<int-number:2> <right-bracket> <semicolon>
\end{lstlisting}
The Syntactic analyzer then works with these tokens.

\subsection{Syntantic and semantic analyzer}
Syntactic analysis accepts tokens and processes them into other intermediate representation. This is
most often an abstract syntax tree (abbr. AST, figure \ref{fig:ast}). It also checks that the source code complies to the grammar of the language.
Semantic analysis then checks that the program is semantically consistent. For example that used variable
has been declared before.

\begin{figure}\label{fig:ast}
    \centering
    \begin{tikzpicture}[,shorten >=1pt,node distance=1.8cm,on grid,initial/.style={}]
    \node (assignment) {$=$};
    \node (foo) [below left =of assignment] {id:foo};
    \node (bar) [below right =of assignment] {call:bar};
    \node (plus) [below right=of bar] {$+$};
    \node (one) [below left =of plus] {$1$};
    \node (two) [below right =of plus] {$2$};
    
    \draw[-, above, scale=0.7] 
    (assignment)   edge node[scale=0.7, left, yshift=0.1cm] {lhs}  (foo)
     (assignment)  edge node[scale=0.7, right, yshift=0.1cm] {rhs}  (bar)
     (bar)         edge node[scale=0.7, right, yshift=0.1cm] {expr} (plus)
     (plus)        edge node[scale=0.7, right, yshift=0.1cm] {rhs}  (two)
     (plus)        edge node[scale=0.7, left, yshift=0.1cm] {lhs}  (one);
    \end{tikzpicture}
    \caption{Simplified example of an abstract syntax tree.}
    \label{fig:astgraph}
\end{figure}
 
\subsection{Intermediate code generation}
This part converts AST into some other representation, most commonly called IR\footnote{IR means intermediate representation.
AST is also intermediate representation, but if we use IR we mean this one.}. IR is closer to machine code, to be easily translated,
but retain some properties that makes it easier to work with it. There are many types of IR. One of the most popular compilers, LLVM, uses
single static assignment (SSA) \cite{llvm}. Example of LLVM IR can be found on figure \ref{fig:llvm-ir-example}. Compilers perform most
optimizations on this intermediate representation. 

\begin{figure}\label{fig:llvm-ir-example}
    \begin{minted}{llvm}
        define dso_local i32 @_Z6squarei(i32 %0) {
          %2 = alloca i32, align 4
          store i32 %0, i32* %2, align 4
          %3 = load i32, i32* %2, align 4
          %4 = load i32, i32* %2, align 4
          %5 = mul nsw i32 %3, %4
          ret i32 %5
        }
    \end{minted}
    \caption{Simplified example of LLVM IR.}
\end{figure}

\subsection{Code generation}
Here, IR is translated directly to the target machine code or possibly assembly. Even though IR can seem very similar to assembly,
there are still some things to take care of. For example SSA IR doesn't have registers, it uses unlimited number of variables.
Other architectures might have some other traits that differ it from the IR and they all have to be accounted for when generating code.

\subsection{Modularity of compilers}
The main advantage of using an IR is that there is a common ground for every language. Imagine we write a compiler for the C language.
We need to write all five parts from figure \ref{fig:compiler-structure}. If we later decided that we also want to create a compiler
for Haskell, we just need to write everything up to the IR translation. Once we can translate Haskell into the IR, we can reuse the
previous part of the compiler to compile to machine code! This also works the other way around. If we compiled IR to the machine code
that works with the x86 architecture, and we want to compile to ARM, we just need to create the code generation part for the ARM architecture,
no need to write whole compiler. Also, most of the optimizations are done on the IR level, this also saves a lot of development time.
The parts of the compiler which are dependent on the source language are called \textbf{frontend} (Syntax, Semantic and IR translation), the parts that are dependent on
the target are called \textbf{backend} (Code generation).

This is widely used in practice. The LLVM \cite{llvm} project is a compiler backend. It uses its own IR (as was mentioned on figure \ref{fig:llvm-ir-example}).
It can compile this IR into many targets, including x86, ARM and Spark \todo{Ocitovat}. The \textit{Clang} project is a compiler frontend for C, C++ and Objective-C languages.
It translates these languages to the LLVM IR. Other frontends for LLVM also include \textit{ghc}, which is a Haskell compiler, or \textit{rustc}, which is a Rust compiler.
With LLVM, creating new programming language comes down to parsing it into an AST and transforming that AST into the LLVM IR.

\subsection{Interpreting programs}
Not all languages are compiled. Imagine a program which can evaluate arithmetic expressions, each phone nowadays has a program like this.
We don't have to stop there. Moving this up a notch, we can create a program that reads source code and executes it.
This is what interpreting means. Dynamically typed languages tend to be interpreted~\cite{python, lua, javascript}\todo{Instead of languages, cite some relevant source}, but it is not a rule~\cite{scala}. Since the interpreter is a program, it is another layer of abstraction. This can make
the resulting languages sometimes more abstract then the compiled ones, sometimes at the cost of performance~\cite{jit}.
Interpreters still \textit{compile} the code into some intermediate representation, but it's not compiled down to machine code, instead that IR is run by a program\footnote{Nowadays, interpreters use JIT compilation, which compiles some of the code some of the time into machine code~\cite{jit}}.

\chapter{Tiny x86}\label{section:T86}
At the FIT CTU, in the NI-GEN course, students have to write a compiler. The
Tiny x86 (T86) architecture was created to make code generation easier. This
allows the students to focus on more interesting parts of compiler design, such
as optimizations. To be able to execute programs written using this
architecture, a virtual machine was created. This was all done as a part of a
master's thesis made by Ivo Strejc~\cite{ivo2021tiny}. This chapter explores
said architecture and the virtual machine. It also delves into the existing
debugging capabilities of said virtual machine.

\section{The T86 Instruction Set Architecture}
The primary goal of the T86 architecture is to be an educational one. The
Virtual Machine made for the ISA allows configuring the number of registers,
the RAM size, or the length of each instruction. This allows the students to
develop their compilers incrementally.

The T86 uses a Harvard architecture, meaning the data and instructions are
physically separated. Memory is addressable by 64-bit blocks, not by 8-bit
blocks, as is the custom in modern computers. It shares some of the registers
we saw on x86-64: the program counter (PC), the stack pointer (SP), the base
pointer (BP), and the flags register. The intended roles for these registers
are the same as on x86-64 architecture. It also has other general-purpose
registers. As previously said, the amount of these is configurable. The
registers stores 64-bit values. It also has float registers, which can store
64-bit float values. These are separated from the standard registers, similarly
to x86-64.

The addressing modes, or what kind of operands instructions can have, include
immediate values, registers, and memory accesses. They can also be combined in
various ways, like \texttt{[R0 + R1 * 2]}, or \texttt{[R0 + 10 + R1 * 2]}. The
addressing modes are, however, not arbitrary, \texttt{[R0 + R1 + R2]} is not a
correct addressing mode for any instruction in the T86 architecture. The
allowed addressing modes also change for each instruction. For example, the
\texttt{MOV} instruction allows a vast range of addressing modes, while the
\texttt{PUSH} instruction only allows a register and an immediate number. For a
full list, refer to~\cite{ivo2021tiny}. The instructions that are taken over
from x86-64 often have more restrictive addressing modes than in x86-64. For
example, the \verb|add| instruction can take a memory offset as a destination
operand in x64-86, but in T86, it can only take registers.

Other than that, the ISA is mainly a subset of the x86-64 most used
instructions, many of which we have already seen in various examples throughout
the thesis. Interesting exceptions are the IO instructions - \texttt{PUTCHAR}
and \texttt{GETCHAR}, which allow for very primitive input and output handling.
Also, a \texttt{DBG} and \texttt{BREAK} instructions are defined. These are
used for debugging, but in a very different way than we have seen in previous
sections. We will touch upon them when discussing the virtual machine
implementation since they are very much tied to it. A small sample of a T86
program can be seen in figure~\ref{fig:t86-example}.

\begin{figure}
    \begin{lstlisting}
MOV  R0, 1
MOV  R1, 50
ADD  R0, R1
MUL  R0, 5
HALT
    \end{lstlisting}
    \caption{A small example of a program in the T86 architecture.}
    \label{fig:t86-example}
\end{figure}

\section{T86 Virtual Machine}\label{section:t86-vm}
The primary objective of the virtual machine is to replicate the CPU as
accurately as possible, without prioritizing execution speed. For instance, the
virtual machine simulates the out-of-order technique briefly described in
section~\ref{section:superscalar-cpu}. The purpose of the virtual machine is to
allow the students to gain a deeper understanding of the effects of pipeline
stalls and similar events on program speed. The virtual machine is able to
generate statistics that provide information about these factors and how
much they influenced the speed of the generated program.

The virtual machine (VM) is implemented in C++, using the newer standards up to
C++17. The VM offers only a single interface, and that is the
\texttt{ProgramBuilder}. This is a class through which one may construct a
program for the T86 VM. An example of how to use this class is in
figure~\ref{fig:t86-intro}. Currently, there is no other way for users to run
programs in the VM. This means that the students are tied to the C++ language,
or use some bindings if they want to use other language.

The \texttt{Cpu} class is the backbone of the interpreter. It is responsible
for running the program. It has a \texttt{halted} function, which returns true
if the \texttt{Cpu} executed a \texttt{HALT} instruction. The \texttt{tick}
performs one tick of the CPU. This does not mean that one instruction gets
executed. The \texttt{Cpu} simulates a superscalar CPU, so one tick is one move
in the pipeline. The \texttt{while} loop in the figure~\ref{fig:t86-intro}
shows to run a T86 program via the \texttt{Cpu} class.

\begin{figure}
    \begin{minted}{cpp}
        ProgramBuilder pb;
        pb.add(MOV{Reg(0), 50);
        pb.add(PUTCHAR{Reg(0)});
        pb.add(HALT{});
        auto program = pb.program();
        Cpu cpu;
        cpu.start(std::move(program));
        while (!cpu.halted()) {
            cpu.tick();
        }
    \end{minted}
    \caption{Simple example of how to create and run a simple program using the
    T86 virtual machine.}
    \label{fig:t86-intro}
\end{figure}

\subsubsection{Debug Instructions}\label{section:t86-debug-cap}
The VM offers some limited debugging capabilities. It has the \texttt{DBG} and
\texttt{BRK} instructions. The \texttt{DBG} instruction takes as an operand a
function. The function has the following signature: \texttt{void fun(Cpu\&)}.
This function then gets executed when the instruction is hit. This can prove
helpful in inspecting the internal state of the CPU. In
figure~\ref{fig:t86-debug}, we show a possible usage of this instruction. The
\texttt{BRK} instruction works similarly. It, however, has no operand. Instead,
a function must be provided before execution to the CPU itself. \texttt{BRK}
then always runs this function when hit.

\begin{figure}
    \begin{minted}{cpp}
    pb.add(DBG{[](Cpu& cpu) {
        if (cpu.getRegister(Reg{0}) == 0) {
            std::cerr << "Register 0 is set to zero!\n";
        }
    });
    \end{minted}
    \caption{Adding a \texttt{DBG} instruction into the T86 program using the
    \texttt{ProgramBuilder}.}
    \label{fig:t86-debug}
\end{figure}

Such debugging capabilities can be helpful but quickly prove insufficient. For
example, when a step-by-step inspection is sought, a debug instruction must be
placed at every second line of the program. Also, interactivity is not present.
However, this function can accept input, so one could create a robust enough to
handle register and memory writing. An idea of how this could be done is
illustrated in~\ref{fig:t86-pocket-debugger} via the \texttt{BRK} instruction.

\begin{figure}
    \begin{minted}{cpp}
    cpu.connectBreakHandler([](Cpu& cpu) {
        char command;
        std::cin >> command;
        if (command == 'c') return; // continue
        else if (command == 'r') { // Read register
            int num;
            std::cin >> num;
            int regval = cpu.getRegister(num);
            std::cerr << std::format("Register {} = {}\n",
                                     num, regval);
        } else if (command == 'w') { // Write register
            int num;
            int val;
            std::cin >> num >> val;
            cpu.setRegister(Reg{num}, Reg{val});
        }
        ... // Other commands
    });
    \end{minted}
    \caption{Small debugger implementation using T86 \texttt{BRK} instruction,
    abbreviated.}
    \label{fig:t86-pocket-debugger}
\end{figure}

This still leaves much to be desired. Not to mention placing the debug
instruction can prove very bothersome.

\chapter{Implementation}
In this chapter, we describe how we went about the implementation of the
debugger and reason about the design choices we made. Also, we describe which
parts of the virtual machine were modified or added to allow the
implementation of the debugger.

\section{T86 ISA Extensions}\label{section:parser}
In chapter~\ref{section:t86-vm}, we showed how to build a program for the T86
VM with the existing builder interface. To allow the usage of other programming
languages, we have created an ELF-like format for the T86 executables. The
format is a text one, making it easy to use. An example of a program in said
format is shown in figure~\ref{fig:t86-program}. It is very similar to the
assembly we have shown in previous sections. Thanks to this, students can
implement their compiler in any programming language they want and emit the T86
program in this format as a text file. An unfortunate side effect is that we
can no longer use the \texttt{DBG} instruction. However, the debugger we will
later present will be much more powerful than the \texttt{DBG} instruction.

As can be apparent from the example, we also use sections. The \texttt{.text}
section is the only mandatory one. It contains the instructions that will be
executed. Another one is the \texttt{.data} section. Here, either raw numbers
or strings can be written. The contents of this section are then loaded by the
VM and stored into the memory, beginning at memory cell 0 and upwards. There
are also debug sections, which we will present when discussing debugging
information.

\begin{figure}
    \begin{lstlisting}
.data
"Hello, World!\n"

.text
0   MOV [BP - 1], 0
1   JMP 8
2   MOV R0, [BP - 1]
3   MOV R1, [R0]
4   PUTCHAR R1
5   MOV R0, [BP - 1]
6   ADD R0, 1
7   MOV [BP - 1], R0
8   MOV R0, [BP - 1]
9   CMP R0, 13
10  JLE 2
11  HALT
    \end{lstlisting}
    \caption{Example of an T86 program which prints "Hello, World!".}
    \label{fig:t86-program}
\end{figure}
We will also add two new instructions. First is the \texttt{PUTNUM}
instruction, which prints the numerical value in the register and a newline.
This is intended as a very primitive debug instruction and to ease the
automated testing of the compiler. The only other way of output was to print a
char which was represented by the ASCII value. With this instruction, students
can bootstrap and test the basic implementation of their compiler more easily.

Another one is the \texttt{BKPT} instruction. This instruction is similar to
the \texttt{INT3} instruction from x86-64 or the \texttt{BKPT} instruction from
ARM. It is a software breakpoint. The virtual machine has no support for
interrupts, which are needed for debugging to work. This will be the focus of
the next section.

\section{T86 Debugging Support}
We could bake the debugger into the virtual machine itself, which would likely
be the simplest way to implement it. However, the goal of the debugger is not
only to ease the code generation part but to be a learning point so that
students might grasp how a real debugger works\footnote{The VM followed the
same philosophy.}. Because of this, we aim to simulate the real-world debuggers
as closely as possible. The compilers may also have more targets in the future,
not just the T86 VM. If we made the debugger part of the T86 VM, we could not
use it for a possible new virtual machine. In conclusion, the virtual machine
and the debugger will be two entirely different programs and as such, two
completely different processes.

In the debugger implementation for Linux, the subject of section
\ref{section:linux-dbg}, we described how an operating system's kernel allows
the debugger's implementation via a specific API. There is no operating system
between the virtual machine and the program. Still, we will strive to make the
API similar to the ptrace API. The debugger and the VM will have to communicate
together somehow. For interprocess communication, there are several
possibilities.

Both the VM and the debugger use an abstract class representing an interface
that provides two methods, \texttt{Send} and \texttt{Receive}. The
implementation of this interface then handles the concrete way of
communication. The debugger and VM do not care about it; they merely use these
two methods. There are currently two implementations of this interface. One is
using network communication through sockets. This way, the debugger may attach
to an existing process, even on an entirely different computer. It, however,
has a disadvantage. The messages sent are often short and we need to send a lot
of them. This proved too slow, even with few messages being sent. The second
implementation is via threads. The debugger runs the VM in another thread, and
they communicate via shared queues. This is far faster and allows the debugger
to run the process by himself, making it easier to use and behave like
real-world debuggers.

The format of the communication is a text one, merely because of the ease
of use as opposed to binary format, it is also clearer to see what is
happening. The commands that the virtual machine API offers are
\begin{itemize}
    \item \texttt{PEEKREG x} - Return values of all normal registers.
    \item \texttt{POKEREG x y} - Set the value in register \texttt{x} to
        \texttt{y}.
    \item \texttt{PEEKFLOATREG} - Return values of all float registers.
    \item \texttt{POKEFLOATREG x y} - Set the value in float register
        \texttt{x} to \texttt{y}.
    \item \texttt{PEEKDEBUGREG} - Return value in all debug registers.
    \item \texttt{POKEDEBUGREG x y} - Set the value in debug register
        \texttt{x} to \texttt{y}.
    \item \texttt{PEEKDATA x cnt} - Return value in memory at addresses $x$ to $\texttt{x} + \texttt{cnt} - 1$.
    \item \texttt{POKEDATA x y} - Writes a value \texttt{y} into a memory at
        address \texttt{x}.
    \item \texttt{PEEKTEXT x cnt} - Return instruction from \texttt{x} to $\texttt{x} + \texttt{cnt} - 1$.
    \item \texttt{POKETEXT x INS} - Rewrite the instruction at address
        \texttt{x} with the newly supplied instruction.
    \item \texttt{CONTINUE} - Continue the execution.
    \item \texttt{TERMINATE} - Stop the execution, terminating the virtual machine.
    \item \texttt{REASON} - Get the reason why the program stopped (breakpoint,
        singlestep, halt...).
    \item \texttt{SINGLESTEP} - Do native level single step.
    \item \texttt{TEXTSIZE} - Return the size of the program.
\end{itemize}

An example of how those commands can be used for communication between the
virtual machine and the debugger is shown in figure~\ref{fig:dbg-vm-seq}. The
interface is similar to basic ptrace commands. If the command should not return
anything, the VM sends back an \texttt{OK} message. We separate the memory and
instruction writing because T86 uses Harvard architecture, whereas Linux does
not separate text and data address spaces~\cite{ptrace}, so the two requests
were equivalent there. The API is made to be simple on purpose. Anything more
complex should be handled in the debugger itself.

\begin{figure}
    \centering
    \scalebox{0.8} {
    \begin{tikzpicture}
        \draw (0,0) -- (0,-20.2) (7,0) -- (7,-20.2);
        \node at (7,.3) {Debugger};
        \node at (0,.3) {Virtual machine};
        \draw[<-] (0,-1) -- node[midway,above] {Initializes connection} (7,-1);
        \draw[->] (0,-2) -- node[midway,above] {Accepts connection} (7,-2);
        \draw[<-] (0,-3) -- node[midway,above] {\texttt{"PEEKTEXT 5 1"}} (7,-3);
        \draw[->] (0,-4) -- node[midway,above] {\texttt{"MOV R0, 1"}} (7,-4);
        \draw[<-] (0,-5) -- node[midway,above] {\texttt{"POKETEXT 5 BKPT"}} (7,-5);
        \draw[->] (0,-6) -- node[midway,above] {\texttt{"OK"}} (7,-6);
        \draw[<-] (0,-7) -- node[midway,above] {\texttt{"CONTINUE"}} (7,-7);
        \draw[->] (0,-8) -- node[midway,above] {\texttt{"OK"}} (7,-8);
        \draw[->] (0,-10) -- node[midway,above] {\texttt{"STOPPED"}} (7,-10);
        \draw[<-] (0,-11) -- node[midway,above] {\texttt{"REASON"}} (7,-11);
        \draw[->] (0,-12) -- node[midway,above] {\texttt{"SW\_BKPT"}} (7,-12);
        \draw[<-] (0,-13) -- node[midway,above] {\texttt{"CONTINUE"}} (7,-13);
        \draw[->] (0,-14) -- node[midway,above] {\texttt{"OK"}} (7,-14);
        \draw[->] (0,-16) -- node[midway,above] {\texttt{"STOPPED"}} (7,-16);
        \draw[<-] (0,-17) -- node[midway,above] {\texttt{"REASON"}} (7,-17);
        \draw[->] (0,-18) -- node[midway,above] {\texttt{"HALT"}} (7,-18);
        \draw[<-] (0,-19) -- node[midway,above] {\texttt{"TERMINATE"}} (7,-19);
        \draw[->] (0,-20) -- node[midway,above] {\texttt{"OK"}} (7,-20);
    \end{tikzpicture}
    }
    \caption{A sequence diagram for the communication between the virtual
    machine and the debugger. If the label is enclosed in quotes, it is the
    actual text message that is being sent.} \label{fig:dbg-vm-seq}
\end{figure}

The \texttt{Cpu} class, which we have described in section~\ref{section:t86-vm},
has no support for interrupts. The \texttt{halted} method is kind of similar
to interrupts, but only allows for signaling the \texttt{HALT} instruction
execution. We need more than that.

We added another manager-like class called \textit{OS}. This class will take
care of running the program via the \texttt{Cpu} class. We also added an
\textit{interrupt} capability to the \texttt{Cpu}. To check if and which
interrupt happened, the \texttt{Cpu} now provides a function, similar to the
\texttt{halted} function. The OS calls the \texttt{tick} method periodically,
and after every tick, it checks if a halt or interrupt occurred. If it did,
then it passes it to some handler. When interrupt happens, unrolling must be
done to display proper values in registers and memory. This was described in
section \ref{section:superscalar-cpu}. The unrolling mechanism was fortunately
already implemented by the T86 VM author. It was used for the \texttt{DBG} and
\texttt{BRK} instructions, and we can use the same mechanisms for our addition
of interrupts.

The \texttt{BKPT} instruction we added is used for software breakpoints.
Executing this instruction causes an interrupt \texttt{3} to occur. It is also
possible to set a special flag that causes the \texttt{Cpu} to send the
interrupt \texttt{1} after every executed instruction. When an interrupt that
is caused by some debugging features happens, the \texttt{OS} calls a method
in the \texttt{Debug} class. This class is also a new addition and is
responsible for communication with the debugger. It uses the text protocol we
mentioned previously.

We also added debug registers. These are a special type of registers designed
for triggering breaks on memory access. There are a total of five debug
registers, with the first four containing the memory cell addresses. The fifth
register, called the control register, contains four bits that indicate the
status of each of the first four registers. If a register is active and the
program writes to a memory cell with the same address as is stored in the
register, an interrupt \texttt{2} is generated. Furthermore, the control
register's bits from 8 to 11 reveal which register caused the interrupt. For
instance, if bit 10 is set to 1, the third register is responsible for the
interrupt and the address stored in that register is the one that was written
into.

\section{Native Debugger}
The implementation is done in the C++ language. It uses newer standards up to
the C++20 standard. The debugger is implemented as a library. We will call this
the backend of the debugger. A command line interface was also developed,
through which the users might interact with the debugger. This will be called
the frontend of the debugger.

The implemented debugger consists of two main parts. The first one aims to
support native (instruction) level debugging. This part work without
\textbf{any} debugging information whatsoever. The second part focuses on
source-level debugging, and is described in
section~\ref{section:source-debugger}.

The native debugger is split into two additional layers to make it more
modular. The first layer is called a \texttt{Process}. It is an interface
representing the debuggee process. The implementation of this interface is
responsible for dealing with the concrete architecture, the API of that
architecture, and the communication with the debuggee. One implementation is
provided for the T86 VM. For instance, it has a method called \texttt{ReadText}
and \texttt{WriteText}. The internals of these methods use the
\texttt{PEEKTEXT} and \texttt{POKETEXT} API we described. Outside of this
class, the communication API is never used. If, in the future, another virtual
machine is made, for whichever architecture, it is only needed to implement
this interface. The rest of the debugger can be used as-is.

Another layer is the \texttt{Native} class, which implements the complicated
logic behind a debugger, like setting a breakpoint, handling single-step, and
so forth. It is the primary bread and butter of the native part of the
debugger. Most algorithms are similar to the Linux debugger implementation
presented in section \ref{section:linux-dbg}. For illustration, in figure
\ref{t86dbg:breakpoint} we show a snippet of code used to create a breakpoint.
It first reads the text at the address where we want to set the breakpoint. The
breakpoint opcode then rewrites this text, and the backup of the text is
stored.

When we arrive at the breakpoint and want to continue further, we need to unset
the breakpoint, i.e., replace the breakpoint opcode in the T86 program with the
backup we saved, do a native-level single step, and write the breakpoint back.

Since breakpoints change the underlying code of the debuggee, we need to be
careful when presenting information to the user. If we printed the text that we
get from the debuggee, it might contain the \texttt{BKPT} instructions we set
earlier. We need to mix it with the backup code stored in breakpoints to show
the assembly of the program correctly.

The Native class uses a \texttt{DebugEvent} structure which indicates what
caused the VM to stop. It is implemented as a \texttt{variant} of multiple
structures, for instance, the \texttt{BreakpointHit} or the
\texttt{WatchpointTrigger} structure. It is a variant because the watchpoint
also needs to convey information about an address that caused the break, as do
breakpoints. It could also signal if the break was caused by reading or writing
to the memory cell, although for now, the T86 VM only interrupts on writing.

\begin{figure}
    \begin{minted}{c++}
SoftwareBreakpoint CreateSoftwareBreakpoint(uint64_t address) {
    auto opcode = GetSoftwareBreakpointOpcode();
    // Read the text at the breakpoint address
    auto backup = process->ReadText(address, 1).at(0);
    // Rewrite it with the breakpoint opcode
    std::vector<std::string> data = {std::string(opcode)};
    process->WriteText(address, data);
    // Check that it was truly written
    auto new_opcode = process->ReadText(address, 1).at(0);
    if (new_opcode != opcode) {
        Error(...);
    }
    // Create a breakpoint object which keeps the text backup
    return SoftwareBreakpoint{backup, true};
}
    \end{minted}
    \caption{Debugger code in the \texttt{Native} class to enable a breakpoint.}
    \label{t86dbg:breakpoint}
\end{figure}

The native debugger has the following features:
\begin{itemize}
    \item Breakpoints - Can set, unset, enable and disable software breakpoints.
    \item Watchpoints - Can set and unset watchpoints on memory writes.
    \item Single stepping - Can do native level step into, which executes
        current instruction, out, which runs the program until it leaves
        current function and over, which treats function calls as a single
        instruction.
    \item Text manipulation - Can read and write into the debuggee text area,
        effectively allowing to overwrite the running code.
    \item Data manipulation - Can read and write into the program memory area.
    \item Register manipulation - Can manipulate with normal, float and debug registers.
\end{itemize}

\section{Debugging Source Code}\label{section:source-debugger}
With the solid foundation represented by the native part of the debugger, we
can extend it by providing some form of source-level debugging. For this part,
we need to remember that the debugger will only be used by students. As such,
we ought to have gentler debugging information than DWARF, but we certainly can
take inspiration from it.

As we previously mentioned, the executable with T86 code is separated into
sections. The \texttt{.text} and \texttt{.data} sections are for the VM. We
will introduce new sections where debugging information will be stored. All
those sections will have \verb|.debug_| prefix. The simplest new section is
\verb|.debug_source|, which should contain the original source code which was
compiled into this executable. This later allows us, with the combination of
other information, to display the source code lines.

The main philosophy of the source-level debugger is to allow an arbitrary
amount of debug information. For instance, the user can generate information
about one function only, and for that function, source debugging capabilities
will work, but not for any other. This means that users can generate debugging
information incrementally.

In the NI-GEN course, students are creating compilers from the TinyC language.
This is a small subset of the C language. It has simpler grammar and only
following types: \texttt{int}, \texttt{double}, \texttt{char}, pointers,
structured types, and static arrays. We aim to support all of TinyC language in
our source layer of the debugger.

The debugger is, however, not only limited to TinyC language. Any imperative
language that can be encoded with the following debugging information is
suitable to be debugged at the source level. We show an example of this in a
provided test case for the debugger where we debug the LLVM IR. It can be found
as an attachment to the thesis on path \texttt{impl/src/dbg-cli/tests/llvm.ir}.
We also provide a documentation for developers who wish to generate this
debugging information in the thesis attachment on path
\texttt{impl/docs/source-info.md}.

The logic behind the source level debugging is mostly handled by the
\texttt{Source} class. It also stores all of the source level mapping, which we
are about to describe below. Some of the methods work closely with the
\texttt{Native} class to achieve functionality. That shouldn't be surprising, as
we said, source-level debugging is built upon native-level debugging.

\subsection{Line Information and Source Code}
The line information is encoded in a table, where every row is:
\texttt{<line>:<address>}. This is far simpler than the DWARF way which we
described in section~\ref{section:line-number-information}. We do not care
about being space efficient, so we did a table instead of virtual machine
specification. We also feel like this should be an entry level debugging
information, so that students are not discouraged outright. It misses some
information that DWARF had, like columns. It still however proves quite
sufficient for most cases. The information should be stored in the
\verb|.debug_line| section.

With this information, we are able to do source-level breakpoints. If the
source code is also provided, we can show the user on which line is the
debugged program currently paused. It is not necessary to specify every line in
the program. The debugger will refuse to put a source-level breakpoint on some
line if it does not have the necessary information. The source code should be
under the \verb|.debug_source| section, which must be the last section in the
executable.

\subsection{Debugging Information Format}
In the line information, we provided a straightforward format. However, we will
need a more sophisticated structure to describe some advanced constructs of the
source code. We will draw inspiration from the Dwarf Debugging Information
(DIE). Take a look at figure~\ref{fig:t86dbg-die}, which shows an example of
such debugging information. It has a tree-like structure which, in some ways,
mimics the original program. The nodes of this tree are also called debugging
information entries (DIEs). Those entries can have other entries as their
children, and each entry has a tag that is part of its name (for example, the
\verb|compilation_unit| tag). They can also have attributes that describe their
properties. As can be seen, this is very similar to the DWARF debugging
information format. Unlike DWARF, it will be a text format. This allows us to
generate the format easily and to spot mistakes quickly. We don't want the
students to debug their generated debugging information.

For instance, the tag \verb|DIE_function| represents a function. As attributes,
it has a name, beginning address, and end address. With this additional
information, we can set a breakpoint on a function name. We can also display in
which function we are located when a break happens.

It also has one direct child, a \verb|DIE_scope|. The scope entry is mainly
used for keeping track of which variables are currently active because the T86
(or any other assembly language in general) has no notion of scopes. In the
scope entry in the example, only one variable called \texttt{d} exists. Thanks
to those entries, we can list currently active variables. We, however, often
need to examine the value of a variable. To achieve this, information about the
location and the variable type is needed. Global variables should be outside of
a function, as descendants of the \verb|compilation_unit| entry.

\begin{figure}
    \begin{lstlisting}
DIE_compilation_unit: {
DIE_function: {
    ATTR_name: main,
    ATTR_begin_addr: 0,
    ATTR_end_addr: 10,
    DIE_scope: {
        ATTR_begin_addr: 0
        ATTR_begin_addr: 10
        DIE_variable: {
            ATTR_name: d,
        },
    }
}
}
    \end{lstlisting}
    \caption{Debugging function information for the T86 debugger.}
    \label{fig:t86dbg-die}
\end{figure}

The type information is encoded as a standalone DIE. Currently, three type
entries are present, one for primitive types (\texttt{int}, \texttt{double}, or
\texttt{char}), one for pointers, and one for structured types (\texttt{struct}
or \texttt{class} in C++). Other types can be easily added in the future. The
types are saved as separate entries, and as such, we need some way to link them
together with the variables. We will use the \verb|ATTR_id| attribute to
achieve this. This attribute should be unique for every entry, having similar
role to the \texttt{id} attribute of HTML elements~\cite{html4}. The variables
themselves have the \verb|ATTR_type| attribute, which will have an id of the
type as its value. An example of a pointer type that points to an int type is
in figure~\ref{fig:t86dbg-types}. If we had a variable that is a pointer to
int, it would need to have the \verb|ATTR_type: 1| attribute because the id of
a pointer type to integer is one.

The primitive types need to have their size. For T86, this is the number of
memory cells it occupies, which will almost always be one since one memory cell
is 64 bits. It also has a name for its primitive type. Currently, three
are supported:
\begin{itemize}
    \item \texttt{int} - A signed integer.
    \item \texttt{float} - A floating point number.
    \item \texttt{char} - A number representing an ASCII character.
\end{itemize}

Additionally, we also support pointer types (including pointers to pointers),
static arrays and structured types, which are a bit more complicated. They need
to have a list of members which are stored in the structure. For each member,
an offset from the beginning of the structure must also be specified. It also
must provide a size because the compiler might align it, and it may be larger
than the sum of the size of its members.

\begin{figure}
    \begin{lstlisting}
DIE_primitive_type: {
    ATTR_name: int,
    ATTR_id: 0,
    ATTR_size: 1,
},
DIE_pointer_type: {
    ATTR_type: 1,
    ATTR_id: 1,
    ATTR_size: 1
},
    \end{lstlisting}
    \caption{Debugging type information for the T86 debugger, showing an
    \texttt{int} primitive type and a pointer to \texttt{int} type.}
    \label{fig:t86dbg-types}
\end{figure}

With this information, we can show the type of a variable. Nevertheless, the
most valuable thing is its value. Variables are either stored in memory,
registered, or optimized out completely. We will follow DWARF's footsteps and
provide a virtual machine specification.

The virtual machine is stack-based one. After all instructions are executed,
the value at the top of the stack represents the resulting location. It can
either be a register or an memory offset. We offer several examples of programs
for the virtual machine:
\begin{itemize}
    \item \texttt{PUSH R0} - Pushes the register \texttt{R0}. No instructions
        remain, so the resulting location is the \texttt{R0} register.
    \item \texttt{PUSH BP; PUSH -2; ADD} - Pushes the \texttt{BP} register and
        the \texttt{-2} offset onto the stack. The \texttt{ADD} instruction
        pops two values from the stack and adds them together. If the value is
        a register, the value that is stored in that register is taken. The
        result from the addition is pushed back onto the stack.
    \item \verb|BASE_REG_OFFSET| - Does the exact same chain of operations as
        the previous example. Since the variables are often stored in memory at
        some offset from the base pointer, we provide this short hand.
\end{itemize}

There is also a dereference instruction, which dereferences a value in memory.
That can be useful for tracking location of pointed variables. This virtual
machine is also easily extendable. With this kind of power, the location of
variables can almost be arbitrary and not only tied to a register or an offset
from the base register. This information is stored in a variable entry
attribute called \texttt{ATTR\_location}. If all this information is provided,
we know where the variable is stored and may look up its value. Together with
the type information, we might also properly interpret the value and report it
to the user.

\subsection{Source Expressions}
We could make a very straightforward implementation of getting variable value
by its name. It is only a matter of finding the variable entry with the correct
name and interpreting its location and type. However, we often need to inspect
some more complicated expressions. For example, we may want to display some
struct member or a value at which pointer points.

The debugger has a built-in interpreter for such expressions. It builds an AST
from the expression and interprets it using an
AST walk~\cite{crafting-interpreters}. The AST interpreter leverages the
\texttt{Native} class to fetch variable values. The interpreter supports almost
all C operators, including the assignment operator. It, however, has stricter
typing than C. An example of such an expression is \texttt{foo[2]->bar + 3}.
This is a very powerful feature, as it allows one to easily inspect or modify
various variables or expressions. If a user wishes to modify part of an array,
there is no need to use the raw memory or register setters. Instead, it is
possible to write \texttt{array[x] = y}.

The AST nodes are distinguished by types. They also need to store the location
of the expression  if it is an expression that can appear on the left-hand
side of the assignment operator. Examples of expressions that must store the
location: \texttt{a}, \texttt{*(a + 1)}, \texttt{a[5]}, while the following
expressions do not have any locations because they cannot appear on the left
side of the assignment operator: \texttt{a + 1}, \texttt{5}, \texttt{array[0]
* y}.

\section{Frontend}
We provide two command line interface programs. First one is for the T86, is
runs the given T86 program on the virtual machine. The second application
leverages the debugger library to create a command line interface for the
debugger. It provides many commands, and its manual can be found in the thesis
attachment under path \texttt{impl/docs/debugging.md}. In this file, a table
can be found showing how some of the commands the application accepts differ
from the GDB debugger.

The main priority of the CLI is to make the debugger easy to use. It consists
of several commands, one of them is \texttt{breakpoint set 5}, which will set a
source-level breakpoint on the fifth line of the program. It is, however, not
necessary to write the whole command. Any prefix will do, like \texttt{b s 5}.
The CLI leverages the \textit{linenoise}~\cite{linenoise} library to make the
REPL satisfying to use.

The CLI also displays various information on program stop, like why the program
stopped, on which address or line, and prints the surrounding lines of assembly
or source. The CLI can also list breakpoints and display their locations in the
disassembly or the source code. Figure~\ref{fig:cli-hit} provides an example of
a breakpoint hit report. The debugger had all debugging information available
here. It can show the line in the source code where the breakpoint happened,
name the offending function, and variables in scope.

When variable values are printed, the format tries to accomodate their type.
For example, variables that are arrays or pointers to the \texttt{char} type
print the value of the variable as a string literal.

\begin{figure}
    \begin{lstlisting}
Process stopped, reason: Software breakpoint hit at line 11
function main at 7-18; active variables: a, b
      9:    int b = 6;
     10:    swap(&a, &b);
@->  11:    print(a);
     12:    print(b);
     13:}
    \end{lstlisting}
    \caption{Example of the debugger CLI reporting a breakpoint hit.}
    \label{fig:cli-hit}
\end{figure}

\chapter{Evaluation}
This chapter aims to assess the effectiveness of the debugger. Speed is
measured to evaluate whether the debugger has any performance impact when
debugging standard and computative demanding programs. We also evaluate its
feature richness and ease of use, comparing it to state-of-the-art debuggers
like the GDB.

\section{The Development Process}
The development of the thesis was done in a GitHub repository. The power of
Github was leveraged not only for keeping history but also for recording
issues, planning the development, or ensuring that the repository is in a
consistent and working state via GitHub actions, which runs tests before a pull
request is accepted. The actions run on two operating systems, Ubuntu and
MacOS, ensuring the project works on both.

The project itself contains many tests. The code is first tested via many unit
tests using the \texttt{GoogleTest}\todo{citace?} framework. The unit tests
cover almost all parts of the code. It also has integration tests for T86 CLI
and the debugger CLI. Another student\todo{citace} created a TinyC to T86
compiler as part of his thesis. He was kind enough to send the implementation
to us so that we could generate various tests more easily. As a result, most of
the integration tests were generated by the said compiler. The integration
tests for the T86 CLI run the program and check if its output is the same as
the expected one. The debugger CLI is also checked against expected output.
Its input, however, is not only the file that will be debugged, but also the
series of commands that will be executed\todo{path to tests?}.

\section{Usage and User Testing}
We followed the interface of the GDB closely so that the users were familiar
with the debugger before even running it. However, we choose to diverge on some
of the features. For example, the GDB uses the \texttt{stepi} command for
single stepping. This form has a common prefix with the \texttt{step} command.
We, however, expect our users to use assembly-level debugging more often than
source-level debugging, so we choose the \texttt{istep} command form instead.
The two commands have no shared prefix, so typing \texttt{is} is enough.

Also, the program is run via \texttt{run}, this starts the VM, but the program
is paused. In GDB, one has to use the \texttt{start} command to get equivalent
behavior, \texttt{run} runs the program without stopping at the beginning. When
we did a brief user testing, it was not very clear to the user. However,
renaming the command \texttt{run} to \texttt{start} would cause it to have the
same prefix as \texttt{step}. Considering this, we've decided to leave the
command as \texttt{run}.

There are other minor things; most of them come from the fact that we
prioritize assembly-level debugging, whereas GDB focuses more on the source
level. We provide a short list of examples of commands that have different
syntax in GDB and in our debugger in the thesis attachment:
\texttt{impl/docs/debugging.md}.

One student already tested the debugger and said the experience was quite
pleasant. There were a few minor things that he did not like and offered
solutions for (for example, the usage message can only be displayed after the
debuggee program is executed, meaning the \texttt{run} command must be invoked
first), most of which we took to heart and corrected. Additionally, the student
that provided us with the TinyC compiler used the debugger to debug the code
his compiler generated. The NI-GEN students are also using the repository,
although no official feedback was collected because it is too early in the
semester

\section{Performance}\label{section:benchmark}
The performance of the debugged program can suffer. In this section, we show
which debugging features hurt performance the most and if there is any way
around it. The tests were done on two programs: a quicksort~\cite{quicksort}
algorithm and a naive prime number checker. The sources for these programs are
in the thesis attachments under path \texttt{benchmark/}. The difference
between those two programs is that the quicksort is recursive, whereas the
primes are implemented via a loop. The runtime of those two programs on the
virtual machine alone is similar.

\begin{table}[]
\centering
\begin{tabular}{||c c c||}
\hline
Test Case & Quicksort - Time & Prime numbers - Time \\
\hline\hline
1. & $9.96$            & $11.68$  \\
2. & $9.32$            & $12.33$  \\
3. & $10.9$            & $58.32$  \\
4. & $10.9$            & $72.85$  \\
5. & $9.73$            & $10.53$  \\
6. & $9.65$            & $230.81$ \\
\hline
\end{tabular}
\caption{Performance comparison when using various features of the debugger.
The Quicksort had $674$ breakpoints hits, while prime numbers had $101266$ hits.
    Each case was run five times and average was taken. The time is in seconds.}
\label{table:benchmark}
\end{table}

% Intel Core i5 8265U Whiskey Lake
% 4 cores, 1.6GHz, 3.9GHz under Core Boost
% 6 MB cache
% 8 GB RAM

We will measure the speed on the following cases:
\begin{enumerate}
    \item Run the VM without the debugger.
    \item Connect the debugger and immediately invoke the \texttt{continue} command.
    \item Connect the debugger and set breakpoint at hot spot of the program.
        For quicksort, this will be the main recursive function. For primes,
        it will be the body of the loop. When the breakpoint is hit invoke
        the \texttt{continue} command.
    \item Same as before, but at every breakpoint, hit read a $100$ cells of
        memory and the \texttt{IP} register.
    \item  Step over a computationally expensive function. For quicksort, it is
        the \texttt{quicksort} function; for primes, it is the \verb|is_prime|
        function.
    \item Set a breakpoint in the most expensive function and run the program.
        Remove the breakpoint on the first hit and step out of the function.
        The expensive functions are the same as in the previous test.
\end{enumerate}

The results are in table \ref{table:benchmark}. It is apparent that just having
the debugger connected introduces almost no slowdown. In the quicksort case,
the program is even faster. This is probably due to measurement errors. The
breakpoints, however, do cause a slowdown. In the quicksort, the program had
$674$ breakpoint hits, while in the prime numbers, it had $101266$. The
quicksort version is $1.1$ times slower, which is negligible. However, in the
prime numbers test performance takes a severe hit with being at least $5$ times
slower. This shouldn't come as a surprise, as there is an enormous number of
breakpoint hits. The communication between the debugger and the debuggee starts
to slow down the program. Given the circumstances, this result is still
satisfactory. Reading memory and registers together with the breakpoints
introduces minimal slowdown.

The step-over result is not very surprising. It puts a breakpoint after the
call and runs the program, so the result should be roughly the same as case
number one. Step out is stepping over until a return is encountered. This works
well in the recursive function because it skips a large part of the program.
But in the prime case, which is implemented via a loop, it severely slows down
the program because it essentially single steps through the entire thing.

The debugger is more than suited for usage on regular programs. In the case of
very computationally intensive programs, certain debugger features will start
to struggle. On the other hand, a hundred thousand breakpoint hits is an
unpractically large number. We consider the quicksort example as a peak of what
the students will have to debug, and in that example, the debugger had almost
no slowdowns. Still, it is something to keep in mind while using the debugger,
and there is a potential for improvements in the future.

\chapter{Conclusion}
\section{Summary}
We explored debugging capabilities of modern CPU architectures. We also
described how debugging is supported at various layers, from operating systems
to compilers. Then, we discuss the T86 architecture and its debugging support.
We remedy this by adding a debugging API inspired by modern architectures. We
also created a native and source-level debugger, which is production ready and
already used in the NI-GEN course at FIT CTU. This debugger is extensible
enough that if a new architecture, virtual machines, or source language comes
into play, it should be fairly easy to add debugging support.

The debugger encourages students in the NI-GEN course to investigate how the
debugger works, the connection between the generated machine code and the
source code, and to emit information about those connections so that the
debugger can work at the source level. It can also make their lives easier
since the debugger works at the native level without additional work, allowing
them to debug the code their compiler generated.

The tool is, along with the enhanced T86 virtual machine, openly available at \\  
\verb|https://github.com/Gregofi/t86-wdebug|.

\section{Future Work}
There are many possible improvements to be made. We have created a new
executable text format for T86 ISA and, consequently, a parser for that format.
Students, however, have to generate this text, which includes not only the
instructions themselves but also the debugging information. A builder interface
could go a long way, especially for the most commonly used languages in the
NI-GEN course.

The native part of the debugger is fairly complete. It can always be extended
for new architectures and virtual machines should they emerge. The expression
command, which evaluates a TinyC expression, cannot handle function calls.
There is also no way to format the output (for instance, to print an integer as
a hexadecimal number). The expression interpreter can always be extended to
handle other languages.

New types can always be added to the debugging information. For example, we are
missing enums, qualifiers like \texttt{const} and \texttt{mutable}, or more
advanced types that are in the C++ standard library. Those types are not in the
TinyC language that is being taught in the NI-GEN course, but the debugger is
not strictly tied to the language. Some parts of the program could be
optimized, like the step out which can prove slow, as demonstrated in
section~\ref{section:benchmark}.

The debugger currently does not handle frame information in any way. This could
be a helpful addition so that it displays call frames and allows one to step
out of them. This would require a new section, which would have to contain
enough information to simulate an unwind. The exact mechanism was described in
section~\ref{section:call-frames}.

The T86 VM generates statistics about the program execution. It records things
like pipeline stalls. Since the debugger has the capability to connect the
source to the assembly, it could display the code hot spots in the source code
directly.

Last but not least, a graphical user interface could be created for the
debugger. Currently, it only offers a command line interface. A graphical
interface could be more pleasant to work in. Alternatively, it could be hooked
to an existing editor, like the Visual Studio Code, which allows the usage of
other debuggers like LLDB or GDB.


% WARNING: If something doesn't work, put these guys back
\appendix\appendixinit % do not remove these two commands

\chapter{GDB to T86 Debugger Command Map}
\renewcommand{\arraystretch}{1.2}
\begin{table}[h]
\centering
    \begin{tabular}{p{5cm}ll}
        \hline
Usecase                                                  & GDB                                    & T86 DBG                                     \\
        \hline
Set a breakpoint at line 5                           & \texttt{br 5}                                   & \texttt{br s 5}                                      \\
Run the process and immediately stop                     & \texttt{start}                                  & \texttt{run}                                        \\
\multirow{2}{=}{Do an instruction level single step}     & \multirow{2}{*}{\texttt{stepi}}                 & \texttt{istep}                                      \\
                                                         &                                        & \texttt{is}                                         \\
Set a breakpoint on a function named main                & \texttt{break main}                             & \texttt{br s main}                                   \\
List all breakpoints                                     & \texttt{info break}                             & \texttt{br list}                                     \\
\multirow{2}{=}{Show the contents of a variable named a} & \multirow{2}{*}{\texttt{p a}}                   & \texttt{p a}                                         \\
                                                         &                                                 & \texttt{expr a}                                      \\
Disable a breakpoint                                     & \texttt{disable <bp-id>} & \texttt{br disable <bp-addr>} \\
List values of registers                                 & \texttt{info registers}                         & \texttt{register}                              \\
Display value stored in the instruction pointer          & \texttt{info registers rip}                     & \texttt{reg get IP}\\
        \hline
\end{tabular}
\caption{List of examples of how some actions can be achieved in the GDB debugger and in
the debugger presented by this thesis.}
\label{table:gdb-vs-dbg}
\end{table}
 % include `appendix.tex' from `text/' subdirectory

\backmatter % do not remove this command

\printbibliography % print out the BibLaTeX-generated bibliography list

\chapter{Contents of enclosed media}


	\dirtree{%
		.1 readme.txt\DTcomment{Overview of the medium contents}.
		.1 impl\DTcomment{Source code of the implementation}.
        .1 linux-debugger\DTcomment{Source code of the proof of concept Linux debugger}.
		.1 thesis\DTcomment{Source code of the text in \LaTeX{}}.
		.1 text\DTcomment{text}.
		.2 thesis.pdf\DTcomment{text in the PDF format}.
	}
 % include `medium.tex' from `text/' subdirectory

\end{document}
