\chapter{Conclusion}
\section{Summary}
We explored debugging capabilities of modern CPU architectures. We also
described how debugging is supported at various layers, from operating systems
to compilers. Then, we discuss the T86 architecture and its lackluster
debugging support. We remedy this by adding a debugging API inspired by modern
architectures. We also created a native and source-level debugger, which is
production ready and already used in the NI-GEN course at FIT CTU. This
debugger is extensible enough that if a new architecture, virtual machine, or
source language comes into play, it should be fairly easy to add debugging
support.

The debugger encourages students in the NI-GEN course to investigate how the
debugger works, the connection between the generated machine code and the
source code, and to emit information about those connections so that the
debugger can work at the source level. It can also make their lives easier
since the debugger works at the native level without additional work, allowing
them to debug the code their compiler generated.

The tool is, along with the enhanced T86 virtual machine, openly available at \\  
\verb|https://github.com/Gregofi/tinyverse-monorepo|.

\section{Future Work}
There are many possible improvements to be made. We have created a new
executable text format for T86 ISA and, consequently, a parser for that format.
Students, however, have to generate this text, which includes not only the
instructions themselves but also the debugging information. A builder interface
could go a long way, especially for the most commonly used languages in the
NI-GEN course.

The native part of the debugger is fairly complete. It can always be extended
for new architectures and virtual machines should they emerge. The source-level
debugger can, however, be extended. The expression command, which evaluates a
TinyC expression, can't handle function calls. There is also no way to format
the output (for instance, to print an integer as a hexadecimal number).

New types can always be added to the debugging information. For example, we're
missing enums, qualifiers like \texttt{const} and \texttt{mutable}, or more
advanced types that are in the C++ standard library. Those types are not in the
TinyC language that is being taught in the NI-GEN course, but the debugger is
not strictly tied to the language. Some parts of the program could be
optimized, like the step out which can prove slow, as demonstrated in
section~\ref{section:benchmark}.

The debugger currently does not handle frame information in any way. This could
be a helpful addition so that it displays call frames and allows one to step
out of them. This would require a new section, which would have to contain
enough information to simulate an unwind. The exact mechanism was described in
section~\ref{section:call-frames}.

The T86 VM generates statistics about the program execution. It records things
like pipeline stalls. Since the debugger has the capability to connect the
source to the assembly, it could display the code hot spots in the source code
directly.

Last but not least, a graphical user interface could be created for the
debugger. Currently, it only offers a command line interface. A graphical
interface could be more pleasant to work in. Alternatively, it could be hooked
to an existing editor, like the Visual Studio Code, which allows the usage of
other debuggers like LLDB or GDB.
