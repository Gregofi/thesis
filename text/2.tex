\chapter{Tiny x86}\label{section:T86}
At the FIT CTU, in the NI-GEN course, students have to write a compiler. The
Tiny x86 (T86) architecture was created to make code generation easier. This
allows the students to focus on more interesting parts of compiler design, such
as optimizations. To be able to execute programs written using this
architecture, a virtual machine was created. This was all done as a part of a
master's thesis made by Ivo Strejc~\cite{ivo2021tiny}. This chapter explores
said architecture and the virtual machine. It also delves into the existing
debugging capabilities of said virtual machine.

\section{The T86 Instruction Set Architecture}
The primary goal of the T86 architecture is to be an educational one. The
Virtual Machine made for the ISA allows configuring the number of registers,
the RAM size, or the length of each instruction. This allows the students to
develop their compilers incrementally.

The T86 uses a Harvard architecture, meaning the data and instructions are
physically separated. Memory is addressable by 64-bit blocks, not by 8-bit
blocks, as is the custom in modern computers. It shares some of the registers
we saw on x86-64: the program counter (PC), the stack pointer (SP), the base
pointer (BP), and the flags register. The intended roles for these registers
are the same as on x86-64 architecture. It also has other general-purpose
registers. As previously said, the amount of these is configurable. The
registers stores 64-bit values. It also has float registers, which can store
64-bit float values. These are separated from the standard registers, similarly
to x86-64.

The addressing modes, or what kind of operands instructions can have, include
immediate values, registers, and memory accesses. They can also be combined in
various ways, like \texttt{[R0 + R1 * 2]}, or \texttt{[R0 + 10 + R1 * 2]}. The
addressing modes are, however, not arbitrary, \texttt{[R0 + R1 + R2]} is not a
correct addressing mode for any instruction in the T86 architecture. The
allowed addressing modes also change for each instruction. For example, the
\texttt{MOV} instruction allows a vast range of addressing modes, while the
\texttt{PUSH} instruction only allows a register and an immediate number. For a
full list, refer to~\cite{ivo2021tiny}. The instructions that are taken over
from x86-64 often have more restrictive addressing modes than in x86-64. For
example, the \verb|add| instruction can take a memory offset as a destination
operand in x64-86, but in T86, it can only take registers.

Other than that, the ISA is mainly a subset of the x86-64 most used
instructions, many of which we have already seen in various examples throughout
the thesis. Interesting exceptions are the IO instructions - \texttt{PUTCHAR}
and \texttt{GETCHAR}, which allow for very primitive input and output handling.
Also, a \texttt{DBG} and \texttt{BREAK} instructions are defined. These are
used for debugging, but in a very different way than we have seen in previous
sections. We will touch upon them when discussing the virtual machine
implementation since they are very much tied to it. A small sample of a T86
program can be seen in figure~\ref{fig:t86-example}.

\begin{figure}
    \begin{lstlisting}
MOV  R0, 1
MOV  R1, 50
ADD  R0, R1
MUL  R0, 5
HALT
    \end{lstlisting}
    \caption{A small example of a program in the T86 architecture.}
    \label{fig:t86-example}
\end{figure}

\section{T86 Virtual Machine}\label{section:t86-vm}
The primary objective of the virtual machine is to replicate the CPU as
accurately as possible, without prioritizing execution speed. For instance, the
virtual machine simulates the out-of-order technique briefly described in
section~\ref{section:superscalar-cpu}. The purpose of the virtual machine is to
allow the students to gain a deeper understanding of the effects of pipeline
stalls and similar events on program speed. The virtual machine is able to
generate statistics that provide information about these factors and how
much they influenced the speed of the generated program.

The virtual machine (VM) is implemented in C++, using the newer standards up to
C++17. The VM offers only a single interface, and that is the
\texttt{ProgramBuilder}. This is a class through which one may construct a
program for the T86 VM. An example of how to use this class is in
figure~\ref{fig:t86-intro}. Currently, there is no other way for users to run
programs in the VM. This means that the students are tied to the C++ language,
or use some bindings if they want to use other language.

The \texttt{Cpu} class is the backbone of the interpreter. It is responsible
for running the program. It has a \texttt{halted} function, which returns true
if the \texttt{Cpu} executed a \texttt{HALT} instruction. The \texttt{tick}
performs one tick of the CPU. This does not mean that one instruction gets
executed. The \texttt{Cpu} simulates a superscalar CPU, so one tick is one move
in the pipeline. The \texttt{while} loop in the figure~\ref{fig:t86-intro}
shows to run a T86 program via the \texttt{Cpu} class.

\begin{figure}
    \begin{minted}{cpp}
        ProgramBuilder pb;
        pb.add(MOV{Reg(0), 50);
        pb.add(PUTCHAR{Reg(0)});
        pb.add(HALT{});
        auto program = pb.program();
        Cpu cpu;
        cpu.start(std::move(program));
        while (!cpu.halted()) {
            cpu.tick();
        }
    \end{minted}
    \caption{Simple example of how to create and run a simple program using the
    T86 virtual machine.}
    \label{fig:t86-intro}
\end{figure}

\subsubsection{Debug Instructions}\label{section:t86-debug-cap}
The VM offers some limited debugging capabilities. It has the \texttt{DBG} and
\texttt{BRK} instructions. The \texttt{DBG} instruction takes as an operand a
function. The function has the following signature: \texttt{void fun(Cpu\&)}.
This function then gets executed when the instruction is hit. This can prove
helpful in inspecting the internal state of the CPU. In
figure~\ref{fig:t86-debug}, we show a possible usage of this instruction. The
\texttt{BRK} instruction works similarly. It, however, has no operand. Instead,
a function must be provided before execution to the CPU itself. \texttt{BRK}
then always runs this function when hit.

\begin{figure}
    \begin{minted}{cpp}
    pb.add(DBG{[](Cpu& cpu) {
        if (cpu.getRegister(Reg{0}) == 0) {
            std::cerr << "Register 0 is set to zero!\n";
        }
    });
    \end{minted}
    \caption{Adding a \texttt{DBG} instruction into the T86 program using the
    \texttt{ProgramBuilder}.}
    \label{fig:t86-debug}
\end{figure}

Such debugging capabilities can be helpful but quickly prove insufficient. For
example, when a step-by-step inspection is sought, a debug instruction must be
placed at every second line of the program. Also, interactivity is not present.
However, this function can accept input, so one could create a robust enough to
handle register and memory writing. An idea of how this could be done is
illustrated in~\ref{fig:t86-pocket-debugger} via the \texttt{BRK} instruction.

\begin{figure}
    \begin{minted}{cpp}
    cpu.connectBreakHandler([](Cpu& cpu) {
        char command;
        std::cin >> command;
        if (command == 'c') return; // continue
        else if (command == 'r') { // Read register
            int num;
            std::cin >> num;
            int regval = cpu.getRegister(num);
            std::cerr << std::format("Register {} = {}\n",
                                     num, regval);
        } else if (command == 'w') { // Write register
            int num;
            int val;
            std::cin >> num >> val;
            cpu.setRegister(Reg{num}, Reg{val});
        }
        ... // Other commands
    });
    \end{minted}
    \caption{Small debugger implementation using T86 \texttt{BRK} instruction,
    abbreviated.}
    \label{fig:t86-pocket-debugger}
\end{figure}

This still leaves much to be desired. Not to mention placing the debug
instruction can prove very bothersome.
