\chapter{Introduction}
At the core of every computer program lies the Central Processing Unit (CPU),
which is responsible for executing programs. The CPU excels at performing very
primitive operations very fast. These operations are called instructions, they
can perform simple arithmetics, move values from and to memory, and change the
control flow of the program. The instructions are encoded as a sequences of
binary numbers which are easy for the CPU to understand, but for humans they
are rather incomprehensible.

To help programmers better understand written programs, a text mapping was
created, called \textit{assembly language}. Each instruction is now assigned an
text representation, as are the operands of the instruction. The control flow
instructions now doesn't have to jump to address offset, but can instead use
labels. Example of a simple program in assembly language can be seen in figure
\ref{fig:simple-assembly}. If a programmer knows the instruction set
architecture~\cite{aps-isa} of the processor he can easily recognize the
instructions the program is made of.

\begin{figure}
    \begin{minipage}{.45\textwidth}
    \begin{lstlisting}
positive:
        push    rbp
        mov     rbp, rsp
        mov     -4[rbp], edi
        cmp     -4[rbp], 0
        jle     neg
        mov     eax, 1
        jmp     pos
neg:
        mov     eax, 0
pos:
        pop     rbp
        ret
    \end{lstlisting}
    \end{minipage}
    \hfill\vline\hfill
    \begin{minipage}{.45\textwidth}
    \begin{lstlisting}
   01010101 01001000
   10001001 11100101
   10001001 01111101
   11111100 10000011
   01111101 11111100
   00000000 01111110
   00000111 10111000
   00000001 00000000
   00000000 00000000
   11101011 00000101
   10111000 00000000
   00000000 00000000
   00000000 01011101
   11000011
    \end{lstlisting}
    \end{minipage}
    \caption{An example of a program that checks if a number is positive, shown
    in both assembly language and machine code.}
    \label{fig:simple-assembly}
\end{figure}

But, as the computers became increasingly more powerful, so did the programs
became bigger and more complex. When programming in the assembly language,
the programmer must have an extensive knowledge of the processor's internal
workings.

To spare the programmers from this, high level programming languages were
created. These are designed to abstract from the specific machine the program
will run on, allowing programmers to focus more on their task. As shown in
figure \ref{fig:c-positive}, even a simple program written in the C programming
language, one of the oldest programming language around, provides a clear
understanding of the functionality. In contrast, examining the equivalent
program in assembly language, as seen in figure \ref{fig:simple-assembly},
requires knowledge of the specific architecture of the machine.

\begin{figure}
\begin{minted}[
]{c}
bool positive(int n) {
    if (n > 0) {
        return true;
    } else {
        return false;
    }
}
\end{minted}
\caption{A function written in the C programming language that check if a
    number is positive.}
\label{fig:c-positive}
\end{figure}

However, we previously said that processors can only understand machine code
and high level languages are far from it. Therefore, a translation of the
high-level language into machine code is necessary, this is a task for a
\textit{compiler}. A compiler is a program that reads some source code of
high-level language and produces machine code. Compilers are a very complicated
piece of software, and we will talk about them in detail in chapter
\ref{section:source-level-debugging}. For now, it is important to uderstand
that the computer cannot run the source code of a high-level language directly,
and that it is translated into machine code. Additionaly, compilers often take
advantage of special features of the architecture to make the programs
faster~\cite{dragon-book}.

\begin{figure}[H]
    \begin{minted}[
            linenos,
]{c}
int binary_search(int* arr, int len, int n) {
    int lo = 0;
    int hi = len;
    while (lo < hi) {
        int i = (lo + hi)/2;
        if (arr[i] < n) {
            lo = i;
        } else if (arr[i] > n) {
            hi = i;
        } else {
            return 1;
        }
    }
    return 0;
    \end{minted}
    \caption{Binary search algorithm written in the C programming language.}
    \label{fig:binary-search}
\end{figure}

In the figure \ref{fig:binary-search} we present a more complicated example of
a program written in high level programming language. This is an implementation
of the binary search algorithm. As an input, it receives sorted sequence of
numbers and some number $n$ and checks if that number $n$ is in the sequence.
This algorithm is widely used when searching in sorted sequence because of its
$\mathcal{O}(\text{log}_2(n))$ complexity~\cite{pruvodce}.

\section{Debugging}
Programs are mostly written by humans, who tend to make
mistakes~\cite{human-error}. We are no exception, as we have also made a
mistake in the binary search program. Let us try to run the program with a
\texttt{[1,2,3]} sequence and search for the number $4$. This number is not in
the sequence, so the expected output would be $0$. Instead, if we ran the
program it would run forever, because of a mistake we made in the source code.
Such mistakes are called a \textit{bug}\footnote{The term \textit{bug}
actually comes from a real bug that got stuck in relays back when computers
were made from relays. They literally had to debug the machine by taking the
bug out.}. The process of finding these mistakes and correcting them is called
\textit{debugging}~\cite{art-of-testing}.

We should fix this bug. Otherwise, users of our program will probably not be
happy if their program gets stuck. We could try to look at the source code and
find the mistake this way. This is a valid strategy. Here we could assume that
the condition \texttt{lo < hi} never comes to be since it is the most obvious
place where we could get stuck forever. Now, it would be helpful if we could
see the states of \texttt{lo} and \texttt{hi} in each iteration of the cycle.
We could resort to print statements, but that is not very flexible. If we
changed our minds and wanted to also see the value of variable \texttt{i}, we
would have to recompile the program and rerun it. The output can also quickly
get overwhelming, especially in an infinite loop. However, all is not lost, as
a special program was created to inspect running programs, called a
\textit{debugger}.

Debugger is able to inspect the state of another program, like the values of
its variables. It is also able to control the flow of the program. They allow
\textit{breakpoints} to be set at each line of the source
code\footnote{Advanced debuggers allow breakpoints to be set inside
expressions. This is especially important for functional languages, as their
functions often consist of one big expression.}. When the program is about to
execute the line of code with the breakpoint, the control is passed back to the
debugger and the user can inspect the state of the program at that line. There
are also conditional breakpoints, which only trigger when some condition holds.
Example of such condition can be that the breakpoint gets activated only when
\texttt{i == 3}.

Finally, debuggers also allow \textit{stepping}. This also modifies the control
flow of the program.
\begin{itemize}
    \item \textit{step in} - Executes current statement and stops on the next
        one. If the current statement is a function call then it will be
        executed and the program will be paused on the first statement in that
        function.
    \item \textit{step over} - Same as a step in, but if the current statement
        is a function call then the program will be paused on the next
        statement after the call.
    \item \textit{step out} -  Executes as much as needed to return from the
        current function. Stops on the next statement that should be executed
        after the function returns.
\end{itemize}

Now, back to our program. Let us put a breakpoint on line $8$ after $i$ is set.
We will monitor how the \texttt{lo} and \texttt{hi} change. If you run the
program with the debugger attached, you will see an output similar to what is
displayed in figure \ref{fig:lldb-debug1}. Here, you can see the line on which
the execution was stopped. It is also possible to print the state of variables.
In each loop, you can print the value of a variable and then continue execution
until another breakpoint. There is only one set, so the execution will again be
stopped on line $8$. The value of \texttt{hi} will not change, which is
expected. Value of \texttt{lo} will gain following values: $0, 1, 2, 2, 2,
\dots$. The value apparently gets stuck at $2$ and never change. The value of
variable $i$ is computed as $i = (\text{lo} + \text{hi})/2 = (2 + 3)/2 = 2$,
because division in C rounds the value down. The fix is to change the line $9$
to \texttt{lo = i + 1}. With the debugger, it was simple to find out where the
error came from and we didn't have to recompile the program.

\begin{figure}
\begin{minted}{c}
   5   	    int hi = len;
   6   	    while (lo < hi) {
   7   	        int i = (lo + hi)/2;
-> 8   	        if (arr[i] < n) {
   9   	            lo = i;
   10  	        } else if (arr[i] > n) {
   11  	            hi = i;
Target 0: (a.out) stopped.
> p lo
(int) $0 = 0
> p hi
(int) $1 = 3
\end{minted}
    \caption{Example of debugging the binary search algorithm in the
    LLDB~\cite{lldb} debugger.}
    \label{fig:lldb-debug1}
\end{figure}

We previously said that processors themselves only understand machine code. So
how it is possible that we can debug the program and the debugger knows about
lines, variables, etc. when the program itself is just machine code. The
compiler has to lend a hand here. It embeds information about the source code.
For example, it maps lines of source code to machine code instructions. Thanks
to this mapping, the debugger knows that line $x$ corresponds to instruction
$y$ in the machine code and can put a breakpoint there. If the compiler doesn't
emit any information into the executable the debugger would only work with
assembly, as seen in figure \ref{fig:lldb-debug2}. This is a lot more
discomforting than debugging source code directly.

\begin{figure}
\begin{lstlisting}
->  0x100003e3c <+112>: b      0x100003e78               ; <+172>
    0x100003e40 <+116>: ldr    x8, [sp, #0x20]
    0x100003e44 <+120>: ldrsw  x9, [sp, #0xc]
    0x100003e48 <+124>: ldr    w8, [x8, x9, lsl #2]
\end{lstlisting}
\caption{Example of debugging a program in the LLDB debugger without debugging
    information generated by the compiler.}
\label{fig:lldb-debug2}
\end{figure}

\section{Teaching Compilers}
Many schools about computer science have a compiler course, and the Faculty of
Information Technology, CTU\todo{cite?} is no exception. The course is called
\textit{Code Generators} (NI-GEN). In this course, students are tasked to build
a simple compiler from a C-like language called TinyC. The target of the
compiler is the Tiny x86 (T86) architecture. This architecture does not have a
processor that implements it. Instead, a virtual machine, a program that reads
the assembly and executes it, was created for it. The architecture is supposed
to ease the code generation and let the students focus on the more interesting
parts of the compiler, like register allocation or optimization, instead of the
nitty gritty detail of real CPU architectures.

There is, however, a problem with using T86, as it has almost non-existing
debugging support. So if a compiler of some student generates the code badly,
which is frankly inevitable, it takes a non-trivial amount of effort to find
the error. T86 has some very light debugging capability, but it is far from
real debugging. Also, compiling debugging information is not taught in the
NI-GEN course because there is no reason to as of now. If a debugger was
provided to the students, it might be incentivizing to compile such information
to ease their lives later, when they will need to find errors in their
compilers. This way, they will also learn how and what information the compiler
needs to embed for the debugger to work.

\section{Goals of the Thesis}
The primary goal is to add debugging support to the T86 and to create a
debugger that supports debugging both on the machine code level and on the
source code level. The debugger should be extensible enough to also work with
an intermediate representation. The debugger should be similar to real-world
debuggers, in terms of how it works. This will require non-trivial changes in
the T86 virtual machine source code. The students' compilers will also have to
generate debugging information. The format of the debugging information should
be so that it is not discouraging for students to generate but also comparable
to debugging information generated by real compilers.

\section{Structure of the Thesis}
\begin{enumerate}
    \item The \textit{Introduction} is the motivation behind the thesis and
        introduces basic terms with which should the reader be familiar.
    \item \textit{Debugging support} describes how are debuggers implemented
        and what debugging support is provided on various levels (OS,
        processors, compilers) to make their implementation possible.
    \item \textit{Tiny x86} describes the T86 architecture and discusses some
        of the parts of the virtual machine, mainly its existing debugging
        capabilities.
    \item \textit{Implementation} focuses on extending the T86 instruction set
        and adding a debugging interface to the virtual machine. It also
        describes the implementation of the debugger and the format of the
        debugging information chosen.
    \item \textit{Evaluation} evaluates the performance of the debugger and its
        ease of use.
    \item \textit{Conclusions} summarizes the result of the thesis and speaks
        of possible future work.
\end{enumerate}
