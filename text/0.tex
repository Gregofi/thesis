\chapter{Introduction}
At the core of every computer program lies the Central Processing Unit (CPU),
which is responsible for executing programs. The CPU excels at performing very
primitive operations very fast. These operations, called instructions, perform
simple arithmetics, move values from and to memory, and change the control flow
of the program. They are encoded as a sequence of binary numbers, which are
easy for the CPU to understand, but are rather incomprehensible for humans.

To help programmers better understand written programs, a text mapping was
created called \textit{assembly language}. Each instruction is assigned a text
representation, as are the operands of the instruction. The control flow
instructions do not have to jump to an address offset but can instead use
labels. An example of a simple program in both an assembly language and a
machine code can be seen in figure~\ref{fig:simple-assembly}. If a programmer
knows the instruction set architecture of the processor, they can easily
recognize the instructions the program is made of.

\begin{figure}
    \begin{subfigure}{1.0\textwidth}
        \begin{lstlisting}
01010101 01001000
10001001 11100101
10001001 01111101
11111100 10000011
01111101 11111100
00000000 01111110
00000111 10111000
00000001 00000000
00000000 00000000
11101011 00000101
10111000 00000000
00000000 00000000
00000000 01011101
11000011
        \end{lstlisting}
        \caption{A machine code program.}
        \label{subfig:machine-code}
    \end{subfigure}
    \begin{subfigure}{1.0\textwidth}
        \begin{lstlisting}
positive:
        push    rbp
        mov     rbp, rsp
        mov     -4[rbp], edi
        cmp     -4[rbp], 0
        jle     neg
        mov     eax, 1
        jmp     pos
neg:
        mov     eax, 0
pos:
        pop     rbp
        ret
        \end{lstlisting}
        \caption{An assembly program.}
        \label{subfig:assembly}
    \end{subfigure}\vspace{0.5cm}
    \begin{subfigure}{1.0\textwidth}
        \begin{minted}[]{c}
bool positive(int n) {
    if (n > 0) {
        return true;
    } else {
        return false;
    }
}
        \end{minted}
        \caption{A program in the C language.}
        \label{subfig:c}
    \end{subfigure}
    \vspace{0.5cm}
    \caption{An example of a program that checks if a number is positive, shown
    in an assembly language, in a machine code, and in the C programming
    language.}
    \label{fig:simple-assembly}
\end{figure}

However, as computers became increasingly more powerful, so did the programs
became bigger and more complex. When programming in the assembly language, the
programmer must have extensive knowledge of the processor's internal workings.

To spare the programmers from this, high-level programming languages were
created. These are designed to abstract from the specific machine the program
will run on, allowing programmers to focus more on their tasks. As shown in
figure~\ref{subfig:c}, even a simple program written in the C programming
language~\cite{krc}, one of the oldest programming languages around, provides a
clear understanding of the functionality. In contrast, examining the equivalent
program in assembly language, as seen in figure~\ref{subfig:assembly}, requires
knowledge of the specific architecture of the machine. High-level languages
also introduce control flow statements, which makes the code easier to follow
compared to assembly jumps~\cite{gotobad}.

But processors only understand machine code and high-level languages are far
from it. Therefore, a translation of a high-level language program into a
machine code program is necessary. This is a task for a \textit{compiler}. A
compiler is a program that reads a source code of high-level language and
produces machine code. Compilers are highly complex software; we will discuss
them in detail in chapter~\ref{section:source-level-debugging}. For now, it is
crucial to understand that the computer cannot directly run the source code of
a high-level language and that it is translated into machine code.
Additionally, compilers often take advantage of unique features of the
architecture to make the programs faster~\cite{dragon-book}.

\section{Debugging}
\begin{figure}
    \begin{minted}[
            linenos,
]{c}
int binary_search(int* arr, int len, int n) {
    int lo = 0;
    int hi = len;
    while (lo < hi) {
        int i = (lo + hi)/2;
        if (arr[i] < n) {
            lo = i;
        } else if (arr[i] > n) {
            hi = i;
        } else {
            return 1;
        }
    }
    return 0;
}
    \end{minted}
    \caption{Implementation of the binary search algorithm written in the C
    programming language.}
    \label{fig:binary-search}
\end{figure}

In figure \ref{fig:binary-search}, we present a more complicated example of a
program written in a high-level programming language. This is an implementation
of the binary search algorithm. As an input, it receives a sorted sequence of
numbers and some number $n$ and checks if that number $n$ is in the sequence.
This algorithm is widely used when searching in sorted sequence because of its
$\mathcal{O}(\text{log}_2(n))$ complexity~\cite{pruvodce}.

Programs are mostly written by humans, who tend to make
mistakes~\cite{human-error}. We are no exception, as we have also made a
mistake in the binary search program. Let us try to run the program with a
\texttt{[1,2,3]} sequence and search for the number $4$. This number is not in
the sequence, so the expected output would be $0$. Instead, if we ran the
program, it would run forever because of a mistake we made in the source code.
Such mistakes are called \textit{bugs}\footnote{The term \textit{bug} actually
comes from an actual bug that got stuck in relays back when computers were made
from relays. They literally had to debug the machine by taking the bug out.}.
The process of finding these mistakes and correcting them is called
\textit{debugging}~\cite{art-of-testing}.

There are several approaches to debugging. We could try to look at the source
code and find the mistake this way\footnote{Do note that this approach does not
scale well with bigger programs.}. Here we could assume that the condition
\texttt{lo < hi} never comes to be since it is the most obvious place where we
could get stuck forever. Now, it would be helpful if we could see the states of
\texttt{lo} and \texttt{hi} in each iteration of the cycle. We could resort to
print statements, but that is not very flexible. If we changed our minds and
wanted to also see the value of variable \texttt{i}, we would have to recompile
the program and rerun it. The output can also quickly get overwhelming,
especially in an infinite loop. However, all is not lost, as a special program
was created to inspect running programs, called a \textit{debugger}.

Debugger is able to inspect the state of another program, like the values of
its variables. It is also able to control the flow of the program. They allow
\textit{breakpoints} to be set at each line of the source
code\footnote{Advanced debuggers allow breakpoints to be set inside
expressions. This is especially important for functional languages, as their
functions often consist of one big expression.}. When the program is about to
execute the line of code with the breakpoint, the control is passed back to the
debugger, and the user can inspect the state of the program at that line. There
are also conditional breakpoints, which only trigger when some condition holds.
An example of such a condition can be that the breakpoint gets activated only
when \texttt{i == 3}.

Finally, debuggers also allow \textit{stepping}. There are several kinds of
steps:
\begin{itemize}
    \item \textit{step-in} - Executes the current statement and stops on the
        next one. If the current statement is a function call, it will be
        executed, and the program will be paused on the first statement in that
        function.
    \item \textit{step-over} - Same as a step-in, but if the current statement
        is a function call, the program will be paused on the next statement
        after the call.
    \item \textit{step-out} - Executes as much as needed to return from the
        current function. Stops on the next statement that should be executed
        after the function returns.
\end{itemize}

To illustrate the workings of the debugger, let us look back at the program in
figure~\ref{fig:binary-search}. We will place a breakpoint on line $8$ after
$i$ is set and we will monitor how the values in variables \texttt{lo} and
\texttt{hi} change. If the program is run with the debugger attached, an output
similar to what is displayed in figure~\ref{fig:lldb-debug1} will be seen.
Here, it is possible to see the line on which the execution was stopped. It is
also possible to print the state of variables. In each loop, we could print the
value of a variable and then continue execution until another breakpoint is
hit. If we continue, the execution will again be stopped on line $8$. The value
of \texttt{hi} will not change, which is expected. Variable \texttt{lo} will
gain the following values: $0, 1, 2, 2, 2, \dots$ It apparently gets stuck at
$2$. The value of variable $i$ is computed as $i = (\text{lo} + \text{hi})/2 =
(2 + 3)/2 = 2$, because division in C rounds the value down. The fix is to
change the line $9$ to \texttt{lo = i + 1}. With the debugger, it was simple to
find out where the error came from, and we did not have to recompile the
program.

\begin{figure}
\begin{minted}{c}
   5   	    int hi = len;
   6   	    while (lo < hi) {
   7   	        int i = (lo + hi)/2;
-> 8   	        if (arr[i] < n) {
   9   	            lo = i;
   10  	        } else if (arr[i] > n) {
   11  	            hi = i;
Target 0: (a.out) stopped.
> p lo
(int) $0 = 0
> p hi
(int) $1 = 3
\end{minted}
    \caption{Debugging session in the LLDB~\cite{lldb} debugger, showing
    breakpoint hit report and printing variable values.}
    \label{fig:lldb-debug1}
\end{figure}

We previously mentioned that processors themselves only understand machine
code. Hence, the question arises as to how the debugger can know about lines,
variables, and similar traits of the source code when the program itself is
just machine code. The compiler has to lend a hand here. It embeds information
about the source code, either into the executable itself or into a separate
file. For example, it maps lines of source code to machine code instructions.
Thanks to this mapping, the debugger knows that line $x$ corresponds to
instruction $y$ in the machine code and can put a breakpoint there. If the
compiler does not emit any information into the executable, the debugger would
only work with assembly, as seen in figure~\ref{fig:lldb-debug2}. This can be
very difficult for the programmer to work with compared to debugging the source
code directly.

\begin{figure}
\begin{lstlisting}
->  0x100003e3c <+112>: b      0x100003e78               ; <+172>
    0x100003e40 <+116>: ldr    x8, [sp, #0x20]
    0x100003e44 <+120>: ldrsw  x9, [sp, #0xc]
    0x100003e48 <+124>: ldr    w8, [x8, x9, lsl #2]
\end{lstlisting}
\caption{Example of debugging a program in the LLDB debugger without debugging
    information generated by the compiler.}
\label{fig:lldb-debug2}
\end{figure}

\section{Teaching Compilers}
Many schools about computer science have a compiler course, and the Faculty of
Information Technology, CTU, is no exception. The course is called \textit{Code
Generators} (NI-GEN). In this course, students are tasked to build a simple
compiler from a C-like language called TinyC. The target of the compiler is the
Tiny x86 (T86) architecture. This architecture does not have a processor that
implements it. Instead, a virtual machine, a program that reads the assembly
and executes it, was created for it. The architecture is supposed to ease the
code generation and let the students focus on the more interesting parts of the
compiler, like register allocation or optimization, instead of the
uninteresting details of real CPU architectures.

There is, however, a problem with using T86, as it has almost non-existing
debugging support. So if a compiler of some student generates the code badly,
which is frankly inevitable, it takes a non-trivial amount of effort to find
the error. T86 has some very light debugging capability, but it is far from
real debugging. Also, compiling debugging information is not taught in the
NI-GEN course because there is no reason to as of now. If a debugger was
provided to the students, it might be incentivizing to compile such information
to ease their lives later when they need to find errors in their compilers.
This way, they will also learn how and what information the compiler needs to
embed for the debugger to work.

\section{Goals of the Thesis}
The primary goal is to add debugging support to the T86 and create a debugger
that supports debugging both on the machine and source code levels. The
debugger should be extensible enough to also work with an intermediate
representation. The debugger should be similar to real-world debuggers in terms
of how it works. This will require non-trivial changes in the T86 virtual
machine source code. The students' compilers will also have to generate
debugging information. The format of the debugging information should be so
that it is not discouraging for students to generate but also comparable to
debugging information generated by real compilers.

\section{Structure of the Thesis}
\begin{enumerate}
    \item The \textit{Introduction} is the motivation behind the thesis and
        introduces basic terms with which should the reader be familiar.
    \item \textit{Debugging Techniques} describes how are debuggers implemented
        and what support is required on various levels (OS, processors,
        compilers) to make their implementation possible.
    \item \textit{Tiny x86} describes the T86 architecture and discusses some
        of the parts of the virtual machine, mainly its existing debugging
        capabilities.
    \item \textit{Implementation} focuses on extending the T86 instruction set
        and adding a debugging interface to the virtual machine. It also
        describes the implementation of the debugger and the chosen format of
        the debugging information.
    \item \textit{Evaluation} evaluates the performance of the debugger and its
        ease of use.
    \item \textit{Conclusions} summarizes the result of the thesis and speaks
        of possible future work.
\end{enumerate}
